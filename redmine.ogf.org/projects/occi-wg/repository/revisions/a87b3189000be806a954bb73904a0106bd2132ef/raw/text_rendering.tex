\documentclass[10pt,a4paper]{article}
\usepackage[utf8]{inputenc}
\usepackage[english]{babel}
\usepackage[activate={true,nocompatibility},final,tracking=true,kerning=true,spacing=true]{microtype}
\usepackage[plainpages=false,pdfpagelabels,unicode]{hyperref}
\usepackage{fullpage}
\usepackage{graphicx}
\usepackage{fancyhdr}
\usepackage{comment}
\usepackage{occi}
\usepackage{lineno}   % adds line numbers, may be removed for non draft versions
\linenumbers          % adds line numbers, may be removed for non draft versions
\usepackage{verbatim} % adds verbatim options
\usepackage{tabularx} % adds extended tabular formatting options
\usepackage{listings}
\usepackage{color}
\definecolor{lightgray}{rgb}{.9,.9,.9}
\definecolor{darkgray}{rgb}{.4,.4,.4}
\definecolor{purple}{rgb}{0.65, 0.12, 0.82}

\lstdefinelanguage{json}{
  ndkeywords={String, Number, Boolean, Null, Object, Array},
  ndkeywordstyle=\itshape
}
\lstset{
   language=json,
   basicstyle=\footnotesize,
}

\setlength{\headheight}{13pt}
\pagestyle{fancy}

% default sans-serif
\renewcommand{\familydefault}{\sfdefault}

% no lines for headers and footers
\renewcommand{\headrulewidth}{0pt}
\renewcommand{\footrulewidth}{0pt}

% header
\fancyhf{}
\lhead{GFD-R}
\rhead{\today}

% footer
\lfoot{occi-wg@ogf.org}
\rfoot{\thepage}

% paragraphs need some space...
\setlength{\parindent}{0pt}
\setlength{\parskip}{1ex plus 0.5ex minus 0.2ex}

% some space between header and text...
\headsep 13pt

\setcounter{secnumdepth}{4}

\begin{document}

% header on first page is different
\thispagestyle{empty}

Draft \hfill Andy Edmons, ICCLab, ZHAW \\
OCCI-WG \hfill Thijs Metsch, Intel\\
\rightline {\today}

\vspace*{0.5in}

\begin{Large}
\textbf{Open Cloud Computing Interface -- Text Rendering}
\end{Large}

\vspace*{0.5in}

\underline{Status of this Document}

This document provides information to the community regarding the
specification of the Open Cloud Computing Interface. Distribution is
unlimited.


\underline{Copyright Notice}

Copyright \copyright Open Grid Forum (2015-2016). All Rights Reserved.

\underline{Trademarks}

OCCI is a trademark of the Open Grid Forum.

\underline{Abstract}

This document, part of a document series, produced by the OCCI working
group within the Open Grid Forum (OGF), provides a high-level
definition of a Protocol and API. The document is based upon
previously gathered requirements and focuses on the scope of important
capabilities required to support modern service offerings.


\newpage
\tableofcontents
\newpage

\section{Introduction}
%!TEX root = nml-base.tex

\section{Introduction}%
\label{sec:introduction}

This document describes the base schema of the Network Markup Language (NML).
Section~\ref{sub:classes} defines the NML classes and their attributes and parameters.
Section~\ref{sub:relations} describes the relations defined between NML classes.

An NML network description can be expressed in XML\cite{xml}, and RDF/XML\cite{rdfxml} syntax.
Section~\ref{s:xmlschema} describes the XSD schema for the XML syntax.
Section~\ref{s:owlschema} describes the OWL 2 schema for the RDF/XML syntax.

These basic classes defined in this document may be extended, or sub-classed, 
to represent technology specific classes.

Section~\ref{s:examples} provides example use cases. This section is informative. 
Only sections~\ref{s:schema}, \ref{s:identifiers}, \ref{s:syntax}, and appendices \ref{s:xmlschema} and \ref{s:owlschema} are normative and considered 
part of the recommendation.

Appendix~\ref{s:g800terms} is informative and explains the relation between terms defined in this document and those defined in the ITU-T G.800 recommendation~\cite{g800}.

\subsection{Context}
\label{sec:context}

The Network Markup Language (NML) has been defined in the context of research and 
education networks to describe so-called hybrid network topologies. The NML is defined
as an abstract and generic model, so it can be applied for other network topologies as well.
See \cite{gfd.165} for an detailed overview including prior work.

\subsection{Scope}
\label{sec:scope}

The Network Markup Language is designed to create a functional description of 
multi-layer networks and multi-domain networks. An example of a multi-layered 
network can be a virtualised network, but also using different technologies. 
The multi-domain network descriptions can include aggregated or abstracted network topologies.
NML can not only describe a primarily static network topology, but also its potential capabilities (services) 
and its configuration.

NML is aimed at logical connection-oriented network topologies, more precisely topologies
where switching is performed on a label associated with a flow, such as a VLAN, wavelength or time slot. 
NML can also be used to describe physical networks or packet-oriented networks, 
although the current base schema does not contain classes or properties 
to explicitly deal with signal degradation, or complex routing tables.

NML only attempts to describe the data plane of a computer network, not the control 
plane. It does contain extension mechanism to easily tie it with network provisioning 
standards and with network monitoring standards.

Finally, this document omits a definition for the terms \emph{Network} or \emph{capacity}. 
This has been a conscious choice. The term \emph{Network} has become 
so widely used for so many diverse meanings that it is impossible to create a 
definition that everyone can agree on, while still expressing something useful.
See \emph{Topology} for the concept of a network domain and a \emph{Link} with multiple 
sources and sinks for the concept of a local area network.
The term \emph{capacity} is used by different technologies in such a different 
way (e.g.\ including or excluding the header and footer overhead) that it is better 
to let technology-specific extensions make an explicit definition.

\subsection{Notational Conventions}%
\label{sec:rfc2119}

The keywords “\MUST{}”, “\MUSTNOT{}”, “\REQUIRED{}”, “\SHALL{}”, “\SHALLNOT{}”, 
“\SHOULD{}”, “\SHOULDNOT{}”, “\RECOMMENDED{}”, “\MAY{}”,  and “\OPTIONAL{}” are 
to be interpreted as described in \cite{rfc2119}.
% except that the words do not appear in uppercase. 

This schema defines classes, attributes, relations, parameters and logic.
Objects are instances of classes, and the type of an object is a class.

Names of classes are capitalised and written in italics (e.g.\ the \emph{Node} class).
Names of relations are written in camel case and in italics (e.g.\ the \emph{hasNode} relation).
Names of identifiers and string literals are written in monspaces font (e.g. \texttt{Port\_X:in}).

Diagrams in this document follow the diagrammatic conventions of UML class diagrams.
\begin{itemize}
\item A subclass-superclass relationship is represented by a line with hollow triangle shape pointing to the superclass.
\item A whole-part relationship is represented by a line with a hollow diamond shape pointing to the whole (group).
\item A entity-relationship is represented by a line, optionally with numbers at each end indicating the cardinality of the relation. A named entity-relationship has a verb next to the line, and a filled triangle pointing to the object of the verb. (e.g. the entitity-relationship
\nmlrelation{BidirectionalPort}{*}{hasPort}{2}{Port} is named \emph{hasPort}, and each \emph{BidirectionalPort} is related to exactly 2 \emph{Port}s, and each \emph{Port} may be associated with zero, one or more \emph{BidirectionalPort}s.)
\end{itemize}



\section{Notational Conventions}
All these parts and the information within are mandatory for
implementors (unless otherwise specified). The key words "MUST", "MUST
NOT", "REQUIRED", "SHALL", "SHALL NOT", "SHOULD", "SHOULD NOT",
"RECOMMENDED", "MAY", and "OPTIONAL" in this document are to be
interpreted as described in RFC 2119 \cite{rfc2119}.


\section{Text rendering}

This document presents the text-based renderings. To be complaint, OCCI implementations
MUST implement the three renderings defined in sections \ref{sec:text}, \ref{sec:header} and \ref{sec:urilist}.

\emph{The following specification of the text-based renderings is in the process of being
deprecated and will be removed or significantly changed in the next MAJOR release of the standard.}

The document is structured by defining base ABNFs, which can then be combined into renderings, which will
be rendered over a protocol (e.g., HTTP) by the specific rendering definitions.

\section{ABNF Definitions}

For the following section of \ref{sec:renderings} these ABNF notations will be used.
Implementations MUST hence implement the renderings according to these definitions.

\subsection{Category ABNF}

The following syntax MUST be used for \hl{Category} renderings:

\begin{verbatim}
Category             = "Category" ":" #category-value
  category-value     = term
                      ";" "scheme" "=" <"> scheme <">
                      ";" "class" "=" ( class | <"> class <"> )
                      [ ";" "title" "=" quoted-string ]
                      [ ";" "rel" "=" <"> type-identifier <"> ]
                      [ ";" "location" "=" <"> URI <"> ]
                      [ ";" "attributes" "=" <"> attribute-list <"> ]
                      [ ";" "actions" "=" <"> action-list <"> ]
  term               = ( ALPHA | DIGIT ) *( ALPHA | DIGIT | "-" | "_" | "." )
  scheme             = URI
  type-identifier    = scheme term
  class              = "action" | "mixin" | "kind"
  attribute-list     = attribute-def
                     | attribute-def *( 1*SP attribute-def)
  attribute-def      = attribute-name
                     | attribute-name
                       "{" attribute-property *( 1*SP attribute-property ) "}"
  attribute-property = "immutable" | "required"
  attribute-name     = term
  action-list        = action
                     | action *( 1*SP action )
  action             = type-identifier
\end{verbatim}

\subsection{Link ABNF}

The following syntax MUST be used to represent OCCI \hl{Link} type
instance references:

\begin{verbatim}
Link               = "Link" ":" #link-value
  link-value       = "<" URI-reference ">"
                    ";" "rel" "=" <"> resource-type <">
                    [ ";" "self" "=" <"> link-instance <"> ]
                    [ ";" "category" "=" link-type
                      *( ";" link-attribute ) ]
  term             = ( ALPHA | DIGIT ) *( ALPHA | DIGIT | "-" | "_" | "." )
  scheme           = URI
  type-identifier  = scheme term
  resource-type    = type-identifier *( 1*SP type-identifier )
  link-type        = type-identifier *( 1*SP type-identifier )
  link-instance    = URI-reference
  link-attribute   = attribute-name "=" ( token | quoted-string )
  attribute-name   = term
\end{verbatim}

The following syntax MUST be used to represent OCCI \hl{Action}
instance references:

\begin{verbatim}
ActionLink         = "Link" ":" #link-value
  link-value       = "<" action-uri ">"
                    ";" "rel" "=" <"> action-type <">
  term             = ( ALPHA | DIGIT ) *( ALPHA | DIGIT | "-" | "_" | "." )
  scheme           = URI
  type-identifier  = scheme term
  action-type      = type-identifier
  action-uri       = URI "?" "action=" term
\end{verbatim}

\subsection{Attribute ABNF}

\begin{verbatim}
Attribute          = "X-OCCI-Attribute" ":" #attribute-repr
  attribute-repr   = attribute-name "=" attribute-value
  attribute-name   = ( ALPHA | DIGIT ) *( ALPHA | DIGIT | "-" | "_" | "." )
  attribute-value  = ( string | number | bool | enum-val )
  string           = quoted-string
  number           = (int | float)
  int              = *DIGIT
  float            = *DIGIT "." *DIGIT
  bool             = ("true" | "false")
  enum-val         = string
\end{verbatim}

\subsection{Location ABNF}

\begin{verbatim}
Location      = "X-OCCI-Location" ":" location-value
  location-value  = URI-reference
\end{verbatim}

\section{Renderings}
\label{sec:renderings}

The renderings defined in this section will be used in the specific text rendering defined in section \ref{sec:text} and \ref{sec:header}

\subsection{Entity Instance Rendering}

Entity instances MUST be rendered according to the following definitions.

\subsubsection{Resource Instance Rendering}

A \hl{Resource} instance MUST be rendered using the following definition:

\begin{verbatim}
  resource_rendering = 1*( Category CRLF )
                        *( Link CRLF )
                        *( Attribute CRLF )
\end{verbatim}

The rendering of a \hl{Resource} instance MUST represent any associated Action instances using the {\tt ActionLink CRLF}.

\paragraph{Action Invocation Rendering}

Upon an \hl{Action} invocation the client MUST send along the following definition:

\begin{verbatim}
  action_definition = 1( Category CRLF )
                      *( Attribute CRLF )
\end{verbatim}

\subsubsection{Link Instance Rendering}

A \hl{Link} instance MUST be rendered using the following definition:

\begin{verbatim}
  link_rendering = 1*( Category CRLF )
                    *( ActionLink CRLF )
                    *( Attribute CRLF )
\end{verbatim}

% HERE I AM

\subsection{Category Instance Rendering}
\label{sec:format_category_instance_rendering}

A \hl{Category} instances MUST be rendered as defined below.

\subsubsection{Kind Instance Rendering}
\label{sec:format_kind}

A \hl{Kind} instance MUST be rendered as a {\tt Category CRLF}.

\subsubsection{Mixin Instance Rendering}
\label{sec:format_mixin}

A \hl{Mixin} instance MUST be rendered as a {\tt Category CRLF}.

\subsubsection{Action Instance Rendering}
\label{sec:format_action}

An \hl{Action} instance MUST be rendered as a {\tt Category CRLF}.

Note that an \hl{Action} instance MUST NOT have \hl{Link} and \hl{Action}s references.

\subsection{Entity Collection Rendering}

A collection of \hl{Resource} or \hl{Link} instances MUST be rendered as following:

\begin{verbatim}
  entity_collection_rendering = *( Location CRLF )
\end{verbatim}

\subsubsection{Resource Collection Rendering}

see above

\subsubsection{Link Collection Rendering}

see above

\subsection{Category Collection Rendering}

For the Query interface the following \hl{Category} instance rendering MUST be used:

\begin{verbatim}
  category_collection_rendering = *( Category CRLF )
\end{verbatim}

\subsubsection{Kind Collection Rendering}

see above

\subsubsection{Mixin Collection Rendering}

see above

\subsubsection{Action Collection Rendering}

see above

\subsection{Attributes Rendering}

\subsubsection{Entity Instance Attribute Rendering Specifics}

For Entity instances the following model attribute name to attribute name rendering mappings MUST be used:

\mytablefloat{
  \label{tbl:link}
  Entity attribute naming convention } {
  \begin{tabular}{ll}
    \toprule
      Attribute         & Attribute name once rendered \\
    \colrule
      Entity.id         & occi.core.id \\
      Entity.title      & occi.core.title \\
      Resource.summary  & occi.core.summary \\
      Link.target       & occi.core.target \\
      Link.target.kind  & occi.core.target.kind \\
      Link.source       & occi.core.source \\
      Link.source.kind  & occi.core.source.kind \\
    \botrule
  \end{tabular}
}

\subsubsection{Mixin Instance Attribute Rendering Specifics}

When rendering {\tt Mixin.depends} and {\tt Mixin.applies} to the {\tt rel} attribute in the \hl{Category} instance rendering,
only {\tt Mixin.depends} value MUST be used. If {\tt Mixin.depends} contains multiple values, only the first value MUST
be used.

\subsubsection{Attribute Description Rendering}
\label{sec:format_attribute_description}

\hl{Attributes} MUST be rendered as defined by the {\tt Attribute CRLF}. If used, the {\tt pattern} model attribute
MUST be represented as a string in the ERE \cite{ere} format.

\section{OCCI Text Plain rendering}
\label{sec:text}
The OCCI Text plain rendering specifies a rendering of OCCI instance types in a simple text format.

The rendering can be used to render OCCI instances independently of the
protocol being used. Thus messages can be delivered by, e.g., the HTTP
protocol as specified in \cite{occi:http_protocol}.

The following media-types MUST be used for the OCCI Text plain rendering:

  {\tt text/occi+plain}

and

  {\tt text/plain}

Each entry in the body consists of a name followed by a colon (``:'') and the field value.

\subsection{Example}

The following example show an \hl{Entity} instance rendering using the Text plain rendering.

\begin{verbatim}
< Category: compute; \
<     scheme="http://schemas.ogf.org/occi/infrastructure#" \
<     class="kind";
< Link: </users/foo/compute/b9ff813e-fee5-4a9d-b839-673f39746096?action=start>; \
<     rel="http://schemas.ogf.org/occi/infrastructure/compute/action#start"
< X-OCCI-Attribute: occi.core.id="urn:uuid:b9ff813e-fee5-4a9d-b839-673f39746096"
< X-OCCI-Attribute: occi.core.title="My Dummy VM"
< X-OCCI-Attribute: occi.compute.architecture="x86"
< X-OCCI-Attribute: occi.compute.state="inactive"
< X-OCCI-Attribute: occi.compute.speed=1.33
< X-OCCI-Attribute: occi.compute.memory=2.0
< X-OCCI-Attribute: occi.compute.cores=2
< X-OCCI-Attribute: occi.compute.hostname="dummy"
\end{verbatim}

\section{OCCI Header Rendering}
\label{sec:header}
The following media-type MUST be used for the OCCI header Rendering:

{\tt text/occi}

While using this rendering the HTTP Protocol \cite{occi:http_protocol} MUST be used and the renderings
MUST be placed in the HTTP Header. The body MUST contain the string ``OK'' on successful operations.

The HTTP header fields MUST follow the specification in RFC 7230 \cite{rfc7230}. A header field consists of a name followed by a colon (``:'') and the field value.

\textbf{Limitations}: HTTP header fields MAY appear multiple times in a HTTP request or response. In order to be OCCI compliant, the specification of multiple message-header fields according to RFC 7230 MUST be fully supported. In essence there are two valid representations of multiple HTTP header field values. A header field might either appear several times or as a single header field with a comma-separated list of field values. Due to implementation issues in many web frameworks and client libraries it is RECOMMENDED to use the comma-separated list format for best interoperability.

HTTP header field values, which contain separator characters, MUST be properly quoted according to RFC~7230.

Space in the HTTP header section of a HTTP request is a limited resource. By this, it is noted that many HTTP servers limit the number of bytes that can be placed in the HTTP header area. Implementers MUST be aware of this limitation in their own implementations and take appropriate measures so that truncation of header data does NOT occur.

\subsection{Example}

The following example shows an \hl{Entity} instance rendering using the Text header rendering.

\begin{verbatim}
< Category: compute; \
    scheme="http://schemas.ogf.org/occi/infrastructure#" \
    class="kind";
< Link: </users/foo/compute/b9ff813e-fee5-4a9d-b839-673f39746096?action=start>; \
    rel="http://schemas.ogf.org/occi/infrastructure/compute/action#start"
< X-OCCI-Attribute: occi.core.id="urn:uuid:b9ff813e-fee5-4a9d-b839-673f39746096", \
 occi.core.title="My Dummy VM", occi.compute.architecture="x86", \
 occi.compute.state="inactive", occi.compute.speed=1.33, \
 occi.compute.memory=2.0, occi.compute.cores=2, \
 occi.compute.hostname="dummy"
< OK
\end{verbatim}

\section{URI Listing Rendering}
\label{sec:urilist}
The following media-types MUST be used for the URI Rendering:

{\tt text/uri-list}

This rendering cannot render resource instances or Kinds or Mixins directly but just links to them. For concrete rendering of Kinds and Categories the Content-types \texttt{text/occi}, \texttt{text/plain} MUST be used. If a request is done with the \texttt{text/uri-list} in the Accept header, while not requesting for a Listing a Bad Request MUST be returned. Otherwise a list of resources MUST be rendered in \texttt{text/uri-list} format as defined in \cite{rfc2483}, which can be used for listing resource in collections or the name-space of the OCCI implementation.

\section{Security Considerations}
OCCI does not require that an authentication mechanism be used nor
does it require that client to service communications are secured. It
does RECOMMEND that an authentication mechanism be used and that where
appropriate, communications are encrypted using HTTP over TLS. The
authentication mechanisms that MAY be used with OCCI are those that
can be used with HTTP and TLS. For further discussion see the
appropriate section in \cite{occi:http_protocol}.

\section{Glossary}
\label{sec:glossary}

\section{Glossary}
\label{s:glossary}

\begin{description}
\item[metric] a metric is a mathematical representation of a well defined aspect of a physical entity
\item[measurement] a measurement is the process of extracting a metric from a physical entity, and by extension also the result of such process. The measurement seldom corresponds exactly to the value of the metric.
\item[SLA] {\em ``An agreement defines a dynamically-established and dynamically
managed relationship between parties. The object of this
relationship is the delivery of a service by one of the parties within
the context of the agreement.''} from {\em SLA@SOI Glossary}
\item[Restful model] {\em ``REST is a coordinated set of architectural constraints that attempts to minimize latency and network communication, while at the same time maximizing
the independence and scalability of component implementations.''} \cite{fie02a}
\item[OCCI] {``\em The Open Cloud Computing Interface (OCCI) is a RESTful Protocol and API for all kinds of management tasks. OCCI was originally initiated to create a remote management API for IaaS model-based services, allowing for the development of interoperable tools for common tasks including deployment, autonomic scaling and monitoring''} \cite{occi:core}
\item[OCCI {\em Kind}] {\em''The Kind type represents the type identification mechanism for all Entity types present in the model''} \cite{occi:core}
\item[OCCI {\em \ln}] {\em''An instance of the Link type defines a base association between two Resource instances.''} \cite{occi:core}
\item[OCCI \mi] {\em''The Mixin type represent an extension mechanism, which allows new resource
capabilities to be added to resource instances both at creation-time and/or run-time.''} \cite{occi:core}
\item[OCCI \rs] {\em''A Resource is suitable to represent real world resources, e.g. virtual machines, networks, services, etc. through specialisation.''} \cite{occi:core}
\item[\sens] The \sens\ is a \rs\ that collects metrics from its input side, and delivers aggregated metrics from its output
\item[\coll] The \coll\ is a link that conveys metrics: it defines both the transport protocol and the conveyed metrics.
\end{description}


\section{Contributors}

We would like to thank the following people who contributed to this
document:

\begin{tabular}{l|p{2in}|p{2in}}
Name & Affiliation & Contact \\
\hline
Michael Behrens & R2AD & behrens.cloud at r2ad.com \\
Mark Carlson & Toshiba & mark at carlson.net \\
Augusto Ciuffoletti & University of Pisa & augusto.ciuffoletti at gmail.com\\
Andy Edmonds & ICCLab, ZHAW & edmo at zhaw.ch \\
Sam Johnston & Google & samj at samj.net \\
Gary Mazzaferro & Independent &  garymazzaferro at gmail.com \\
Thijs Metsch & Intel & thijs.metsch at intel.com \\
Ralf Nyrén & Independent & ralf at nyren.net \\
Alexander Papaspyrou & Adesso & alexander at papaspyrou.name \\
Boris Parák & CESNET & parak at cesnet.cz \\
Alexis Richardson & Weaveworks & alexis.richardson at gmail.com \\
Shlomo Swidler & Orchestratus & shlomo.swidler at orchestratus.com \\
Florian Feldhaus & Independent & florian.feldhaus at gmail.com \\
Zden\v{e}k \v{S}ustr & CESNET & zdenek.sustr at cesnet.cz \\
\end{tabular}

Next to these individual contributions we value the contributions from
the OCCI working group.


\section{Intellectual Property Statement}
The OGF takes no position regarding the validity or scope of any
intellectual property or other rights that might be claimed to pertain
to the implementation or use of the technology described in this
document or the extent to which any license under such rights might or
might not be available; neither does it represent that it has made any
effort to identify any such rights. Copies of claims of rights made
available for publication and any assurances of licenses to be made
available, or the result of an attempt made to obtain a general
license or permission for the use of such proprietary rights by
implementers or users of this specification can be obtained from the
OGF Secretariat.

The OGF invites any interested party to bring to its attention any
copyrights, patents or patent applications, or other proprietary
rights which may cover technology that may be required to practice
this recommendation. Please address the information to the OGF
Executive Director.


\section{Disclaimer}
This document and the information contained herein is provided on an
``As Is'' basis and the OGF disclaims all warranties, express or
implied, including but not limited to any warranty that the use of the
information herein will not infringe any rights or any implied
warranties of merchantability or fitness for a particular purpose.


\section{Full Copyright Notice}
Copyright \copyright ~Open Grid Forum (2009-2011). All Rights Reserved.

This document and translations of it may be copied and furnished to
others, and derivative works that comment on or otherwise explain it
or assist in its implementation may be prepared, copied, published and
distributed, in whole or in part, without restriction of any kind,
provided that the above copyright notice and this paragraph are
included on all such copies and derivative works. However, this
document itself may not be modified in any way, such as by removing
the copyright notice or references to the OGF or other organizations,
except as needed for the purpose of developing Grid Recommendations in
which case the procedures for copyrights defined in the OGF Document
process must be followed, or as required to translate it into
languages other than English.

The limited permissions granted above are perpetual and will not be
revoked by the OGF or its successors or assignees.


\bibliographystyle{IEEEtran}
\bibliography{references}

\appendix

\newpage
\section{Change Log}
\label{sec:change_log}

The corrections introduced by the {\today} update are summarized below.
This section describes the possible impact of the corrections on existing
implementations and associated dependent specifications.

\begin{itemize}
  \item Relaxed rules on {\tt term} values allowing the use of: alphanumerical characters (a-zA-Z0-9),
        ``\_'', ``-'' and ``.''.
  \item Explicitly stated how {\tt Mixin.depends} and {\tt Mixin.applies} should be rendered to {\tt rel}
        on \hl{Mixin} instances.
\end{itemize}

\end{document}
