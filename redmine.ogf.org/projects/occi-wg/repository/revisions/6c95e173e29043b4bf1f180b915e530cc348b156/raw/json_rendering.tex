\documentclass[10pt,a4paper]{article}
\usepackage[utf8]{inputenc}
\usepackage[english]{babel}
\usepackage[activate={true,nocompatibility},final,tracking=true,kerning=true,spacing=true]{microtype}
\usepackage[plainpages=false,pdfpagelabels,unicode]{hyperref}
\usepackage{fullpage}
\usepackage{graphicx}
\usepackage{fancyhdr}
\usepackage{comment}
\usepackage{occi}
\usepackage{lineno}   % adds line numbers, may be removed for non draft versions
\linenumbers          % adds line numbers, may be removed for non draft versions
\usepackage{verbatim} % adds verbatim options
\usepackage{tabularx} % adds extended tabular formatting options
\usepackage{listings}
\usepackage{color}
\definecolor{lightgray}{rgb}{.9,.9,.9}
\definecolor{darkgray}{rgb}{.4,.4,.4}
\definecolor{purple}{rgb}{0.65, 0.12, 0.82}

\lstdefinelanguage{json}{
  ndkeywords={String, Number, Boolean, Null, Object, Array},
  ndkeywordstyle=\itshape
}
\lstset{
   language=json,
   basicstyle=\footnotesize,
}

\setlength{\headheight}{13pt}
\pagestyle{fancy}

% default sans-serif
\renewcommand{\familydefault}{\sfdefault}

% no lines for headers and footers
\renewcommand{\headrulewidth}{0pt}
\renewcommand{\footrulewidth}{0pt}

% header
\fancyhf{}
\lhead{GWD-R}
\rhead{\today}

% footer
\lfoot{occi-wg@ogf.org}
\rfoot{\thepage}

% paragraphs need some space...
\setlength{\parindent}{0pt}
\setlength{\parskip}{1ex plus 0.5ex minus 0.2ex}

% some space between header and text...
\headsep 13pt

\setcounter{secnumdepth}{4}

\begin{document}

% header on first page is different
\thispagestyle{empty}

Draft \hfill Ralf Nyrén, Independent \\
OCCI-WG \hfill Florian Feldhaus, GWDG\\
\rightline {Boris Parák, CESNET}\\
\rightline {February 25, 2011}\\
\rightline {Updated: \today}

\vspace*{0.5in}

\begin{Large}
\textbf{Open Cloud Computing Interface -- JSON Rendering}
\end{Large}

\vspace*{0.5in}

\underline{Status of this Document}

This document is a \underline{draft} providing information to the community regarding the specification of the Open Cloud Computing Interface.

\underline{Copyright Notice}

Copyright \copyright Open Grid Forum (2012-2015). All Rights Reserved.

\underline{Trademarks}

OCCI is a trademark of the Open Grid Forum.

\underline{Abstract}

This document, part of a document series, produced by the OCCI working
group within the Open Grid Forum (OGF), provides a high-level
definition of a Protocol and API. The document is based upon
previously gathered requirements and focuses on the scope of important
capabilities required to support modern service offerings.

\newpage
\tableofcontents
\newpage

\section{Introduction}
%!TEX root = nml-base.tex

\section{Introduction}%
\label{sec:introduction}

This document describes the base schema of the Network Markup Language (NML).
Section~\ref{sub:classes} defines the NML classes and their attributes and parameters.
Section~\ref{sub:relations} describes the relations defined between NML classes.

An NML network description can be expressed in XML\cite{xml}, and RDF/XML\cite{rdfxml} syntax.
Section~\ref{s:xmlschema} describes the XSD schema for the XML syntax.
Section~\ref{s:owlschema} describes the OWL 2 schema for the RDF/XML syntax.

These basic classes defined in this document may be extended, or sub-classed, 
to represent technology specific classes.

Section~\ref{s:examples} provides example use cases. This section is informative. 
Only sections~\ref{s:schema}, \ref{s:identifiers}, \ref{s:syntax}, and appendices \ref{s:xmlschema} and \ref{s:owlschema} are normative and considered 
part of the recommendation.

Appendix~\ref{s:g800terms} is informative and explains the relation between terms defined in this document and those defined in the ITU-T G.800 recommendation~\cite{g800}.

\subsection{Context}
\label{sec:context}

The Network Markup Language (NML) has been defined in the context of research and 
education networks to describe so-called hybrid network topologies. The NML is defined
as an abstract and generic model, so it can be applied for other network topologies as well.
See \cite{gfd.165} for an detailed overview including prior work.

\subsection{Scope}
\label{sec:scope}

The Network Markup Language is designed to create a functional description of 
multi-layer networks and multi-domain networks. An example of a multi-layered 
network can be a virtualised network, but also using different technologies. 
The multi-domain network descriptions can include aggregated or abstracted network topologies.
NML can not only describe a primarily static network topology, but also its potential capabilities (services) 
and its configuration.

NML is aimed at logical connection-oriented network topologies, more precisely topologies
where switching is performed on a label associated with a flow, such as a VLAN, wavelength or time slot. 
NML can also be used to describe physical networks or packet-oriented networks, 
although the current base schema does not contain classes or properties 
to explicitly deal with signal degradation, or complex routing tables.

NML only attempts to describe the data plane of a computer network, not the control 
plane. It does contain extension mechanism to easily tie it with network provisioning 
standards and with network monitoring standards.

Finally, this document omits a definition for the terms \emph{Network} or \emph{capacity}. 
This has been a conscious choice. The term \emph{Network} has become 
so widely used for so many diverse meanings that it is impossible to create a 
definition that everyone can agree on, while still expressing something useful.
See \emph{Topology} for the concept of a network domain and a \emph{Link} with multiple 
sources and sinks for the concept of a local area network.
The term \emph{capacity} is used by different technologies in such a different 
way (e.g.\ including or excluding the header and footer overhead) that it is better 
to let technology-specific extensions make an explicit definition.

\subsection{Notational Conventions}%
\label{sec:rfc2119}

The keywords “\MUST{}”, “\MUSTNOT{}”, “\REQUIRED{}”, “\SHALL{}”, “\SHALLNOT{}”, 
“\SHOULD{}”, “\SHOULDNOT{}”, “\RECOMMENDED{}”, “\MAY{}”,  and “\OPTIONAL{}” are 
to be interpreted as described in \cite{rfc2119}.
% except that the words do not appear in uppercase. 

This schema defines classes, attributes, relations, parameters and logic.
Objects are instances of classes, and the type of an object is a class.

Names of classes are capitalised and written in italics (e.g.\ the \emph{Node} class).
Names of relations are written in camel case and in italics (e.g.\ the \emph{hasNode} relation).
Names of identifiers and string literals are written in monspaces font (e.g. \texttt{Port\_X:in}).

Diagrams in this document follow the diagrammatic conventions of UML class diagrams.
\begin{itemize}
\item A subclass-superclass relationship is represented by a line with hollow triangle shape pointing to the superclass.
\item A whole-part relationship is represented by a line with a hollow diamond shape pointing to the whole (group).
\item A entity-relationship is represented by a line, optionally with numbers at each end indicating the cardinality of the relation. A named entity-relationship has a verb next to the line, and a filled triangle pointing to the object of the verb. (e.g. the entitity-relationship
\nmlrelation{BidirectionalPort}{*}{hasPort}{2}{Port} is named \emph{hasPort}, and each \emph{BidirectionalPort} is related to exactly 2 \emph{Port}s, and each \emph{Port} may be associated with zero, one or more \emph{BidirectionalPort}s.)
\end{itemize}



\section{Notational Conventions}
All these parts and the information within are mandatory for
implementors (unless otherwise specified). The key words "MUST", "MUST
NOT", "REQUIRED", "SHALL", "SHALL NOT", "SHOULD", "SHOULD NOT",
"RECOMMENDED", "MAY", and "OPTIONAL" in this document are to be
interpreted as described in RFC 2119 \cite{rfc2119}.


\section{OCCI JSON Rendering}
\label{sec:json_format}
The OCCI JSON Rendering specifies a rendering of OCCI instance types in the JSON
data interchange format as defined in \cite{rfc4627}.

The rendering can be used to render OCCI instances independently of the
protocol being used. Thus messages can be delivered by, e.g., the HTTP
protocol as specified in \cite{occi:protocol}.

The following media-type MUST be used for the OCCI JSON Rendering:

{\tt application/occi+json}

The OCCI JSON Rendering consists of a JSON object containing information on the
OCCI Core instances OCCI Kind, OCCI Mixin, OCCI Action,
OCCI Link and OCCI Resource. The rendering also include a JSON object to invoke
the operation identified by OCCI Actions.
The rendering of each OCCI Core instance will be
described in the following sections.

\subsection{Entity Instance Rendering}
\label{sec:format_entity_instance_rendering}

Entity instances MUST be rendered as JSON hashmaps.

\subsubsection{Resource Instance Rendering}
\label{sec:format_resource}

\todo{Section EXAMPLE RESOURCE is missing.}

The OCCI Resource Instance Rendering consists of a JSON object as shown in the
following example. Section \ref{sec:example_resource} contains a detailed
example.
Table~\ref{tbl:format_resource} defines the object members.
\begin{lstlisting}

        {
            "kind": String,
            "mixins": Array,
            "attributes": Object,
            "actions": Array,
            "id": String,
            "links": Array,
            "summary": String,
            "title": String,
        }


\end{lstlisting}
\mytablefloat{
    \label{tbl:format_resource}
    OCCI Resource instance rendered with the following entries:
    } {
    \begin{tabularx}{\textwidth}{llXll}
    \toprule
    Object member & JSON type & Description & Mutability & Multiplicity \\
    \colrule
    kind & String & Type identifier & immutable & 1 \\

    mixins & Array of Strings & List of type identifiers of associated OCCI
Mixins  &
mutable & 0..1 \\

    attributes & Object & Instance Attributes (see
\ref{sec:format_attribute_description}) & mutable & 0..1 \\

    actions & Array of Strings & List of type identifiers of OCCI
Actions applicable to the OCCI Resource instance & mutable & 0..1 \\

    id & String & ID of the OCCI Resource & immutable & 1\\

    links & Array of Strings & List of URIs of OCCI Links & mutable & 0..1\\
    summary & String & Summary text of resource & mutable & 0..1 \\
    title & String & Title of resource & mutable & 0..1 \\
    \botrule
    \end{tabularx}
}

\paragraph{Action Invocation Rendering}
\label{sec:format_action_invocation}

The OCCI Action Invocation Rendering identifies an invocable operation on a OCCI Resource or
OCCI Link instance. To trigger such an operation the OCCI Action Invocation
Rendering is required.

\todo{Section EXAMPLE ACTION INVOCATION is missing.}

The OCCI Action Invocation Rendering consists of a top-level JSON object as shown in the
following example. Section \ref{sec:example_action_invocation} contains a detailed example.
Table~\ref{tbl:format_action_invocation} defines the object members.

\begin{lstlisting}
{
    "action": String,
    "attributes": Object
}
\end{lstlisting}

\mytablefloat{
    \label{tbl:format_action_invocation}
    An OCCI Action invocation is rendered with
 the following entries:
    } {
    \begin{tabularx}{\textwidth}{llXll}
    \toprule
    Object member & JSON type & Description & Mutability & Multiplicity \\
    \colrule
    action & String & Type identifier & immutable & 1 \\

    attributes & Object & Instance attributes (see
\ref{sec:format_attribute_description}) & mutable & 0..1 \\
    \botrule
    \end{tabularx}
}


\subsubsection{Link Instance Rendering}
\label{sec:format_link}

\todo{Section EXAMPLE LINK is missing.}

The OCCI Link Instance Rendering consists of a JSON object as shown in the
following example. Section \ref{sec:example_link} contains a detailed example.
Table~\ref{tbl:format_link} defines the object members.
\begin{lstlisting}

        {
            "kind": String,
            "mixins": Array,
            "attributes": Object,
            "actions": Array,
            "id": String,
            "source": String,
            "source.type": String,
            "target": String,
            "target.type": String,
            "title": String
        }

\end{lstlisting}

\mytablefloat{
    \label{tbl:format_link}
    OCCI Link instances are rendered with the following entries:
    } {
    \begin{tabularx}{\textwidth}{llXll}
    \toprule
    Object member & JSON type & Description & Mutability & Multiplicity \\
    \colrule
    kind & String & Type identifier & immutable & 1 \\

    mixins & Array of Strings & List of type identifiers of associated OCCI
Mixins &
    mutable & 0..1 \\

    attributes & Object & Instance attributes (see
\ref{sec:format_attribute_description}) & mutable & 0..1 \\

    actions & Array of Strings & List of type identifiers of OCCI
Action Categories applicable to the OCCI Link instance & mutable & 0..1 \\

    id & String & ID of the OCCI Link & immutable & 1\\

    source & String & URI of the source OCCI Resource. If only one OCCI
Resource is rendered in the same collection, this OCCI Resource is the
source of the OCCI Link if this entry is omitted & immutable & 0..1\\

    source.type & string & Type identifier of the source Resource. & immutable & 0..1 \\

    target & String & URI of the target Resource & immutable & 1\\

    target.type & string & Type identifier of the target Resource, to be supplied if
the target is an OCCI Resource. & immutable & 0..1 \\
	title & String & title of the Link & mutable & 0..1\\
    \botrule
    \end{tabularx}
}


\subsection{Category Instance Rendering}
\label{sec:format_category_instance_rendering}
Category instances MUST be rendered as JSON hashmaps.

\subsubsection{Kind Instance Rendering}
\label{sec:format_kind}

\todo{Section EXAMPLE KIND is missing.}

The OCCI Kind Instance Rendering consists of a JSON object as shown in the
following example. Section \ref{sec:example_kind} contains a detailed example.
Table~\ref{tbl:format_kind} defines the top-level object members.

\mytablefloat{
    \label{tbl:format_kind}
    OCCI Kind instances are rendered with the following entries:
    } {
    \begin{tabularx}{\textwidth}{llXll}
    \toprule
    Object member & JSON type & Description & Mutability & Multiplicity \\
    \colrule
    term & String & Unique identifier within the categorization scheme &
immutable & 1 \\

    scheme & String & Categorization scheme & immutable & 1 \\

    title & String & Title of the OCCI Kind & immutable & 0..1 \\

    attributes & Object & Attribute description, see
~\ref{tbl:format_attribute_description} & immutable & 0..1 \\

    parent & String & OCCI Kind type identifier of the
related ``parent'' \hl{Kind} instance & immutable & 0..1 \\

    actions & Array of Strings & List of OCCI Action type
identifiers & immutable & 0..1 \\

    location & string & Transport protocol specific URI bound to the OCCI Kind
instance. MUST be supplied for the OCCI Kinds of all OCCI Entities except OCCI
Entity itself & immutable & 0..1 \\
    \botrule
    \end{tabularx}
}

\begin{lstlisting}

        {
            "term": String,
            "scheme": String,
            "title": String,
            "attributes": Object,
            "actions": Array,
            "parent": String,
            "location": String
        }

\end{lstlisting}

\subsubsection{Mixin Instance Rendering}
\label{sec:format_mixin}

\todo{Section EXAMPLE MIXIN is missing.}

The OCCI Mixin Instance Rendering consists of a JSON object as shown in the following example. Section \ref{sec:example_mixin} contains a detailed example.
Table~\ref{tbl:format_mixin} defines the top-level object members.

\begin{lstlisting}

        {
            "term": String,
            "scheme": String,
            "title": String,
            "attributes": Object,
            "actions": Array,
            "depends": Array,
            "applies": Array,
            "location": String
        }

\end{lstlisting}

\mytablefloat{
    \label{tbl:format_mixin}
    OCCI Mixin instances are rendered with the following entries:
    } {
    \begin{tabularx}{\textwidth}{llXll}
    \toprule
    Object member & JSON type & Description & Mutability & Multiplicity \\
    \colrule
    term & String & Unique identifier within the categorization scheme &
immutable & 1 \\

    scheme & String & Categorization scheme & immutable & 1 \\

    title & String & Title of the OCCI Mixin & immutable & 0..1 \\

    attributes & Object & Attribute description, see
~\ref{tbl:format_attribute_description} & immutable & 0..1 \\

    depends & Array of Strings & List of type identifiers of the dependent
 \hl{Mixin} instances & immutable & 0..1 \\

    applies & Array of Strings & List of OCCI Kind type identifiers this OCCI
Mixin can be applied to & immutable & 0..1 \\

    actions & Array of Strings & List of OCCI Action type identifiers
& immutable & 0..1 \\

    location & String & Transport protocol specific URI bound to the OCCI Mixin
instance & immutable & 1 \\
    \botrule
    \end{tabularx}
}

\subsubsection{Action Instance Rendering}
\label{sec:format_action}

The OCCI Action Instance Rendering consists of a JSON object as shown in the
following example.
Table~\ref{tbl:format_action} defines the top-level object members.

\mytablefloat{
    \label{tbl:format_action}
    OCCI Actions are rendered inside the
top-level JSON object with name {\em actions} as an array of JSON Objects with
 the following entries:
    } {
    \begin{tabularx}{\textwidth}{llXll}
    \toprule
    Object member & JSON type & Description & Mutability & Multiplicity \\
    \colrule
    term & String & Unique type identifier within the categorization scheme &
immutable & 1 \\

    scheme & String & Categorization scheme & immutable & 1 \\

    title & String & Title of the OCCI Action & immutable & 0..1 \\

    attributes & Object & Attribute description, see
~\ref{tbl:format_attribute_description} & immutable & 0..1 \\
    \botrule
    \end{tabularx}
}

\begin{lstlisting}

        {
            "term": String,
            "scheme": String,
            "title": String,
            "attributes": Object,
        }

\end{lstlisting}

\subsection{Entity Collection Rendering}
Collections of Entity instances MUST be rendered as JSON arrays. The content of that array is a set of entity instance renderings.

That array MUST be a member of a JSON hashmap that is associated with the relevant key name specific to the type of Entity collection being rendered.

\subsubsection{Resource Collection Rendering}

The JSON hashmap key-name associated with the array of resource instances MUST be \hl{resources}.

\begin{lstlisting}
{
    "resources": []
}

\end{lstlisting}

\subsubsection{Link Collection Rendering}

The JSON hashmap key-name associated with the array of link instances MUST be \hl{links}.

\begin{lstlisting}
{
    "links": []
}
\end{lstlisting}

\subsection{Category Collection Rendering}
Collections of Category instances MUST be rendered as JSON arrays. The content of that array is a set of Category instance renderings.

That array MUST be a member of a JSON hashmap that is associated with the relevant key name specific to the type of Category collection being rendered.


\subsubsection{Kind Collection Rendering}

The JSON hashmap key-name associated with the array of kind instances MUST be \hl{kinds}.

\begin{lstlisting}
{
    "kinds": []
}
\end{lstlisting}

\subsubsection{Mixin Collection Rendering}

The JSON hashmap key-name associated with the array of mixin instances MUST be \hl{mixins}.

\begin{lstlisting}
{
    "mixins": []
}
\end{lstlisting}

\subsubsection{Action Collection Rendering}

The JSON hashmap key-name associated with the array of action instances MUST be \hl{actions}.

\begin{lstlisting}
{
    "actions": []
}
\end{lstlisting}


Collections of Category instances are rendered as JSON arrays.

\subsection{Attributes Rendering}

Attribute names consist of alphanumeric characters separated by dots. The dots
define a logical namespace-like hierarchy. This hierarchy is NOT reflected in JSON
objects. As shown in the following example, the attribute name is an opaque
identifier rendered as hashmap \textit{key}. The hashmap \textit{value} contains either a
Number, a String, a Boolean, an Array or an Object (as an attribute value or an attribute
description, following the Attribute Description Rendering, see \ref{sec:format_attribute_description}).
\begin{lstlisting}
{
    "one.two.three": Number|String|Boolean|Array|Object,
    "one.two.four" : Number|String|Boolean|Array|Object
}
\end{lstlisting}

\subsubsection{Attribute Description Rendering}
\label{sec:format_attribute_description}

Attribute Descriptions are rendered as JSON objects as defined in table~\ref{tbl:format_attribute_description}

\mytablefloat{
    \label{tbl:format_attribute_description}
    All properties of the Attribute definition are optional, but may contain
defaults which MUST be used if the Attribute is not present in the instantiated
OCCI Entity.
    } {
    \begin{tabularx}{\textwidth}{llXll}
    \toprule
    Object member & JSON type & Description & Default \\
    \colrule
    mutable & Boolean & Defines if the Attribute is mutable after initialization
& false \\

    required & Boolean & Defines if the Attribute MUST be specified at
instantiation of the OCCI Entity & false \\

    type & String & Type of the Attribute. MUST be either ``string'', ``number'',
``boolean'', ``array'' or ``object''. & string \\

    pattern & string & POSIX Extended Regular Expression as defined in
\cite{iso9945:2009}. For interoperability reasons, POSIX character classes
 (e.g. [:alpha:]) MUST NOT be used. & .* \\

    default & String, Number, Boolean, Array, Object & Attribute default. MUST be the same
type as defined in the type property and MUST  be used if the Attribute is not
present in the instantiated OCCI Entity & \\

    description & String & Description of the attribute & \\
    \botrule
    \end{tabularx}
}
\begin{lstlisting}
{
    "mutable": Boolean,
    "required": Boolean,
    "type": String,
    "default": String | Number | Boolean | Array | Object,
    "description": String,
    "pattern": String
}
\end{lstlisting}

\section{Security Considerations}
OCCI does not require that an authentication mechanism be used nor
does it require that client to service communications are secured. It
does RECOMMEND that an authentication mechanism be used and that where
appropriate, communications are encrypted using HTTP over TLS. The
authentication mechanisms that MAY be used with OCCI are those that
can be used with HTTP and TLS. For further discussion see the
appropiate section in \cite{occi:protocol}.

\section{Glossary}
\label{sec:glossary}

\section{Glossary}
\label{s:glossary}

\begin{description}
\item[metric] a metric is a mathematical representation of a well defined aspect of a physical entity
\item[measurement] a measurement is the process of extracting a metric from a physical entity, and by extension also the result of such process. The measurement seldom corresponds exactly to the value of the metric.
\item[SLA] {\em ``An agreement defines a dynamically-established and dynamically
managed relationship between parties. The object of this
relationship is the delivery of a service by one of the parties within
the context of the agreement.''} from {\em SLA@SOI Glossary}
\item[Restful model] {\em ``REST is a coordinated set of architectural constraints that attempts to minimize latency and network communication, while at the same time maximizing
the independence and scalability of component implementations.''} \cite{fie02a}
\item[OCCI] {``\em The Open Cloud Computing Interface (OCCI) is a RESTful Protocol and API for all kinds of management tasks. OCCI was originally initiated to create a remote management API for IaaS model-based services, allowing for the development of interoperable tools for common tasks including deployment, autonomic scaling and monitoring''} \cite{occi:core}
\item[OCCI {\em Kind}] {\em''The Kind type represents the type identification mechanism for all Entity types present in the model''} \cite{occi:core}
\item[OCCI {\em \ln}] {\em''An instance of the Link type defines a base association between two Resource instances.''} \cite{occi:core}
\item[OCCI \mi] {\em''The Mixin type represent an extension mechanism, which allows new resource
capabilities to be added to resource instances both at creation-time and/or run-time.''} \cite{occi:core}
\item[OCCI \rs] {\em''A Resource is suitable to represent real world resources, e.g. virtual machines, networks, services, etc. through specialisation.''} \cite{occi:core}
\item[\sens] The \sens\ is a \rs\ that collects metrics from its input side, and delivers aggregated metrics from its output
\item[\coll] The \coll\ is a link that conveys metrics: it defines both the transport protocol and the conveyed metrics.
\end{description}


\section{Contributors}

We would like to thank the following people who contributed to this
document:

\begin{tabular}{l|p{2in}|p{2in}}
Name & Affiliation & Contact \\
\hline
Michael Behrens & R2AD & behrens.cloud at r2ad.com \\
Mark Carlson & Toshiba & mark at carlson.net \\
Augusto Ciuffoletti & University of Pisa & augusto.ciuffoletti at gmail.com\\
Andy Edmonds & ICCLab, ZHAW & edmo at zhaw.ch \\
Sam Johnston & Google & samj at samj.net \\
Gary Mazzaferro & Independent &  garymazzaferro at gmail.com \\
Thijs Metsch & Intel & thijs.metsch at intel.com \\
Ralf Nyrén & Independent & ralf at nyren.net \\
Alexander Papaspyrou & Adesso & alexander at papaspyrou.name \\
Boris Parák & CESNET & parak at cesnet.cz \\
Alexis Richardson & Weaveworks & alexis.richardson at gmail.com \\
Shlomo Swidler & Orchestratus & shlomo.swidler at orchestratus.com \\
Florian Feldhaus & Independent & florian.feldhaus at gmail.com \\
Zden\v{e}k \v{S}ustr & CESNET & zdenek.sustr at cesnet.cz \\
\end{tabular}

Next to these individual contributions we value the contributions from
the OCCI working group.

% FIXME: Insert an up-to-date table here!

\section{Intellectual Property Statement}
The OGF takes no position regarding the validity or scope of any
intellectual property or other rights that might be claimed to pertain
to the implementation or use of the technology described in this
document or the extent to which any license under such rights might or
might not be available; neither does it represent that it has made any
effort to identify any such rights. Copies of claims of rights made
available for publication and any assurances of licenses to be made
available, or the result of an attempt made to obtain a general
license or permission for the use of such proprietary rights by
implementers or users of this specification can be obtained from the
OGF Secretariat.

The OGF invites any interested party to bring to its attention any
copyrights, patents or patent applications, or other proprietary
rights which may cover technology that may be required to practice
this recommendation. Please address the information to the OGF
Executive Director.


\section{Disclaimer}
This document and the information contained herein is provided on an
``As Is'' basis and the OGF disclaims all warranties, express or
implied, including but not limited to any warranty that the use of the
information herein will not infringe any rights or any implied
warranties of merchantability or fitness for a particular purpose.


\section{Full Copyright Notice}
Copyright \copyright ~Open Grid Forum (2009-2011). All Rights Reserved.

This document and translations of it may be copied and furnished to
others, and derivative works that comment on or otherwise explain it
or assist in its implementation may be prepared, copied, published and
distributed, in whole or in part, without restriction of any kind,
provided that the above copyright notice and this paragraph are
included on all such copies and derivative works. However, this
document itself may not be modified in any way, such as by removing
the copyright notice or references to the OGF or other organizations,
except as needed for the purpose of developing Grid Recommendations in
which case the procedures for copyrights defined in the OGF Document
process must be followed, or as required to translate it into
languages other than English.

The limited permissions granted above are perpetual and will not be
revoked by the OGF or its successors or assignees.


\bibliographystyle{IEEEtran}
\bibliography{references}

\end{document}
