
\section{Glossary}
\label{s:glossary}

\begin{description}
\item[metric] a metric is a mathematical representation of a well defined aspect of a physical entity
\item[measurement] a measurement is the process of extracting a metric from a physical entity, and by extension also the result of such process. The measurement seldom corresponds exactly to the value of the metric.
\item[SLA] {\em ``An agreement defines a dynamically-established and dynamically
managed relationship between parties. The object of this
relationship is the delivery of a service by one of the parties within
the context of the agreement.''} from {\em SLA@SOI Glossary}
\item[Restful model] {\em ``REST is a coordinated set of architectural constraints that attempts to minimize latency and network communication, while at the same time maximizing
the independence and scalability of component implementations.''} \cite{fie02a}
\item[OCCI] {``\em The Open Cloud Computing Interface (OCCI) is a RESTful Protocol and API for all kinds of management tasks. OCCI was originally initiated to create a remote management API for IaaS model-based services, allowing for the development of interoperable tools for common tasks including deployment, autonomic scaling and monitoring''} \cite{occi:core}
\item[OCCI {\em Kind}] {\em''The Kind type represents the type identification mechanism for all Entity types present in the model''} \cite{occi:core}
\item[OCCI {\em \ln}] {\em''An instance of the Link type defines a base association between two Resource instances.''} \cite{occi:core}
\item[OCCI \mi] {\em''The Mixin type represent an extension mechanism, which allows new resource
capabilities to be added to resource instances both at creation-time and/or run-time.''} \cite{occi:core}
\item[OCCI \rs] {\em''A Resource is suitable to represent real world resources, e.g. virtual machines, networks, services, etc. through specialisation.''} \cite{occi:core}
\item[\sens] The \sens\ is a \rs\ that collects metrics from its input side, and delivers aggregated metrics from its output
\item[\coll] The \coll\ is a link that conveys metrics: it defines both the transport protocol and the conveyed metrics.
\end{description}
