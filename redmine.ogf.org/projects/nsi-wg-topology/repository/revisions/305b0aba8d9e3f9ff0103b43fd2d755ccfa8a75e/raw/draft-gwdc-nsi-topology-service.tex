%!TEX TS-program = pdflatex

%%%% Latex preamble and page formatting %%%%

\documentclass[12pt]{article}  % larger font to compensate for long lines with fullpage
\usepackage[utf8]{inputenc}
\usepackage[T1]{fontenc}
\usepackage[pdfborder={0 0 0}]{hyperref} % use hyperref without borders
\usepackage{ifpdf}
\usepackage{ifthen}
\usepackage{graphicx}
\usepackage{url}
\usepackage{color}
\usepackage{listings}
\usepackage[title,titletoc]{appendix}
% Read pictures from img/ and current directory
\graphicspath{{img/}{./}}

%%% GWD/GFD header follows %%%
% Feel free to make changes, as long as your document follows the guidelines of GFP.152

\usepackage[numbers]{natbib} % Use [1] for references, 
\bibliographystyle{plainnat} % References show full author name(s) and document URL

\usepackage[sf,compact]{titlesec} % Use sans-serif for section headers

\usepackage[titles]{tocloft} % Format table of contents
% (tocloft is used, since titletoc is incompatible with xetex.)
\renewcommand{\cftsecfont}{\sffamily}
\renewcommand{\cftsubsecfont}{\sffamily}
\renewcommand{\cftsubsubsecfont}{\sffamily}
\renewcommand{\cftsecpagefont}{\sffamily}
\renewcommand{\cftsubsecpagefont}{\sffamily}
\renewcommand{\cftsubsubsecpagefont}{\sffamily}
\renewcommand{\cftsecleader}{\cftdotfill{\cftsubsecdotsep}} % dots for sections the same as for sections
\setlength{\cftbeforesecskip}{0.5ex}

\usepackage{parskip} % Blank lines between paragraphs, no indentation.

% font style for headers and footers
\newcommand{\headerstyle}{\sffamily} % sans-serif

% Set page margins
\usepackage{fancyhdr}
\addtolength{\headheight}{15pt}
\renewcommand{\headrulewidth}{0pt}
% \setlength{\headrulewidth}{0pt}
\setlength{\headsep}{20pt}
\usepackage[headings]{fullpage}  % small margins

% Macro to make some editorial notes
\newenvironment{note}{\framebox{note:} \color[gray]{0.5}}{}

% Macro to check if (optional) values above are defined or not.
\newcommand{\ifnonempty}[2]{\ifthenelse{\isundefined{#1}}{}{\ifthenelse{\equal{#1}{}}{}{#2}}}

%%%% Document header and title page %%%%

\title{Network Service Interface Topology Service Distribution Mechanisms}
\author{Jeroen van der Ham}
\newcommand{\shortdoctitle}{NSI TS}  % Title used in page header
% \date{} and \author{} are currently ignored
\newcommand{\authorsshort}{Jeroen van der Ham, UvA}
\newcommand{\publicationdate}{December 2012}  % Date of first publication of the document
% \newcommand{\revisiondate}{August 2012}  % Optional: date of last revision of the document
\newcommand{\copyrightyears}{2008-2012}  % Years used in copyright notice
\newcommand{\docseries}{GWD-C}  % GWD-R, GWD-I or GWD-C (for working drafts)
% \newcommand{\docseries}{GFD.191}  % GFD.000 (for approved documents)

\ifpdf
\hypersetup{
    pdftitle = {Network Service Interface Topology Service},
    pdfauthor = {Jeroen van der Ham},
    pdfsubject = {Description of distribution mechanisms for the topology service},
    pdfkeywords = {nsi topology distribution}
}
\fi


% Define page header and footers
\pagestyle{fancyplain}
\fancyhf{}
\lhead{\fancyplain{}{\headerstyle\docseries}}
% use \revisiondate if defined, otherwise \publicationdate for right header:
\rhead{\fancyplain{}{\headerstyle\ifthenelse{\isundefined{\revisiondate }}{\publicationdate}{\ifthenelse{\equal{\revisiondate}{}}{\publicationdate}{\revisiondate}}}}
\lfoot{\headerstyle\ifnonempty{\groupurl}{\groupurl}}
\rfoot{\headerstyle\thepage}
\thispagestyle{plain}

\begin{document}

% Title page header
{\noindent
\begin{minipage}[t]{1.5in}
\headerstyle
\docseries \\
NSI-WG \\
\href{mailto:nsi-wg@ogf.org}{nsi-wg@ogf.org}
\end{minipage}
\hfill
\raggedleft
\begin{minipage}[t]{4.5in}
\raggedleft
\headerstyle
\authorsshort \\
\vspace{1em}
\publicationdate \\
\ifnonempty{\revisiondate}{Revised \revisiondate \\}
\end{minipage}
}

\vspace{1em}
\begin{center}
\makeatletter
\Large\bf\textsf \@title
\makeatother
\end{center}


\section*{Status of This Document}

Group Working Draft (GWD), Community Practice (C).
% TODO: before publication:
%Grid Final Draft (GFD), Community Practice (C).


% \section*{Document Change History}
% 
% TODO: use this for formal revisions of this document

\section*{Copyright Notice}

Copyright \copyright \ Open Grid Forum (\copyrightyears).  Some Rights Reserved.  
Distribution is unlimited.

\phantomsection\addcontentsline{toc}{section}{Abstract}
\section*{Abstract}

This document describes a normative schema which allows the
description of service plane objects required for the Network Service Interface Connection Service. Additionally it describes a set of distribution mechanisms for the network topology descriptions.

\phantomsection\addcontentsline{toc}{section}{Contents}
\tableofcontents

\newcommand{\qq}{\symbol{34}} % 34 is the decimal LaTeX code for "
\newcommand{\q}{\symbol{39}} % 39 is the decimal LaTeX code for '
\newcommand{\underscore}{\symbol{95}} % 39 is the decimal LaTeX code for _

\newcommand{\MUST}{\textsc{must}}
\newcommand{\MUSTNOT}{\textsc{must not}}
\newcommand{\REQUIRED}{\textsc{required}}
\newcommand{\SHALL}{\textsc{shall}}
\newcommand{\SHALLNOT}{\textsc{shall not}}
\newcommand{\SHOULD}{\textsc{should}}
\newcommand{\SHOULDNOT}{\textsc{should not}}
\newcommand{\RECOMMENDED}{\textsc{recommended}}
\newcommand{\MAY}{\textsc{may}}
\newcommand{\OPTIONAL}{\textsc{optional}}

\newpage

\section{Introduction}

 The NSI Connection Service requires topology descriptions to do 
pathfinding. In order to do that some representation of the topology is required. 
Once represented, some form of topology distribution is also needed. This document 
describes some requirements for the NSI Topology Service, suggests a short-term 
implementation and a strategy for better long-term support.

 In the first section we describe what is necessary for the topology 
to support, what kind of elements should be in there. In the next section we describe 
the distribution requirements, some possible solutions and a recommended solution 
for the short-term and also for the longer term\label{h.15n4tpv97j8w}


\section{Distribution of NSI Topology}

 Some form of Topology distribution is required in order for an 
inter-domain NSI network to function. In NSI 1.0 this process was performed out-of-band, 
mostly through e-mail. For NSI 2.0 we take the opportunity to define an NSI Topology 
Service for NSI topology exchange, which can support the NSI Connection Service.\label{h.c2129mkn366h}


\subsection{Transport and Service plane relationship}

 The NSI Connection Service is implemented on Network Service Agents 
(NSAs), which together form a network and tree-like structure. This graph represents 
how reservation requests would propagate through the network, but not necessarily 
reflects the transport-plane. One NSA may be an aggregation point for other NSAs, 
not visible from the outside.\label{h.mtia1ajwgzmv}


The messaging between the NSAs will happen on the 
service plane, which is completely separate from the transport plane.\label{h.dc66uw3zo12t}\label{h.ywjdj9kuwkou}


\subsection{Elements of a Topology Exchange Mechanism}

 There are three main elements of topology exchange: \label{h.u0b214m8krs}


\paragraph{Bootstrapping Topology Exchange}

 To start the initial Network Service the NSAs must be able to 
find each other, in order to communicate details about the network. So some form 
of bootstrapping is required, with initial synchronization between domains on both 
the service plane and the transport plane, i.e. the NSAs of both domains must be 
able to contact each other, and the details of transport plane connections between 
them have to be synchronized as well.\label{h.fzohc79rts8y}


Initiating a transport plane connection between two 
networks is not a frequent occurrence, and a longer process, involving out-of-band 
(for NSI) contact. Part of that process can be that the networks also communicate 
the access details for the NSAs, thus forming an NSA relationship.\label{h.30v92h80y7jj}


\paragraph{Expanding the Topology Exchange}

 Once the neighboring (on the control-plane) NSAs have exchanged 
details, they can also distribute details about the rest of the network, both the 
control plane details and connectivity, but also some transport details.\label{h.c1dz1g10oju}

\paragraph{Update Mechanism}

 The transport network is not static, and links are added or removed 
from time to time. An update mechanism is thus required to inform other NSAs about 
these kinds of events.\label{h.uzmy62lgeii3}


\subsection{Topology Exchange Implementations}

 The above mechanisms can be implemented in five different ways:


\paragraph{Centralized Manual Distribution}
 An initial attempt at topology distribution in the Automated GOLE 
demonstration was through a central maintainer. This maintainer collected all topology 
information from the networks involved, gathered all the topology data and sent 
out a topology file through e-mail. The network maintainers would than download 
the attachment and insert it into their provisioning system. Updates to the topology 
were all handled through the central maintainer, distributed through e-mail.\label{h.31tbtceozcoc}


While this system worked initially, it soon ran into 
scaling problems. This system also does not allow to have a good way of doing automatic 
updates or insertion.


\paragraph{Version Controlled Distribution}

 The Automated GOLE demonstration has transitioned into a different 
distribution mechanism using a Git source code repository, available on GitHub. 
This mechanism also still has a central maintainer, this also allows networks to 
manipulate their own topology information. The distribution mechanism is either 
directly from the GitHub website, or through the git version control system itself. 
The git system has the added advantage that it is a distributed version control 
system, so it is not required to download directly from GitHub.\label{h.rtljwvhj8cbo}


Bootstrapping and updating all happens through the 
git system itself.


\paragraph{PerfSonar Lookup Service}

 The PerfSonar monitoring system suite also contains a service 
for looking up information. This service uses a ``home Lookup Service''(hLS) where 
metadata of information is registered, which is then uploaded into the ``global 
Lookup Service''(gLS) (this can be cloud of services).\label{h.pg14lxc0zu6o}


The retrieval of information happens by first querying 
the gLS, then the relevant hLS, followed by the service where the actual information 
is stored.\label{h.tjjrmby8htjk}


Since topology information would be stored locally, 
no update mechanism is necessary, except for location changes of the services itself. 
However, this method of storing and lookup does require full connectivity between 
all NSAs to provide and retrieve information, which may not be possible.


 A soon to be released (target Dec 2012) updated PS Lookup service 
will incorporate subscription capabilities, which allow an hLS to push information 
to a remote hLS and have it cached locally there. By adapting the new Lookup service 
to store topology, and selectively managing the subscription of data, full connectivity 
between NSAs would not be necessary to disseminate global topology.


\paragraph{HTTP Distribution}

 A common way of distributing information is using the HTTP protocol. 
The topology files would include links to topology description files, which would 
allow other domains to directly fetch the topology description.\label{h.efkcsmt7dwgr}


However, as with the Lookup Service, this requires 
direct access between NSAs which may not be possible.\label{h.mbgns4xsvbt4}


\paragraph{A Peer-to-Peer Distribution Protocol}

 Another method is to define a new protocol for NSI topology distribution. 
As explained above the protocol would only require a small set of primitives, and 
would work directly between peering NSAs.\label{h.jo1it2zr3py}


Once an NSA comes up, it contacts its neighbors to 
request the topology information that they know about, and subscribes to future 
topology updates. These updates are propogated in a peer-to-peer fashion through 
the whole network.\label{h.3qkmh8hazvti}


The end-result would be that all the NSAs have a 
global view of the topology, with only local interaction.\label{h.40nvr2if8zj}


\subsection{Summarizing Topology Distribution}

 Above we have described five different topology distribution mechanisms, 
from these five only the version controlled and peer-to-peer systems fit the requirements 
that the NSI has, others require too much manual operation, or require globally 
reachable NSAs. The peer-to-peer system would be the best solution with minimal 
interaction between systems, and no reliance on outside mechanisms. For practical 
reasons we believe the best solution right now is a version controlled distribution 
system, and in time evolve to a peer-to-peer distribution protocol.\label{h.j5x37xsutbrh}




\end{document}
