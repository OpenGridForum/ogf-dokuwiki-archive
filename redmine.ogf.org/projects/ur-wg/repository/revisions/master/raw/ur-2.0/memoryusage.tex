% !TeX root = draft-ur-v2.tex

\section{MemoryUsageBlock}

This block contains the properties related to memory usage. The block may be present several times to account for different types of memory (e.g., RAM and swap).

\begin{XMLexample}
<ur:MemoryUsageBlock>
<!— Memory Usage properties go in here -->
</ur:MemoryUsageBlock>
\end{XMLexample}



%%%%%%%%%%%%%%%



\subsection{MemoryClass}

The class of memory used. RAM and swap must be supported but others might be specified. This is a descriptive value, which allows the memory system to provide details about the memory used.

\begin{itemize}
\item \emph{MemoryClass} MUST be present in the \emph{MemoryUsageBlock}.
\item \emph{MemoryClass} MUST be a string.
\item The values ``RAM'' and ``swap'' MUST be supported.
\end{itemize}

\begin{XMLexample}
<ur:MemoryClass>"RAM"</ur:MemoryClass>
\end{XMLexample}



%%%%%%%%%%%%%%%



\subsection{MemoryResourceCapacityUsed}

The number of physical bytes used on the memory system (e.g., the amount of memory resources used for this process). This is the main metric for measuring memory consumption.

\begin{itemize}
\item \emph{MemoryResourceCapacityUsed} MUST be present in the \emph{MemoryUsageBlock}.
\item \emph{MemoryResourceCapacityUsed} MUST be a positive integer.
\end{itemize}

\begin{XMLexample}
<ur:MemoryResourceCapacityUsed>14728</ur:MemoryResourceCapacityUsed> 
\end{XMLexample}



%%%%%%%%%%%%%%%



\subsection{MemoryResourceCapacityAllocated}

The number of bytes allocated for this process on the memory system (e.g., the amount of memory resources made available for this process). This value may be higher than the \emph{MemoryResourceCapacityUsed} because it may also include bytes that are not really used by the process.

\begin{itemize}
\item \emph{MemoryResourceCapacityAllocated} MUST be a positive integer.
\end{itemize}

\begin{XMLexample}
<ur:MemoryResourceCapacityAllocated>56437</ur:MemoryResourceCapacityAllocated> 
\end{XMLexample}



%%%%%%%%%%%%%%%



\subsection{MemoryResourceCapacityRequested}

The number of bytes requested by the process (e.g., the memory requested in a job description).

\begin{itemize}
\item \emph{Memory\-Resource\-Capacity\-Allocated} MUST be a positive integer.
\end{itemize}

\begin{XMLexample}
<ur:MemoryResourceCapacityRequested>42000</ur:MemoryResourceCapacityRequested>
\end{XMLexample}



%%%%%%%%%%%%%%%



\subsection{StartTime}

See \emph{StartTime} in \emph{ComputeUsageBlock} (see chapter \ref{ComputeStartTime}).



%%%%%%%%%%%%%%%



\subsection{EndTime}

See \emph{EndTime} in \emph{ComputeUsageBlock} (see chapter \ref{ComputeEndTime}).



%%%%%%%%%%%%%%%



\subsection{ExecutionHost} \label{MemoryHost}

See \emph{ExecutionHost} in \emph{ComputeUsageBlock} (see chapter \ref{ComputeHost}).

%The system on which the resources have been consumed. This value should be chosen in such a way that it globally identifies the system, on which resources are being consumed (e.g. the Fully Qualified Domain Name of the system could be used). 

%\begin{itemize}
%\item \emph{Host} MUST be a string. 
%\end{itemize}

%\begin{XMLexample}
%<ur:Host>host.example.org</ur:Host>
%\end{XMLexample}



%%%%%%%%%%%%%%%



\subsection{HostType}

See \emph{HostType} in \emph{ComputeUsageBlock} (see chapter \ref{ComputeHostType}).



%%%%%%%%%%%%%%%



\subsection{Charge}

See \emph{Charge} in \emph{ComputeUsageBlock} (see chapter \ref{ComputeCharge}).
