% !TeX root = draft-ur-v2.tex

\section{JobUsageBlock}

The block is intended to be used for jobs submitted to a batch system, a grid, or some similar distributed computing infrastructure.

\begin{XMLexample}
<ur:JobUsageBlock>
<!—Job Record properties go in here -->
</ur:JobUsageBlock> 
\end{XMLexample}



%%%%%%%%%%%%%%%



\subsection{GlobalJobId}

The global identity of the job. The property should identify the job globally, such that clashes do not happen. This could be a combination of a time stamp, a local job id and a host name.

\begin{itemize}
\item The \emph{GlobalJobId} field type MUST be a string.
\end{itemize}

\begin{XMLexample}
<ur:GlobalJobId>"host.example.org/ab1234/2013-05-09T09:06:52Z"</ur:GlobalJobId>
\end{XMLexample}



%%%%%%%%%%%%%%%



\subsection{LocalJobId}

The local identity of the job, for example the ID the job was assigned by the local resource management system (LRMS).

\begin{itemize}
\item \emph{LocalJobId} MUST be a string.
\end{itemize}

\begin{XMLexample}
<ur:LocalJobId>"ab1234"</ur:LocalJobId>
\end{XMLexample}



%%%%%%%%%%%%%%%



\subsection{JobName}

A descriptive name of the job. % It has to be stressed that user defined job names are often difficult to retrieve from an accounting perspective and are not suitable for reliable accounting purposes.

\begin{itemize}
\item \emph{JobName} MUST be a string.
\end{itemize}

\begin{XMLexample}
<ur:JobName>"HiggsGammaGamma42"</ur:JobName>
\end{XMLexample}



%%%%%%%%%%%%%%%



\subsection{MachineName} \label{JobMachineName}

A descriptive name of the machine on which the job ran. This may be a system hostname, the LRMS server hostname or a sites name for a cluster of machines. 
The identification of the machine by name may assume the context of the site or Grid in which the machine participates, i.e. machine
names may be unique within a specific site or Grid, but do not need to be unique globally.

\begin{itemize}
\item \emph{MachineName} SHOULD be present in the \emph{JobUsageBlock}.
\item \emph{MachineName} MUST be a string.
\end{itemize}

\begin{XMLexample}
<ur:MachineName>"ce.example.org"</ur:MachineName>
\end{XMLexample}



%%%%%%%%%%%%%%%



\subsection{SubmitHost} \label{JobSubmitHost}

The host from which the jobs was submitted.

\begin{itemize}
\item In a Grid environment \emph{SubmitHost} MUST report the Computing Element Unique ID. 
\item The \emph{SubmitHost} field type MUST be a string.
\end{itemize}

\begin{XMLexample}
<ur:SubmitHost>
   "nordugrid-cluster-name=ce.example.org,Mds-Vo-name=local,o=grid"
</ur:SubmitHost>
\end{XMLexample}



%%%%%%%%%%%%%%%



\subsection{SubmitType}

The purpose of this element is to mark whether the job was submitted locally or through a (Grid) middleware. At least the values ``local'' and ``grid'' MUST be supported. The attribute \emph{description} SHOULD be used to give additional information on the used middleware.
\begin{itemize}
\item \emph{SubmitType} MUST be a string.
%\item The values ``local'' and ``grid'' MUST be supported.
\item The attribute \emph{description} SHOULD be used.
\item The attribute \emph{description} MUST be a string.
\item This value should be defined in a community specific profile.
\end{itemize}

\begin{XMLexample}
<ur:SubmitType ur:description="ARC CE">grid</ur:SubmitType>
\end{XMLexample}



%%%%%%%%%%%%%%%



\subsection{Queue}

The name of the queue from which the job was executed or submitted.
\begin{itemize}
\item \emph{Queue} MUST be a string.
\item The attribute \emph{description} MAY be specified.
\item The attribute \emph{description} MUST be a string.
\end{itemize}

\begin{XMLexample}
<ur:Queue ur:description="execution">"Bigmem"</ur:Queue>
\end{XMLexample}



%%%%%%%%%%%%%%%



\subsection{TimeInstant} \label{JobTimeInstant}

Time instant related to the user payload. Three optional values for the attribute \emph{type} are defined as they are of common usage by batch systems. The semantic is derived from Torque.

\begin{itemize}
\item \emph{TimeInstant} SHOULD be present in the \emph{JobUsageBlock}.
\item \emph{TimeInstant} MAY be present multiple times.
\item \emph{TimeInstant} MUST be an ISO 8601:2004 timestamp.
\item The attribute \emph{type} MUST be a string.
\item The following three values of the attribute \emph{type} are defined and SHOULD be reported:
\begin{itemize}
\item \emph{Ctime} - Time job was created
\item \emph{Qtime} - Time job was queued
\item \emph{Etime} - Time job became eligible to run
\end{itemize}
\end{itemize}

\begin{XMLexample}
<ur:TimeInstant ur:type="Etime">2013-05-31T10:59:42</ur:TimeInstant>
\end{XMLexample}



%%%%%%%%%%%%%%%



\subsection{ServiceLevel} \label{JobServiceLevel}

This property identifies the quality of service associated with the resource consumption.
Service level may represent a priority associated with the usage.
\begin{itemize}
\item \emph{ServiceLevel} MUST be a String.
\end{itemize}

\begin{XMLexample}
<ur:ServiceLevel>BigMem</ur:ServiceLevel>
\end{XMLexample}



%%%%%%%%%%%%%%%



\subsection{Status}

Completion status of the job. This may
represent the exit status of an interactive running process or the exit status from the batch queuing systems accounting record. The semantic meaning of status is site dependent.
\begin{itemize}
\item \emph{Status} MUST be of type String.
\item \emph{Status} MUST exist in the record.
\item  \emph{Status} MUST support the following values:
\begin{itemize}
\item \emph{aborted} – A policy or human intervention caused the job to cease execution.
\item \emph{completed} – The execution completed.
\item \emph{failed} – Execution halted without external intervention.
\item \emph{held} – Execution is held at the time this usage record was generated.
\item \emph{queued} – Execution was queued at the time this usage record was generated.
\item \emph{started} – Execution started at the time this usage record was generated.
\item \emph{suspended} – Execution was suspended at the time this usage record was generated.
\end{itemize}
\item The \emph{Status} property MAY support other values, as agreed upon within the implementation context.
\end{itemize}

\begin{XMLexample}
<ur:Status>"aborted"</ur:Status>
\end{XMLexample}
