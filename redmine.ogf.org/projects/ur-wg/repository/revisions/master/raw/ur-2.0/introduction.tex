% !TeX root = draft-ur-v2.tex
% encoding=utf-8

\section{Introduction}
\label{s:introduction}

In order for resources to be shared, sites must be able to exchange basic accounting and usage data in a common format. This document focuses on the representation of resource consumption data. The document then goes on to describe an XML-based format for usage records. The record format is intended to be specific enough to facilitate information sharing among grid sites, yet general enough that the usage data can be used for a variety of purposes: traditional usage accounting, charging, service usage monitoring, performance tuning, etc. The purpose of this document is to outline the basic building blocks of the accounting record, and how to properly represent them. All other tangential concerns such as the use, transport mechanism, and security are out of scope for this representation layer.



%%%%%%%%%%%%%%%%%%



\subsection{Context}

To comprehend the structure of the schema presented in this document, it is important to understand the context in which this specification has been developed. The accounting of different use-cases involves recording:

\begin{itemize}
\item General properties related to the record itself
\item Properties related to the consumer of the resources
\item Usage of one or more resources.
\end{itemize}

Hence, the usage record schema is made up of a set of blocks for general properties, consumer and distinct resources. 



%%%%%%%%%%%%%%%%%%



\subsubsection{History}

Before the definition of UR-2.0 different usage record definitions building on the job accounting definition of UR-1.0 \cite{mach2007usagerecord}, started to surface to describe various resource usages, such as the EMI Compute Accounting Reccord CAR \cite{guarise2011definition}, the EMI Storage Accounting Record StAR \cite{nilsen2011emi}, the EGI Cloud Usage Record CUR \footnote{https://wiki.egi.eu/wiki/Fedcloud-tf:WorkGroups:Scenario4} and the Storage Accounting Implementation SAI \cite{cristofori2011grid}. The definition of UR-2.0 came from the experiences of these record definitions and is built as an easily extensible superset of these new usage records.



%%%%%%%%%%%%%%%%%%



\subsubsection{What This Document Is Not}

This document and specification do not attempt to define a comprehensive ``grid accounting'' standard. As with all accounting implementations, there is no one-size-fits-all solution, that will meet the needs of all projects and resource providers. This document does not address summary records, ``grid job'' records, consolidated records, or anything other than an atomic resource consumption instantiation. Sufficient resource and user information is collected to allow for effective and appropriate levels of aggregation, consolidation, and summarization, but the details of how sites implement these features (e.g., what grids do with the atomic data) are beyond the scope of this document. This definition of UR-2.0 does not enter into the details of how the Usage Record should be used or the way records are transported from the information producers to its consumers. Neither does it enter into implementation details of the accounting sensors.



%%%%%%%%%%%%%%%%%%



\subsection{Format of the Record Specification}

UR-2.0 defines the building blocks necessary for the accounting of different resources. This is achieved by combining the blocks in different ways. In this document all the fields and blocks that are part of UR-2.0 will be described. Additionally, example records for accounting of storage, grid, cloud are given. This record specification is aiming at being as general as possible. Different communities should then create their own profiles where they specify the combination of blocks and fields required for their implementation. 



%%%%%%%%%%%%%%%%%%