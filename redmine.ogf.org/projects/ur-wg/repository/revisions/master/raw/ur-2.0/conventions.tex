% !TeX root = draft-ur-v2.tex
% encoding=utf-8

\section{Conventions Used in this Document}
\label{s:conventions}

\subsection{Notational Conventions}
\label{s:rfc2119}

The key words ``MUST'' ``MUST NOT'', ``REQUIRED'', ``SHALL'', ``SHALL NOT'', ``SHOULD'', ``SHOULD NOT'', ``RECOMMENDED'', ``MAY'',  and ``OPTIONAL'' are to be interpreted as described in RFC 2119~\cite{rfc2119}, except that the words do not appear in uppercase. 



%%%%%%%%%%%%%%%%%%



\subsection{Meta Properties}

Meta properties are associated with individual base properties to provide additional information and semantic meaning of the value for a base property.  The meta properties outlined below are commonly encountered and should be supported for the indicated base properties.



%%%%%%%%%%%%%%%%%%



\subsubsection{Description}

The description provides a mechanism for additional, optional information to be attached to a Usage Record base property.  The value of this meta-property MAY provide clues to the semantic context to use while interpreting or examining the value of the owning base property.



%%%%%%%%%%%%%%%%%%



\subsubsection{Metric}

This meta-property identifies the type of measurement used for quantifying the associated resource consumption if there are multiple methods to measure resource usage. As an example, disk usage may be measured as total, average, minimum or maximum usage. However, even if pertinent to the assessed charge, this meta-property does not attempt to differentiate between requested and utilized quantities of resource usage.



%%%%%%%%%%%%%%%%%%



\subsubsection{Time Stamps}

Time stamps should follow the ISO 8601\cite{wolf1998date} standard as well. This includes enumerating the time zone, as specified in the standard.



%%%%%%%%%%%%%%%%%%



\subsection{Conventions}

Unless otherwise stated, all fields are optional. A required field is only required if the corresponding block is present.



%%%%%%%%%%%%%%%%%%



\subsection{Supported Data Types}

\begin{enumerate}
\item String: Data of this type has no required restrictions on the length or available characters.
\item Integer
\item Positive integer: Data of this type must have a value of zero or greater.
\item Non-zero integer: Data of this type must have a value of one or greater.
\item Float: Data of this type must be a decimal number.
\item Timestamp: Data of this type must comply with the UTC time zone format specified in ISO 8601.
\item DomainName: Data of this type must comply with RFC 1034\cite{mockapetrisrfc} format for fully qualified domain names.  The constraints are a maximum 255 characters long, containing only alphabetic and numeric characters, the “-“,  and the “.” characters.
\end{enumerate}


%%%%%%%%%%%%%%%%%%