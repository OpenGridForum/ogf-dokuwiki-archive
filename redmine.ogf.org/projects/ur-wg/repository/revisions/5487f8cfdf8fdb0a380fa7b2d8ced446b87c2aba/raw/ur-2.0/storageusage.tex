% !TeX root = draft-ur-v2.tex

\section{StorageUsageBlock}

This block contains the properties related to storage usage.

\begin{XMLexample}
<ur:StorageUsageBlock>
<!—Storage Record properties go in here -->
</ur:StorageUsageBlock>
\end{XMLexample}



%%%%%%%%%%%%%%%



\subsection{StorageShare}

The part of the storage system which is accounted for in the record. For a storage system, which is split into several logical parts, this can be used to account for consumption on each of these parts. The value should be able to identity the share of the storage system, given the storage system property.

\begin{itemize}
\item \emph{StorageShare} MUST be a string.
\end{itemize}

\begin{XMLexample}
<ur:StorageShare>pool-003</ur:StorageShare>
\end{XMLexample}



%%%%%%%%%%%%%%%



\subsection{StorageMedia}

The media type of storage that is accounted for in the record (e.g. ``disk'' or ``tape''.) This allows for accounting of different backend storage types. 


\begin{itemize}
\item \emph{StorageMedia} MUST be a string.
\end{itemize}

\begin{XMLexample}
<ur:StorageMedia>disk</ur:StorageMedia>
\end{XMLexample}



%%%%%%%%%%%%%%%



\subsection{StorageClass}

The class of the stored data, e.g. ``pinned'', ``replicated''  or ``precious''. \emph{StorageClass} is a descriptive value which allows details about the class of the stored data to be provided.

\begin{itemize}
\item \emph{StorageClass} MUST be a string.
\item This value should be defined in a community specific profile.
%\item The values ``pinned'', ``replicated''  and ``precious'' MUST be supported.
\end{itemize}

\begin{XMLexample}
<ur:StorageClass>replicated</ur:StorageClass>
\end{XMLexample}



%%%%%%%%%%%%%%%



\subsection{DirectoryPath}

The directory path being accounted for. If the property is included in the record, the record should account for all usage in the directory and only that directory. For systems not based on directories such as databases or cloud storages this might be a name defining the collection.

\begin{itemize}
\item \emph{DirectoryPath} MUST be a string.
\end{itemize}

\begin{XMLexample}
<ur:DirectoryPath>/projectA</ur:DirectoryPath>
\end{XMLexample}



%%%%%%%%%%%%%%%



\subsection{FileCount}

The number of files which are accounted for in the record.

\begin{itemize}
\item \emph{FileCount} MUST be a positive non-zero integer.
\end{itemize}

\begin{XMLexample}
<ur:FileCount>42</ur:FileCount>
\end{XMLexample}



%%%%%%%%%%%%%%%



\subsection{StorageResourceCapacityUsed} \label{StorageResourceCapacityUsed}

The number of bytes used on the storage system or storage share where appropriate. This is the main metric for measuring storage resource consumption. \emph{StorageResourceCapacityUsed} should include all resources for which the identity of the record is accountable.

It should include all resources for which the identity of the record is accountable for.

\emph{StorageResourceCapacityUsed} can include reserved space, file metadata, space used for redundancy in RAID setups, tape holes, or similar. The decision about including such ``additional'' space is left to the resource owner but should be made known to the user e.g. via the usage policy or a service level agreement. In contrast the element \emph{StorageLogicalCapacityUsed} denotes the pure file size (see chapter  \ref{StorageLogicalCapacityUsed}). If available, reserved space can be recorded explicitly with \emph{StorageResourceCapacityAllocated} (see chapter  \ref{StorageResourceCapacityAllocated}).

\begin{itemize}
\item \emph{StorageResourceCapacityUsed} MUST be present in the \emph{StorageUsageBlock}.
\item \emph{StorageResourceCapacityUsed} MUST be a non-negative integer.
\item \emph{StorageResourceCapacityUsed} SHOULD include all resources that are used to store the files. 
\item \emph{StorageResourceCapacityUsed} MAY also include resources that are no longer in use but are unavailable for reuse (e.g., if a file is removed from tape, the tape may not be immediately available for reuse), as documented in the appropriate service level agreement or usage policy documents.
\end{itemize}

\begin{XMLexample}
<ur:StorageResourceCapacityUsed>14728</ur:StorageResourceCapacityUsed>
\end{XMLexample}

\subsubsection*{Implementation Note:}
By using bytes we avoid any possible inconsistencies which may arise due to the arbitrary choice of 1000 or 1024 as a base. However, this also means that the number reported can be very large. Therefore any implementation should use at least a 128-bit integer to hold this variable (a signed 64-bit integer will overflow at 8 Exabytes).



%%%%%%%%%%%%%%%



\subsection{StorageLogicalCapacityUsed} \label{StorageLogicalCapacityUsed} 

The number of “logical” bytes used on the storage system or storage share where appropriate.
The term "logical" is used to denote the sum in bytes of the stored files files stored, i.e. excluding reservation, any underlying replicas of files, RAID overhead etc.

\begin{itemize}
\item \emph{StorageLogicalCapacityUsed} MUST be a non-negative integer.
\end{itemize}

\begin{XMLexample}
<ur:StorageLogicalCapacityUsed>13617</ur:StorageLogicalCapacityUsed>
\end{XMLexample}

\subsubsection*{Implementation Note:}
Same as for \emph{StorageResourceCapacityUsed} property (see chapter \ref{StorageResourceCapacityUsed}).



%%%%%%%%%%%%%%%



\subsection{StorageResourceCapacityAllocated} \label{StorageResourceCapacityAllocated} 

The number of bytes allocated on the storage system or storage share where appropriate. Depending on the implementation this property may be equal to \emph{StorageResourceCapacityUsed}. \emph{StorageResourceCapacityAllocated} should only take into account space allocated to the entity described in the record, not resources used for redundancy in RAID setups, tape holes, or similar.

\begin{itemize}
\item \emph{StorageLogicalCapacityUsed} MUST be a non-negative integer.
\end{itemize}

\begin{XMLexample}
<ur:StorageResourceCapacityAllocated>14624</sr:StorageResourceCapacityAllocated>
\end{XMLexample}

\subsubsection*{Implementation Note:}
Same as for \emph{StorageResourceCapacityUsed} (see chapter \ref{StorageResourceCapacityUsed}).



%%%%%%%%%%%%%%%



\subsection{StartTime}

See \emph{StartTime} in \emph{ComputeUsageBlock} (see chapter \ref{ComputeStartTime}).



%%%%%%%%%%%%%%%



\subsection{EndTime}

See \emph{EndTime} in \emph{ComputeUsageBlock} (see chapter \ref{ComputeEndTime}).



%%%%%%%%%%%%%%%



\subsection{Host}

The system on which the resources have been consumed. This value should be chosen in such a way that it globally identifies the system, on which resources are being consumed (e.g. the Fully Qualified Domain Name of the system could be used). 

\begin{itemize}
\item \emph{Host} MUST be a string. 
\end{itemize}

\begin{XMLexample}
<ur:Host>host.example.org</ur:Host>
\end{XMLexample}


%%%%%%%%%%%%%%%



\subsection{HostType}

See \emph{HostType} in \emph{ComputeUsageBlock} (see chapter \ref{ComputeHostType}).



%%%%%%%%%%%%%%%



\subsection{Charge}

See \emph{Charge} in \emph{ComputeUsageBlock} (see chapter \ref{ComputeCharge}).
