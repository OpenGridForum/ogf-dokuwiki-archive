% !TeX root = draft-ur-v2.tex

\section{ComputeUsageBlock}

This block contains the properties related to compute usage.

\begin{XMLexample}
<ur:ComputeUsageBlock>
<!—Compute Record properties go in here -->
</ur:ComputeUsageBlock>
\end{XMLexample}



%%%%%%%%%%%%%%%



\subsection{CpuDuration}

The CPU time consumed. If the task ran on many cores/processors/nodes, all separate consumptions shall be aggregated in this value. This has an impact for example on MPI usage, where the consumption of all the ``nodes'' get aggregated into this CPU consumption.
\begin{itemize}
\item \emph{CpuDuration} MUST contain a time duration as defined in ISO 8601:2004\cite{wolf1998date}.
\end{itemize}

\begin{XMLexample}
<ur:CpuDuration>PT3600S</ur:CpuDuration>
\end{XMLexample}



%%%%%%%%%%%%%%%



\subsection{WallDuration}

WallClock time elapsed during the process execution. In the case of parallel applications (like MPI) \emph{WallDuration} might be lower than \emph{CpuDuration}.
\begin{itemize}
\item \emph{WallDuration} MUST contain a time duration as defined in ISO 8601:2004\cite{wolf1998date}.
\end{itemize}

\begin{XMLexample}
<ur:WallDuration>PT3600S</ur:WallDuration>
\end{XMLexample}



%%%%%%%%%%%%%%%



\subsection{StartTime} \label{ComputeStartTime}

A timestamp indicating the time at which the measured resource consumption started. Together with \emph{EndTime} this defines a period over which the resource has been consumed.
\begin{itemize}
\item \emph{StartTime} MUST be present in the \emph{ComputeUsageBlock}.
\item \emph{StartTime} MUST be an ISO 8601:2004 timestamp.
\item The time zone may be specified as Z (UTC) or (+|-)hh:mm. Time zones that are not specified are considered undetermined.
\end{itemize}

\begin{XMLexample}
<ur:StartTime>2013-05-31T11:00:00</ur:StartTime>
\end{XMLexample}



%%%%%%%%%%%%%%%



\subsection{EndTime} \label{ComputeEndTime}

A timestamp indicating the time at which the measured resource consumption ended. Together with \emph{StartTime} this defines a period over which the resource has been consumed.
\begin{itemize}
\item \emph{EndTime} MUST be present in the \emph{ComputeUsageBlock}.
\item \emph{EndTime} MUST be an ISO 8601:2004 timestamp.
\item The time zone may be specified as Z (UTC) or (+|-)hh:mm. Time zones that are not specified are considered undetermined.
\end{itemize}

\begin{XMLexample}
<ur:EndTime>2013-05-31T12:00:00</ur:EndTime>
\end{XMLexample}



%%%%%%%%%%%%%%%



\subsection{ExecutionHost} \label{ComputeHost}

This property is a container for various information about the host where the application was executed.  As an example, in case of MPI processes, more than one \emph{ExecutionHost} property can be specified. 
\begin{itemize}
\item \emph{ExecutionHost} MAY be present multiple times.
\item \emph{ExecutionHost} MUST NOT have a value.
\item \emph{ExecutionHost} MUST contain at least a child element.
\end{itemize}

\begin{XMLexample}
<ur:ExecutionHost>
  <!-- Various host properties go in here -->
</ur:ExecutionHost>
\end{XMLexample}

\subsubsection{Hostname}

The name of the \emph{ExecutionHost}.
\begin{itemize}
\item \emph{Hostname} MUST be a string.
\item \emph{Hostname} MUST be a child of \emph{ExecutionHost}.
\item \emph{Hostname} MUST be present if \emph{ExecutionHost} is present.
\item The attribute \emph{primary} MAY be present in this element.
\item The attribute \emph{primary} MUST be a boolean.
\end{itemize}

\begin{XMLexample}
<ur:Hostname primary=false>"compute-0-1.abel.uio.no"</ur:Hostname>
\end{XMLexample}

\subsubsection{ProcessId}

The process ID of the process running at the host. For example this could be used in case of MPI processes that use multiple hosts.

\begin{itemize}
\item \emph{ProcessId} MUST be a child of \emph{ExecutionHost}.
\item \emph{ProcessId}  MAY be present multiple times in \emph{Host}.
\item \emph{ProcessId}  MUST be a non-zero integer.
\end{itemize}

\begin{XMLexample}
<ur:ProcessId>1042</ur:ProcessId>
\end{XMLexample}

\subsubsection{Benchmark}

This element is used to insert computing benchmarks associated to the host.
\begin{itemize}
\item \emph{Benchmark} MUST a child of \emph{ExecutionHost}.
\item \emph{Benchmark} MAY be present multiple times in \emph{ExecutionHost}.
\item \emph{Benchmark} MUST be a float.
\item The attribute \emph{type} MUST be present in the element.
\item The attribute \emph{type} MUST be a string.
\item This value should be defined in a community specific profile.
%\item At least the following types should be supported:
%\begin{itemize}
%\item \emph{Si2k} – SpecInt2000
%\item \emph{Sf2k} – SpecFloat2000
%\item \emph{HEPSPEC} – HEPSpec
%\end{itemize}
\end{itemize}

\begin{XMLexample}
<ur:Benchmark type="Si2k">3.14</ur:Benchmark>
\end{XMLexample}



%%%%%%%%%%%%%%%



\subsection{HostType} \label{ComputeHostType}

The type of service according to a namespace-based classification. The namespace MAY be related to a middleware name, an organization or other concepts. org.ogf.glue.* is reserved for types defined by the OGF GLUE Working Group\footnote{https://forge.ogf.org/sf/go/projects.glue-wg/wiki}.
\begin{itemize}
\item \emph{HostType} MUST be a string.
\end{itemize}

\begin{XMLexample}
<ur:HostType>"org.nordugrid.arex"</ur:HostType>
\end{XMLexample}



%%%%%%%%%%%%%%%



\subsection{Processors}

The number of processors used or requested. A processor definition may be dependent on
the machine architecture. Typically, \emph{Processors} is equivalent to the number of physical
CPUs used. For example, if a process uses two cluster ``nodes'', each node having 16 CPUs
each, the total number of processors would be 32.
In addition, a processor can consist of several cores which may be used independently. In the example above the use of dual-core processors would thus increase the number to 64.
\begin{itemize}
\item \emph{Processors} MUST be a non-zero integer.
\end{itemize}

\begin{XMLexample}
<ur:Processors>1</ur:Processors>
\end{XMLexample}



%%%%%%%%%%%%%%%



\subsection{NodeCount}

Number of nodes used. A node definition may be dependent on the architecture, but typically a node is a physical machine. For example a cluster of 16 physical machines with each machine having one processor each is a 16 ``node'' machine, each with one ``processor''. A 16 processor SMP machine however, is one physical node (machine) with 16 processors.
\begin{itemize}
\item \emph{NodeCount} MUST be a non-zero integer.
\end{itemize}

\begin{XMLexample}
<ur:NodeCount>1</ur:NodeCount>
\end{XMLexample}



%%%%%%%%%%%%%%%

\subsection{ExitStatus}

This element allows the numeric exit status value for the application to be specified.
\begin{itemize}
\item \emph{ExitStatus} SHOULD be present in the \emph{ComputeUsageBlock}.
\item \emph{ExitStatus} MUST be an integer.
\end{itemize}

%%%%%%%%%%%%%%%


\subsection{Charge} \label{ComputeCharge}

The charge applied to the users resource usage. Charge is a site dependent value and may be used for economic accounting purposes.

\begin{itemize}
\item \emph{Charge} MUST be a float.
\end{itemize}

\begin{XMLexample}
<ur:Charge>1.75</ur:Charge>
\end{XMLexample}
