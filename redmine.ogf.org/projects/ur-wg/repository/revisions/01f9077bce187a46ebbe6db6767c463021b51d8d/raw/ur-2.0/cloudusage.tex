% !TeX root = draft-ur-v2.tex

\section{CloudUsageBlock}

This block contains the properties related to cloud usage.

\subsubsection*{Example}
\begin{XMLexample}
<ur:CloudUsageBlock>
<!—Cloud Record properties go in here -->
</ur:CloudUsageBlock>
\end{XMLexample}



%%%%%%%%%%%%%%%



\subsection{LocalVirtualMachineId}

The local identity of the Virtual Machine. For example, this may be the ID assigned to by the Cloud management system.

\begin{itemize}
\item \emph{LocalVirtualMachineId} MUST be a string.
\end{itemize}

\begin{XMLexample}
<ur:LocalVirtualMachineId>"ab1234"</ur:LocalVirtualMachineId>
\end{XMLexample}



%%%%%%%%%%%%%%%



\subsection{GlobalVirtualMachineId}

The global identity of the Virtual Machine. The property should uniquely identify the Virtual Machine globally, such that clashes do not happen accidentally. This could be a combination of time stamp, local Virtual Machine ID and host name.

\begin{itemize}
\item \emph{GlobalVirtualMachineId} MUST be a string.
\end{itemize}

\begin{XMLexample}
<ur:GlobalVirtualMachineId>
    host.example.org/ab1234/2013-05-09T09:06:52Z
</ur:GlobalVirtualMachineId>
\end{XMLexample}



%%%%%%%%%%%%%%%



\subsection{Status} \label{CloudStatus}

The status of the Virtual Machine.

\begin{itemize}
\item \emph{Status} MUST be present in the \emph{CloudUsageBlock}.
\item \emph{Status} MUST be string.
%\item \emph{Status} MUST support the following values:
%  \begin{itemize}
%  \item completed – The execution is completed.
%  \item started – The execution started at the time this usage record was generated.
%  \item suspended – The execution was suspended at the time this usage record was generated.
%  \end{itemize}
%\item \emph{Status} MAY support other values, as agreed upon within the implementation context.
\item This value should be defined in a community specific profile.
\end{itemize}

\begin{XMLexample}
<ur:Status>"started"</ur:Status>
\end{XMLexample}



%%%%%%%%%%%%%%%



\subsection{SuspendDuration}

The amount of time in which the Virtual Machine status was  ``suspended''.

\begin{itemize}
\item \emph{SuspendDuration} MUST be present if the property \emph{Status} (see chapter \ref{CloudStatus}) of the Virtual Machine is ``suspended''.
\item \emph{SuspendDuration} MUST be a time duration as defined in ISO 8601:2004.
%\item time zone may be specified as Z (UTC) or (+|-)hh:mm. Time zones that are not specified are considered undetermined.

\end{itemize}

\begin{XMLexample}
<ur:SuspendDuration>PT3600S</ur:SuspendDuration>
\end{XMLexample}


%%%%%%%%%%%%%%%



\subsection{ImageId}

The ID of the image used to instanciate the Virtual Machine.

\begin{itemize}
\item \emph{ImageId} MUST be a string.
\end{itemize}

\begin{XMLexample}
<ur:ImageId>"UbuntuImage2013"</ur:ImageId>
\end{XMLexample}

%%%%%%%%%%%%%%%



\subsection{MachineName}

See \emph{MachineName} in \emph{JobUsageBlock} (see chapter \ref{JobMachineName}).



%%%%%%%%%%%%%%%



\subsection{SubmitHost}

See \emph{SubmitHost} in \emph{JobUsageBlock} (see chapter \ref{JobSubmitHost}).



%%%%%%%%%%%%%%%



\subsection{TimeInstant}

See \emph{TimeInstant} in \emph{JobUsageBlock} (see chapter \ref{JobTimeInstant}).



%%%%%%%%%%%%%%%



\subsection{ServiceLevel}

See \emph{ServiceLevel} in \emph{JobUsageBlock} (see chapter \ref{JobServiceLevel}).




