%!TEX root = nml-base.tex

\section{Introduction}%
\label{sec:introduction}

This document describes the base schema of the Network Markup Language (NML).
Section~\ref{sub:classes} defines the NML classes and their attributes and parameters.
Section~\ref{sub:relations} describes the relations defined between NML classes.

An NML network description can be expressed in XML\cite{xml}, and RDF/XML\cite{rdfxml} syntax.
Section~\ref{s:xmlschema} describes the XSD schema for the XML syntax.
Section~\ref{s:owlschema} describes the OWL 2 schema for the RDF/XML syntax.

These basic classes defined in this document may be extended, or sub-classed, 
to represent technology specific classes.

Section~\ref{s:examples} provides example use cases. This section is informative. 
Only sections~\ref{s:schema}, \ref{s:identifiers}, \ref{s:syntax}, and appendices \ref{s:xmlschema} and \ref{s:owlschema} are normative and considered 
part of the recommendation.

Appendix~\ref{s:g800terms} is informative and explains the relation between terms defined in this document and those defined in the ITU-T G.800 recommendation~\cite{g800}.

\subsection{Context}
\label{sec:context}

The Network Markup Language (NML) has been defined in the context of research and 
education networks to describe so-called hybrid network topologies. The NML is defined
as an abstract and generic model, so it can be applied for other network topologies as well.
See \cite{gfd.165} for an detailed overview including prior work.

\subsection{Scope}
\label{sec:scope}

The Network Markup Language is designed to create a functional description of 
multi-layer networks and multi-domain networks. An example of a multi-layered 
network can be a virtualised network, but also using different technologies. 
The multi-domain network descriptions can include aggregated or abstracted network topologies.
NML can not only describe a primarily static network topology, but also its potential capabilities (services) 
and its configuration.

NML is aimed at logical connection-oriented network topologies, more precisely topologies
where switching is performed on a label associated with a flow, such as a VLAN, wavelength or time slot. 
NML can also be used to describe physical networks or packet-oriented networks, 
although the current base schema does not contain classes or properties 
to explicitly deal with signal degradation, or complex routing tables.

NML only attempts to describe the data plane of a computer network, not the control 
plane. It does contain extension mechanism to easily tie it with network provisioning 
standards and with network monitoring standards.

Finally, this document omits a definition for the terms \emph{Network} or \emph{capacity}. 
This has been a conscious choice. The term \emph{Network} has become 
so widely used for so many diverse meanings that it is impossible to create a 
definition that everyone can agree on, while still expressing something useful.
See \emph{Topology} for the concept of a network domain and a \emph{Link} with multiple 
sources and sinks for the concept of a local area network.
The term \emph{capacity} is used by different technologies in such a different 
way (e.g.\ including or excluding the header and footer overhead) that it is better 
to let technology-specific extensions make an explicit definition.

\subsection{Notational Conventions}%
\label{sec:rfc2119}

The keywords “\MUST{}”, “\MUSTNOT{}”, “\REQUIRED{}”, “\SHALL{}”, “\SHALLNOT{}”, 
“\SHOULD{}”, “\SHOULDNOT{}”, “\RECOMMENDED{}”, “\MAY{}”,  and “\OPTIONAL{}” are 
to be interpreted as described in \cite{rfc2119}.
% except that the words do not appear in uppercase. 

This schema defines classes, attributes, relations, parameters and logic.
Objects are instances of classes, and the type of an object is a class.

Names of classes are capitalised and written in italics (e.g.\ the \emph{Node} class).
Names of relations are written in camel case and in italics (e.g.\ the \emph{hasNode} relation).
Names of identifiers and string literals are written in monspaces font (e.g. \texttt{Port\_X:in}).

Diagrams in this document follow the diagrammatic conventions of UML class diagrams.
\begin{itemize}
\item A subclass-superclass relationship is represented by a line with hollow triangle shape pointing to the superclass.
\item A whole-part relationship is represented by a line with a hollow diamond shape pointing to the whole (group).
\item A entity-relationship is represented by a line, optionally with numbers at each end indicating the cardinality of the relation. A named entity-relationship has a verb next to the line, and a filled triangle pointing to the object of the verb. (e.g. the entitity-relationship
\nmlrelation{BidirectionalPort}{*}{hasPort}{2}{Port} is named \emph{hasPort}, and each \emph{BidirectionalPort} is related to exactly 2 \emph{Port}s, and each \emph{Port} may be associated with zero, one or more \emph{BidirectionalPort}s.)
\end{itemize}

