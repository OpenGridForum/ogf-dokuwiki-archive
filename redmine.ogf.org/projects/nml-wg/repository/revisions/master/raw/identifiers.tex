%!TEX root = nml-base.tex

\section{Identifiers}%
\label{s:identifiers}

\subsection{Schema Identifier}%
\label{sub:schema_uri}

% The \path command is from the hyperref package, it is the non-clickable variant of the \url command. It does do nice wrapping for URLs/paths/etc.
The namespace for the schema defined in document is \path{http://schemas.ogf.org/nml/2013/05/base\#}.

All classes, relations, parameters and attributes defined in this document reside in this namespace. For example, the Link class is identified by \texttt{http://schemas.ogf.org/nml/2013/05/base\#Link}

\subsection{Instance Identifiers}%
\label{sub:identifiers}

Section \ref{class:network_object} requires that instances of Network Objects \SHOULD{} have an \emph{id} attribute, which \MUST{} be a unique URI.

% It is possible to describe additional information on an instance, in this case an \emph{idRef} attribute can be used.

Implementations that receive a network topology description \MUST{} be prepared to accept any valid URI as an identifier.

Implementations that publish a network topology description instance identifiers \MAY{} adhere to the syntax of Global Network Identifiers as defined in~\cite{gfd.202}, which ensures global uniqueness and easy recognition as Network Object instances.

Two different Network Objects instances \MUST{} have two different identifiers.

Once an identifier is assigned to a resource, it \MUSTNOT{} be re-assigned to another resource.

A URI \MAY{} be interpreted as an International Resource Identifier (IRI) for display purposes, but URIs from external source domains \MUSTNOT{} be IRI-normalised before transmitting to others.

\subsubsection{Lexical Equivalence}

Two identifier are lexical equivalent if they are binary equivalent after case folding\footnote{\emph{Case folding} is primarily used for caseless comparison of text. \emph{Case mapping} is used for display purposes.}~\cite{casefolding}.

Other interpretation (such as percent-decoding or Punycode decoding~\cite{punycode}) \MUSTNOT{} take place.

For the purpose of equivalence comparison, any possible fragment part or query part of the URI is considered part of the URI.

For example the following identifiers are equivalent:

\begin{quote}
  \texttt{1 - urn:ogf:network:example.net:2013:local\_string\_1234}\\
  \texttt{2 - URN:OGF:network:EXAMPLE.NET:2013:Local\_String\_1234}
\end{quote}

While the following identifiers are not equivalent (in this case, the percentage encoding even makes URI \#3 an invalid Global Network Identifier.):

\begin{quote}
  \texttt{1 - urn:ogf:network:example.net:2013:local\_string\_1234}\\
  \texttt{3 - urn:ogf:network:example.net:2013:local\%5Fstring\%5F1234}
\end{quote}

\subsubsection{Further Restrictions}

An assigning organisation \MUSTNOT{} assign Network Object Identifier longer than 255 characters in length.

Parsers \MUST{} be prepared to accept identifiers of up to 255 characters in length.

A Parser \SHOULD{} verify if an identifier adheres to the general URI syntax rules, as specified in RFC 3986~\cite{rfc3986}.

Parsers \SHOULD{} reject identifiers which do not adhere to the specified rules. A parser encountering an invalid identifier \SHOULD{} reply with an error code that includes the malformed identifier, but \MAY{} accept the rest of the message, after purging all references to the Network Object with the malformed identifier.

\subsubsection{Interpreting Identifiers}

A Network Object identifier \MUST{} be treated as a opaque string, only used to uniquely identify a Network Object. The local-part of a Global Network Identifier \MAY{} have certain meaning to it's assigning organisation, but \MUSTNOT{} be interpreted by any other organisation.

\subsubsection{Network Object Attribute Change}

A Network Object may change during its lifetime. If these changes are so drastic that the assigning organisation considers it a completely new Network Object, the assigning organisation should be assigned a new identifier. In this case, other organisations MUST treat this object as completely new Network Resource.

If the assigning organisation considers the changes are small, it \MUST{} retain the same identifier for the Network Object, and use some mechanism to signal it's peers of the changes in the attributes of the Network Object. An appropriate mechanism is to send a new description of the Topology or the Network Object with an updated \emph{version} attribute.

\subsection{Unnamed Objects}%
\label{sub:unnamed_objects}

Network Objects that do not have a regular URI as id attribute, may have either:
\begin{itemize}
    \item Have no id attribute. These are so-called \emph{unnamed} network objects.
    \item Have an id attribute which is a fragment identifier only, thus an URI starting with a crosshatch (\texttt{\#}) character. These are so-called \emph{ad-hoc named} network objects.
\end{itemize}

A \emph{unnamed} network object can not be referenced. A network objects generally \SHOULDNOT{} be unnamed, since there is no possibility for an external party to refer to the object.

A \emph{ad-hoc named} network object can only be referenced from within the same topology description. \emph{ad-hoc} \emph{id}s must be considered a syntactical construct, not as a persistent identifier. The \MUSTNOT{} be referred to from another scope or another topology description. \emph{ad-hoc id}s \SHOULDNOT{} be stored. If a two peers exchange topology messages, it is perfectly valid to change the ad-hoc id in each message (since they are only valid within scope of that message anyway).

A possible reason to use unnamed or ad-hoc named network objects it to make a statement such as \textit{“Port A and Port B are grouped in a BidirectionalPort”} without actually assigning an identifier to this BidirectionalPort.



