%!TEX TS-program = pdflatex
\documentclass[12pt]{article}  % larger font to compensate for long lines with fullpage
\usepackage{ifpdf,ifxetex}
\ifxetex
% \usepackage{CJK} % Chinese/Japanese/Korean language support for single line of Japanese
\usepackage{fontspec}
\usepackage[nospace]{xeCJK}
% \setCJKfamilyfont
% \setCJKmainfont{Osaka}
\setmainfont[Mapping=tex-text]{Times New Roman}
\setCJKmainfont{STSong}
\setCJKsansfont{STSong}
\setCJKmonofont{STSong}
% \setCJKmainfont{FZKaiTi}
\else % inputenc is incompatible with xetex (which always takes UTF-8 as input)
\usepackage[T1]{fontenc}
\usepackage[utf8]{inputenc}
\usepackage{CJKutf8} % Chinese/Japanese/Korean language support for single line of Japanese
\fi
\usepackage{ifthen}
\usepackage{graphicx}
\usepackage[pdfborder={0 0 0}]{hyperref} % use hyperref without borders
\usepackage{url}
\usepackage{authblk}
\usepackage{framed}  % frames sections with page breaks
\usepackage{soul} % for strike-through text (provides \st)
% Define some basics for your document:

\title{Procedure for Registration of Subnamespace Identifiers in the URN:OGF Hierarchy}
\author{Freek Dijkstra \and Richard Hughes-Jones \and Gregory B.\ Newby \and Joel Replogle}
\newcommand{\shortdoctitle}{URN:OGF Subnamespace registration}  % Title used in page header
% \date{} and \author{} are currently ignored
\newcommand{\authorsshort}{Freek Dijkstra, SARA\\Richard Hughes-Jones, DANTE\\Gregory B.\ Newby, Arctic Region Supercomputing Center\\Joel Replogle, Open Grid Forum}
\newcommand{\publicationdate}{December 2011}  % Date of first publication of the document
% \newcommand{\revisiondate}{August 2012}  % Optional: date of last revision of the document
\newcommand{\copyrightyears}{2011}  % Years used in copyright notice
% \newcommand{\docseries}{GWD-C}  % GWD-R, GWD-I or GWD-C (for working drafts)
\newcommand{\docseries}{GFD.191}  % GFD.000 (for approved documents)
\newcommand{\groupname}{GFSG}  % Optional: name of the authoring working or research group
%\newcommand{\groupurl}{\href{mailto:example-wg@ogf.org}{example-wg@ogf.org}}  % Optional: URL or email address of the authoring working or research group
%\newcommand{\documenturl}{}  % Optional: URL of this document


% Read pictures from img/ and current directory
\graphicspath{{img/}{./}}

%%% GWD/GFD header follows %%%
% Feel free to make changes, as long as your document follows the guidelines of GFP.152

\usepackage[numbers]{natbib} % Use [1] for references, 
\bibliographystyle{plainnat} % References show full author name(s) and document URL

\usepackage[sf,compact]{titlesec} % Use sans-serif for section headers

\usepackage[titles]{tocloft} % Format table of contents
% (tocloft is used, since titletoc is incompatible with xetex.)
\renewcommand{\cftsecfont}{\sffamily}
\renewcommand{\cftsubsecfont}{\sffamily}
\renewcommand{\cftsubsubsecfont}{\sffamily}
\renewcommand{\cftsecpagefont}{\sffamily}
\renewcommand{\cftsubsecpagefont}{\sffamily}
\renewcommand{\cftsubsubsecpagefont}{\sffamily}
\renewcommand{\cftsecleader}{\cftdotfill{\cftsubsecdotsep}} % dots for sections the same as for subsections
\setlength{\cftbeforesecskip}{0.5ex}


\usepackage{parskip} % Blank lines between paragraphs, no indentation.

% font style for headers and footers
\newcommand{\headerstyle}{\sffamily} % sans-serif

% Set page margins
\usepackage{fancyhdr}
\addtolength{\headheight}{15pt}
\renewcommand{\headrulewidth}{0pt}
% \setlength{\headrulewidth}{0pt}
\setlength{\headsep}{20pt}
\usepackage[headings]{fullpage}  % small margins

% Macro to check if (optional) values above are defined or not.
\newcommand{\ifnonempty}[2]{\ifthenelse{\isundefined{#1}}{}{\ifthenelse{\equal{#1}{}}{}{#2}}}

% Define page header and footers
\pagestyle{fancyplain}
\fancyhf{}
\lhead{\fancyplain{}{\headerstyle\docseries}}
% use \revisiondate if defined, otherwise \publicationdate for right header:
\rhead{\fancyplain{}{\headerstyle\ifthenelse{\isundefined{\revisiondate }}{\publicationdate}{\ifthenelse{\equal{\revisiondate}{}}{\publicationdate}{\revisiondate}}}}
\lfoot{\headerstyle\ifnonempty{\groupurl}{\groupurl}}
\rfoot{\headerstyle\thepage}
\thispagestyle{plain}

\ifpdf
\hypersetup{
    pdftitle = {Procedure for Registration of Subnamespace Identifiers in the URN:OGF Hierarchy},
    pdfauthor = {Freek Dijkstra, Richard Hughes-Jones, Gregory B. Newby, and Joel Replogle},
    pdfsubject = {urn:ogf subnamespace delegation procedure},
    pdfkeywords = {URN OGF procedure namespace}
}

\fi

\begin{document}

% Title page header
{\noindent
\begin{minipage}[t]{1.5in}
\headerstyle
\docseries \\
\ifnonempty{\groupname}{\groupname \\}
\ifnonempty{\groupurl}{\groupurl \\}
\ifnonempty{\documenturl}{\documenturl \\}
\end{minipage}
\hfill
\raggedleft
\begin{minipage}[t]{4.5in}
\raggedleft
\headerstyle
\authorsshort \\
\vspace{1em}
\publicationdate \\
\ifnonempty{\revisiondate}{Revised \revisiondate \\}
\end{minipage}
}

\vspace{1em}
\begin{center}
\makeatletter
\Large\bf\textsf \@title
\makeatother
\end{center}

\newcommand{\qq}{\symbol{34}} % 34 is the decimal LaTeX code for "
\newcommand{\q}{\symbol{39}} % 39 is the decimal LaTeX code for '
\newcommand{\underscore}{\symbol{95}} % 39 is the decimal LaTeX code for _

%%% End of header, insert content below this line %%%

\subsection*{Status of This Document}

Community Practice (CP)

% \subsection*{Document Change History}
% 
% This template has been updated to confirm to GFD-C.152~\cite{gfd152}.

\subsection*{Copyright Notice}

Copyright \copyright \ Open Grid Forum (\copyrightyears).  Some Rights Reserved.  
Distribution is unlimited.

\phantomsection\addcontentsline{toc}{section}{Abstract}
\section*{Abstract}

URNs in the OGF namespace take the form 
\\\hspace*{1cm}\texttt{urn:ogf:<snid>:<subnamespace-specific-string>}.

This document describes the procedure how to register subnamespace 
identifiers (\texttt{<snid>}) in the \texttt{urn:ogf:} namespace.

\phantomsection\addcontentsline{toc}{section}{Contents}
\tableofcontents

\newpage

\section{introduction}%
\label{sec:introduction}

Uniform Resource Names (URNs) are persistent, globally unique 
identifiers~\cite{rfc2141,rfc3406}.

An identifier labels a resource, which facilitates unambiguous identification 
of that resource. 
Any resource can be named, including a work, an instance of a work, an entry in 
an ontology or the ontology as a whole.

\subsection{Notational Conventions}
\label{sec:rfc2119}

\newcommand{\MUST}{\textsc{must}}
\newcommand{\MUSTNOT}{\textsc{must not}}
\newcommand{\REQUIRED}{\textsc{required}}
\newcommand{\SHALL}{\textsc{shall}}
\newcommand{\SHALLNOT}{\textsc{shall not}}
\newcommand{\SHOULD}{\textsc{should}}
\newcommand{\SHOULDNOT}{\textsc{should not}}
\newcommand{\RECOMMENDED}{\textsc{recommended}}
\newcommand{\MAY}{\textsc{may}}
\newcommand{\OPTIONAL}{\textsc{optional}}

The keywords “\MUST{}”, “\MUSTNOT{}”, “\REQUIRED{}”, “\SHALL{}”, “\SHALLNOT{}”, 
“\SHOULD{}”, “\SHOULDNOT{}”, “\RECOMMENDED{}”, “\MAY{}”,  and “\OPTIONAL{}” are 
to be interpreted as described in \cite{rfc2119}.
% except that the words do not appear in uppercase. 

\subsection{Globally Uniqueness of URNs}%
\label{sec:uniqueness}

\cite{rfc3406} stipulates:
\begin{itemize}
  \item A URN \MUSTNOT{} be assigned to more than one resource. 
  \item A URN \MUSTNOT{} be re-assigned to a different resource.
  \item A single resource \MAY{} have more than one URN assigned to it for different purposes.
\end{itemize}

For the purpose of this document, we will call an identifier \emph{unique} if
adheres to the first two requirements above. So a unique identifier
identifies only one resource, although one resource does not have to be
represented by a single identifier.

Any organisation that assigns identifiers to its resources (the \emph{assigning 
organisation}) must adhere to the
above requirements. This ensures that the identifiers assigned by this
organisation are unique. However, this only ensures that the identifier is
locally unique, but does not yet guarantee that the identifier is globally
unique.

To ensure global uniqueness of identifiers, each local identifier is
prepended with a \emph{prefix} that in itself is a unique identifier of the
organisation that assigns the local identifiers. In the context of URNs, this
prefix is often referred to as a \emph{namespace identifier}. The group of
all identifiers that start with a certain prefix is called a
\emph{namespace}. 
A second purpose of a prefix is to provide a cue for the
type of resource being identified.

For example, a working group in the OGF may define an ontology with three
terms, identified by their names \emph{Grid}, \emph{Cloud}, and
\emph{Cluster}. A different organisation may use the same terms, but with
different meaning. To prevent ambiguity when using these terms, the working
group may register the namespace identifier \path{urn:ogf:example}, and
use the identifiers \path{urn:ogf:example:grid},
\path{urn:ogf:example:cloud} and \path{urn:ogf:example:cluster}.
When these identifiers are used, it is unambiguous which concept is 
referred to.

This document describes the procedure to register a namespace identifier
within the \texttt{urn:ogf:} namespace.

\subsection{Persistency of URNs}%
\label{sec:persistency}

The goal of URNs is to provide persistent identification of resources. An
identifier must remain valid and unmodified from its creation to well beyond
the lifespan of the resource it identifies. (The identifier may still be used 
for archived information, like monitoring data, even after the resource itself 
is disbanded.) This persistency is the
responsibility of the organisation maintaining the URN namespace as well as
the delegate organisation(s) that assign the identifier.

The requirement that a URN \MUSTNOT{} be re-assigned only partly ensures longevity.
An identifier \MUSTNOT{} change name, and (where applicable) validation and 
resolution procedures \SHOULD{} still yield results, even after the following events:
\begin{itemize}
  \item the resource ceases to exist; 
  \item the properties of the resource change; 
  \item a new version of the resource is created; 
  \item the namespace organisation ceases to exist; 
  \item the organisation that assigned the identifier ceases to exist; or 
  \item the organisation that assigned the identifier changes its name.
\end{itemize}

For example, the Open Grid Forum (OGF) emerged from the merger of the Global
Grid Forum (GGF) and Enterprise Grid Alliance (EGA) in 2006. URI namespaces
created before that time still have \texttt{ggf.org} in their prefix, and this
has not changed. If the Open Grid Forum (OGF) would change name, for example to 
Open Cloud Forum (OCF), the prefix \texttt{urn:ogf} of existing identifiers 
\MUSTNOT{} change, as not to break the persistency of the identifiers.

\section{Selecting a Namespace}%
\label{sec:namespace_selection}

This document describes how to register a prefix in the \texttt{urn:ogf:} namespace.
OGF working groups or interested third parties that wish to assign identifiers must 
first determine if the \texttt{urn:ogf:} namespace is the most suitable namespace.

Potential namespaces include:
\begin{itemize}
  \item \texttt{urn:ogf:} namespace, as described in this document.
  \item \texttt{urn:} namespace, as maintained by the IETF and IANA~\cite{rfc3406}.
  \item \texttt{http://schemas.ogf.org/} namespace~\cite{gfd84}
    (a location identifier that also serves as a resource identifier).
  \item A namespace in the handle system~\cite{rfc3650}.
  \item Non-prefixed namespaces, such as UUIDs~\cite{x.667} or hash values.
\end{itemize}

URNs in the \texttt{urn:ogf:} namespace may be suitable for identifying individual 
grid resources.

URNs in the \texttt{urn:} namespace may be suitable for identifying resources other 
than grid resources.

A URL may be suitable as identifier for ontologies and schemata,
as the URL can double as the location where the normative schemata definition
can be found. URLs may not be suitable to identify individual resources, as the
same resource may be available from multiple locations, or may not be accessible 
via a protocol with an associated URL scheme, or the resource may outlive the 
lifetime of a given scheme (e.g. a resource which is now accessible with the HTTP 
scheme may later be accessible with the HTTPS scheme).

Systems that like to integrate with the handle system for resource resolution
of digital resources should consider using the (non-URN) identifiers defined in 
the handle system.

A non-prefixed system may be a suitable identifier namespace if a centrally 
maintained registry is considered a drawback.

DNS-based identifiers are generally \textsc{not recommended} for persistent 
identifiers.

\section{Canonical Syntax of URN:OGF identifiers}%
\label{sec:syntax}

The canonical syntax of URNs in the \texttt{urn:ogf:} namespace is specified 
in sections 2.4 and 2.11 of \cite{rfc6453}. 
The sections of this RFC are replicated below.

\begin{quote}
  \texttt{OGF-URN  =  \qq{}urn:ogf:\qq{} SNID \qq{}:\qq{} SUBNAMESPACE-SPECIFIC-STRING}
\end{quote}

where \texttt{<SNID>} is a unique subnamespace identifier, and 
\texttt{<SUBNAMESPACE-SPECIFIC-STRING>} is a unique local identifier within the 
namespace with the prefix \texttt{\qq{}urn:ogf:\qq{} SNID \qq{}:\qq{}}.

Clearly, each \texttt{SNID \qq{}:\qq{} SUBNAMESPACE-SPECIFIC-STRING} \MUST{} be unique.

\texttt{<SNID>} has the same syntax as a namespace identifier \texttt{<NID>} 
as defined in \cite{rfc2141}:

\begin{quote}
  \texttt{SNID  =  ( ALPHA / DIGIT )  *31( ALPHA / DIGIT / \qq{}-\qq{} )}
\end{quote}

\texttt{<ALPHA>} and \texttt{<DIGIT>} are defined in Appendix B of \cite{rfc5234}.

Both \texttt{\qq{}urn:ogf:\qq{}} and the subnamespace identifiers are case insensitive.

Section~\ref{sec:procedure} of this document describes the procedure how subnamespace 
identifiers are assigned by the OGF.

The syntax of \texttt{<SUBNAMESPACE-SPECIFIC-STRING>} is dependent on the \texttt{<SNID>}.
\cite{rfc6453} does not pose any additional restrictions to the
\texttt{<SUBNAMESPACE-SPECIFIC-STRING>} other than what is defined in the
\texttt{<NSS>} syntax as defined by \cite{rfc2141}:

\begin{quote}
  \texttt{SUBNAMESPACE-SPECIFIC-STRING  =  1*<URN chars>}
\end{quote}

\texttt{<URN chars>} is defined in Section 2.2 of \cite{rfc2141}.

Documents defining a subnamespace identifier \SHOULD{} specify further
syntactic restrictions in \texttt{<SUBNAMESPACE-SPECIFIC-STRING>}.  It is
\RECOMMENDED{} that these documents forbid the assignment of URNs
containing characters in the \texttt{<reserved>} set (\texttt{\qq{}\%\qq{}}, 
\texttt{\qq{}/\qq{}}, \texttt{\qq{}?\qq{}} and \texttt{\qq{}\#\qq{}})
as defined in \cite{rfc2141}. This is in accordance with section 2.2 of
\cite{rfc3986}.

For forward compatibility (e.g. updates in the URN syntax~\cite{rfc2141bis}), 
it is \RECOMMENDED{} that software
implementations that don't validate specific subnamespace-specific 
strings, validate the syntax according to the generic rules for validating
URIs, as defined in \cite{rfc3986}.  URIs may contain all characters
defined in \texttt{<URN chars>}, including the characters in \texttt{<reserved>}
(albeit they have a special meaning), as well as the characters \texttt{\qq{}\&\qq{}}
and \texttt{\qq{}\textasciitilde\qq{}}.

\section{Procedure for Registering a Namespace Identifier}%
\label{sec:procedure}

The Open Grid Forum delegates parts of its namespace \texttt{urn:ogf:} by
assigning subnamespace identifiers (\texttt{SNID}s). The formal application
for an SNID is made via publication of a Grid Forum Document (GFD). The GFD
\SHOULD{} be an Informational Document, and \MUST{} be published according to the
normal procedure as described in \cite{gfd152} or its successor.

A GFD that registers a namespace \MUST{} at least include the following sections:
\begin{itemize}
    \item The prefix to register
    \item The version of the specification (usually starting with version 1)
    \item Syntax of URNs in the registered namespace
    \item Rules for lexical equivalence of two URNs in the namespace
    \item Procedure for assignment of identifiers (and possible further subdelegation of the registered namespace)
    \item Identifier immutability considerations
    \item Intended use and application area for URNs in the namespace
\end{itemize}
The template defined in section~\ref{sec:template} \MAY{} be used as
(part of) an GFD that registers a namespace.

The Open Grid Forum maintains a registry of namespace identifiers in the 
\texttt{urn:ogf:} namespace at the URL \url{http://www.ogf.org/urn/}.

The OGF \MUST{} be notified of any (proposed) use of a namespace identifier in the 
\texttt{urn:ogf:} namespace, by contacting \href{mailto:urn@ogf.org}{urn@ogf.org}.
The OGF grid forum steering group may, at their discretion “reserve” a
subnamespace identifier for a limited period of time prior to approval of
an Informational GFD document describing the namespace.

This document does not describe the resolution procedure of any namespace
conflicts.

\section{Review Criteria for SNID Proposals}%
\label{sec:review}

The editor of the GFD series \SHOULD{} make the following checks before assigning 
a subnamespace identifier (SNID):
\begin{itemize}
    \item Is there a grid working document that includes all required sections (as listed above)?
    \item Does the intended use fall within scope of the OGF?
    \item Is the application technically sound?
    \begin{itemize}
        \item Is the SNID well-chosen? Does it still make sense in 10 years if the 
          working group is disbanded?
        \item Is the syntax unambiguous? Is the allowed character set defined? 
          What is the maximum length of a URN?
        \item If subdelegation within an SNID is allowed: how is uniqueness and immutability ensured 
          for third parties (who may be less versed in the process of assignment of 
          \emph{globally unique} and \emph{persistent} identifiers).
        \item If percentage-escaping is allowed, is a normalization function defined for the 
          lexical equivalence?
        \item How is immutability ensured? Does the URN contain attributes? Can these 
          attributes change within the lifetime of the identifier (see also 
          section~\ref{sec:persistency})? If so, what measures are in place to ensure 
          immutability?
    \end{itemize}
\end{itemize}

\section{Template for Registering a Namespace Identifier}%
\label{sec:template}

\newenvironment{example}[1][]
{
\begin{leftbar}\begin{sloppypar}\noindent
    Example{#1}:
}{
\end{sloppypar}\end{leftbar}
}

In this section, vertical solid bars at the left hand side indicate text given as 
illustrative examples.

\hrulefill\hspace{1ex} \textsf{Begin of Template} \hspace{1ex}\hrulefill

\subsection*{1. Registration}

\subsubsection*{1.1. Namespace Identifier}

The subnamespace identifier.

\begin{example}
“\texttt{urn:ogf:example:}”
\end{example}

Consideration: This namespace is likely to exist for a prolonged period of
time. It \MUSTNOT{} include volatile data, such as a working group name
(\texttt{example-wg}), although a name of a protocol created by the working 
group is acceptable. 
\texttt{urn:ogf:} is a formal namespace, subnamespace identifiers \MUSTNOT{} 
start with \texttt{x-}.

\subsubsection*{1.2. Document Version}

Registration version number: starting with 1, incrementing by 1 with each new version of the GFD. \\
Registration date: date of publication of the GFD.

Consideration: The version number only increases for approved GFD documents, not for each revision of the working draft.

\subsubsection*{1.3. Declared Registrant}

The name and address of the namespace organisation and/or contact person that
is responsible for making the registration.

This section may be omitted if the registrant is the same as the authors of the GFD.

\subsection*{2. Syntax}

\subsubsection*{2.1. Syntactic Structure}

This section should outline any structural features of identifiers in this namespace.  
It must describe the syntax of all current and future URNs within this namespace 
that are considered valid.
This description may be used to introduce terminology used in other sections.

\begin{example}
\begin{quote}
  \texttt{EXAMPLE-URN = \qq{}urn:ogf:\qq{} SNID \qq{}:\qq{} TYPE \qq{}:\qq{} YEAR \qq{}:\qq{} DOMAIN} \\
  \hspace*{2cm}\texttt{\qq{}:\qq{} INDEX *1(QUERY) *1(FRAGMENT) } \\
  \texttt{SNID = \qq{}example\qq{}} \\
  \texttt{TYPE = ALPHA *15( ALPHA / DIGIT / \qq{}-\qq{} )} \\
  \texttt{YEAR = 4DIGIT} ; Year in Gregorian calendar \\
  \texttt{DOMAIN = LDH-LABEL *( \qq{}.\qq{} LDH-LABEL )} ; domain name. \\
  \texttt{LDH-LABEL = *63( ALPHA / DIGIT / \qq{}-\qq{} )} ; part of a domain name. \\
  \texttt{INDEX = 1*DIGIT} ; Identifier of a historic supercomputer \\
  \texttt{QUERY = \qq{}?\qq{} 1*<URN chars>} \\
  \texttt{FRAGMENT = \qq{}\#\qq{} 1*<URN chars>}
\end{quote}

The total length of a \texttt{<EXAMPLE-URN>} \MUSTNOT{} exceed 255 bytes.
If a recipient receives a \texttt{<EXAMPLE-URN>} of longer length, it \MUST{} be discarded 
as invalid.

Example-URNs that are assigned to a resource \MUST{} set \texttt{<YEAR>} 
to the year of assignment. \texttt{<DOMAIN>} \MUST{} be a fully 
qualified domain name. The domain name \MAY{} not contain a record 
in the DNS systems, but the domain \MUST{} be under control of the 
assigning organisation at the time of the assignment.
\end{example}

Consideration: The specification may use the augmented BFR format \cite{rfc5234}.

This section may be combined with section 2.2 or 2.3.

\subsubsection*{2.2. Reserved Use}

This section should outline which of valid URNs \MAY{} be assigned, and which URNs 
\MUSTNOT{} (yet) be assigned, but are reserved for future use.

\begin{example}
For current assignments, \texttt{<TYPE>} \MUST{} be either “\texttt{supercomputer}” or “\texttt{cluster}”:

\begin{quote}
  \texttt{TYPE = (\qq{}cluster\qq{} / \qq{}supercomputer\qq{})}
\end{quote}

The \texttt{<QUERY>} and \texttt{<FRAGMENT>} part \MUSTNOT{} be included in
assigned URNs, until its use is documented by a revision of this document.
\end{example}

Considerations: \cite{rfc2141} does not allow query or fragment identifiers,
but they may be allowed in the future. It is recommended to explicitly
note if query or fragment identifiers are allowed in the namespace.

This section may be combined with section 2.1 or 2.3.

\subsubsection*{2.3. Encoding}

This section should specify the encoding or decoding algorithms. 
This is particularly applicable in the case (a part of) a URN maps to a 
external naming system.

This section \MUST{} specify if percent-escaping as defined in \cite{rfc2141}
is allowed, and if so, what the binary code represents. This section \MUST{}
specify if the \texttt{<reserved>} characters as defined in \texttt{<URN
chars>} in \cite{rfc2141} are allowed (it is \textsc{recommended} that they are \textsc{not} allowed).

\begin{example}
\texttt{<DOMAIN>} \MAY{} be an internationalized domain name. All labels 
of the domain must be written in classic LDH (letter, digit, hyphen) format.
International labels must be converted to an A-label (starting with 
\texttt{xn-{}-})~\cite{rfc5890}.

Percentage-escaped strings \MUSTNOT{} be used in \texttt{<EXAMPLE-URN>}. 
It is \RECOMMENDED{} that for display purposes, protocols that exchange 
URNs include an attribute of the URN that is the human readable name 
which may include Unicode characters. The definition of such attribute 
is out of scope of this specification.
\end{example}

\begin{example}[ where a URN contains a reference to a external naming system]
\begin{itemize}
    \item The \texttt{<LOCODE>} part of the URN \MUST{} map to a valid 
    UN/LOCODE string~\cite{unlocode}, with the \texttt{\qq{} \qq{}} 
    (space) in the UN/LOCODE mapped to a \texttt{\qq{}-\qq{}} in the \texttt{<LOCODE>} 
    part of the URN. Since no \texttt{\qq{}-\qq{}} can occur in UN/LOCODE identifiers, 
    the inverse mapping is also one-to-one.
\end{itemize}
\end{example}
\begin{example}[ where a URN may contain percentage-escaped characters]
\begin{itemize}
    \item The \texttt{<NAME>} part of the URN \MAY{} contain percent-escaped 
    characters (\texttt{\qq{}\%\qq{} <hex> <hex>}) as described in \cite{rfc2141}. 
    The percentage-decoded \texttt{<NAME>} \MUST{} be a valid UTF-8 
    encoded byte sequence.
    \item Assigning organisations \SHOULD{} only assign \texttt{<NAME>}
    whose Unicode code points are NFKD-normalised according to the Unicode 5 
    specification)\cite{uax15} and are allowed by 
    the rules stipulated in \cite{rfc5892} (no upper case or control code points).
    \item Receiving software \SHOULD{} accept URNs that are valid UTF-8 encoded, 
    even if it contains disallowed code points. Receiving software \SHOULDNOT{} 
    accept URNs that are not valid UTF-8 encoded.
    \item Receiving software \SHOULDNOT{} display the percentage-decoded URN if it contains 
    disallowed code points. This prevents ambiguity if these URNs are copied and 
    pasted by a user.
\end{itemize}
\end{example}

This section may be combined with section 2.1 or 2.2.

\subsubsection*{2.4. Rules for Lexical Equivalence}

This section should list the algorithm for determining lexical equivalence 
between two identifiers in the underlying namespace.

\begin{example}
URNs are lexical equivalent if and only if they are byte-equivalent after case 
normalisation.
\end{example}

\begin{example}[ where a URN may contain percentage-escaped characters]

Interpretation or normalisation of percentage-escaped characters must not take place.
\end{example}

Considerations: \cite{rfc2141,rfc6453} specify that the
\texttt{\qq{}urn:ogf:\qq{}} part and the \texttt{<SNID>} are case insensitive and that
hex-encoding in a percentage-escaped character is case insensitive. The
lexical equivalence algorithm defined in this section \MUST{} comply with
these requirements. Typical normalisation functions to consider are case
normalisation, percentage-encoding using UTF-8, diacritical normalisation 
using canonical (de)composition (e.g. NFC and NFD in Unicode \cite{uax15}), 
and ligature normalisation using compatibility (de)composition (e.g. NFKC 
and NFKD in Unicode \cite{uax15}). If Unicode normalisation is to occur, 
the specification \SHOULD{} list a specific Unicode version (e.g. Unicode 
2.0 or Unicode 6.0) since this function may change between versions 
(e.g. there is no pre-composed character for Y-caron in Unicode version 6, 
but this may be created in a future version).

It is recommended to list a few examples.

\begin{example}
The following two Example-URNs are equivalent to each other:

  1- \texttt{urn:ogf:example:cluster:2009:example.net:42} \\
  2- \texttt{URN:OGF:Example:CLUSTER:2009:EXAMPLE.net:42}
\end{example}

\subsection*{3. Services}

\subsubsection*{3.1. Validation mechanism}

A URN namespace may provide mechanisms for “validating” a URN -- 
i.e., determining whether a given string is currently an assigned URN 
or a non-assigned URN.
For example, even if a telephone number-based URN namespace was
created, it is not clear that all possible telephone numbers would
immediately become “valid” URNs.

Where applicable, this section should describe if the operation of 
validation servers is an open process, or that it is subject to 
some authoritative delegation procedure.

Considerations: If such mechanism exists, this section may describe 
this mechanism, or may refer to a document that does describe it.
This section may contain requirement how a recipient of a malformed 
URN should act: e.g. accept the URN as-is, attempt to normalise it, or 
reject it.

\begin{example}
The Example-URN namespace has a Dynamic Delegation Discovery System 
(DDDS)~\cite{rfc3402}. The DDDS is a lookup service where applications 
can verify the validity of a given Example-URN.

The DDDS for the Example-URN namespace uses DNS as the lookup database.
For persistency reasons, the first lookup is done within the specially 
registered grid.example.net domain, which services as a persistent entry 
point, even if the domain names that are part of Example-URNs are no longer 
in use.

The Application Unique String is the full Example-URN, and the First Well 
Known Rule (as per \cite{rfc3402}) is:

\begin{quote}
\texttt{/urn:ogf:example:([a-z0-9\textbackslash{}-]+):([0-9]+):([a-z0-9\textbackslash{}-\textbackslash{}.]+):([0-9]+)/ \textbackslash{}2.\textbackslash{}3.\textbackslash{}1.grid.example.net/i}
\end{quote}

A lookup client \SHOULD{} apply the above First Well Know Rule to the full 
Example-URN, and do a DNS lookup for a NAPTR record for the resulting DNS 
record. The result is a rule set, which \MUST{} be applied to the full 
Example-URN as described in~\cite{rfc3402}.

A URN is considered valid if a record with a NAPTR record is found that 
terminates the lookup sequence (thus with the S, A, U or P flag set) 
within at most 8 iterations.
\end{example}

This section is optional. It may be combined with section 3.2.

\subsubsection*{3.2. Process for identifier resolution}

A URN namespace may provide mechanisms for “resolving” a URN -- 
i.e., determining attributes of the resources that is described, 
such as metadata or the location(s) where the resource can be retrieved.
A resolver may also be used to point to related URNs -- i.e. URNs that 
refer to the server resource, or URNs that refer to a more general or 
more specific part of the resource.

Where applicable, this section should describe if the operation of resolution 
servers is an open process, or that it is subject to some authoritative 
delegation procedure.

Considerations: If such mechanism exists, this section may describe this 
mechanism, or may refer to a document that does describe it.
This section may contain an estimate of the volatility of the resource 
attributes and a reasonable caching time for these attributes, or it could 
dictate that any resolution mechanism contains caching time-outs.

\begin{example}
The URN-Example namespace does currently not define a Resolution Discovery 
System (RDS), but clients \MAY{} translate an Example-URN to a identifier in 
the handle systems\cite{rfc3650} using a yet-to-be-defined mapping, and use 
the handle system to query information about the Example-URN.
\end{example}

This section is optional. It may be combined with section 3.1.

\subsection*{4. Namespace Considerations}

\subsubsection*{4.1. Scope}

This section should address the type of resource to be identified in 
the proposed namespace.

It is recommended that a namespace is limited in scope.  
For example, a namespace claiming to deal in “computers” must 
have a global scope and address all computers, which is unlikely.
On the other hand, a namespace claiming to deal with only top500 
supercomputers is more reasonable. It is expected that more than one 
namespace may serve the same “functional” purpose.

This section should also make it clear which version of the resource is 
identified. For example, the namespace could deal with a general
resource (e.g. “the weather in Seattle”), a particular manifestation or
version (e.g. “the weather in Seattle on May 1st, 2011”) or a dynamic
manifestation (e.g. “the current weather in Seattle”).

\begin{example}
The following resource types for bibliographic records exist:
\begin{description}
    \item [work] e.g. Jane Austin's “Gone with the Wind”, or the 
    \item [expression] e.g. the Swedish translation of the work
    \item [manifestation] e.g. a hard cover of the expression
    \item [item] e.g. an actual book on a shelf
\end{description}
The Example-URN namespace only deals with manifestations, but not with works, expressions or individual items.
\end{example}


% If desired, the URN
% syntax may include a part which specifies the version of the resource (e.g.
% \path{urn:ogf:example:cluster:2009-10:example.net:42} or \path{urn:ogf:example:cluster:2011-08:example.net:42})

\begin{example}
The Example-URN namespace only deals with top500 supercomputers and equally 
large scale grid clusters. The namespace is intended for supercomputers 
and clusters for scientific use, though there are 
no technical or organisational limitations for other use 
(e.g. clusters at large corporations or military).

Each Example-URN refers to the hardware of the supercomputer 
or cluster in its latest operational state. An assigning organisation 
\MAY{} assign a new identifier if a significant hardware update is made. 
(In which case, any \emph{status} attribute of the old identifier \SHOULD{} 
be set to \emph{decommissioned}.)

Clients that process Example-URNs \SHOULD{} consider attributes such 
as the size of the cluster or supercomputer to be dynamic, and \SHOULD{} 
regularly query the assigning organisation for changes in these properties.
\end{example}

This section may be combined with sections 4.2 and/or 4.3.

\subsubsection*{4.2. Identifier uniqueness considerations}

This section should address the requirement that URN identifiers be assigned 
uniquely -- they are assigned to at most one resource, and are not reassigned.

\begin{example}
An Example-URN always refers to the latest version (latest operational state) of 
a resource. Attributes of Example-URNs are considered dynamic. Example-URNs can 
not refer to earlier states of a cluster of supercomputer, unless the 
assigning organisation deliberately created a different URN for each version.

Domain names are volatile for the duration of persistent URNs. The \texttt{<YEAR>} 
part in the Example-URN makes sure that the assigning organisations never re-use the 
same identifier, even if the domain is transferred to a different assigning organisation.
This assumes that an assigning organisation \SHOULD{} retain control of a domain 
name for one year after the last identifier was assigned
\end{example}

Considerations: non-reassignment does not prevent all situations where a single 
URN refers to multiple resources. Another risk is that a single URN indadvertedly 
refers to multiple versions of the same resource, where each version has 
different properties. A common solution to this potential problem is to either 
limiting the scope of the URN namespace (e.g. a URN always refers to the latest 
version, and it is not possible to refer to an earlier version), or adding a 
version part to the URN syntax.

This section may be combined with sections 4.1 and/or 4.3.

\subsubsection*{4.3. Exposition of Structure}

This section should include any other considerations dealing with the
interpretation the of URNs in the namespace, in addition to the syntactic checks 
that are described before.

Considerations: If part of the structure is opaque in meaning (the meaning 
is not exposed), this should be noted. 

\begin{example}
\begin{itemize}
    \item The \texttt{<INDEX>} part of an Example-URN is opaque; 
      clients \MUSTNOT{} infer any attributes about the resource being 
      identified from this index.
    \item Clients \MAY{} infer the assigning organisation from the combination 
      of \texttt{<DOMAIN>} and \texttt{<YEAR>} (but \SHOULDNOT{} infer that 
      from the \texttt{<DOMAIN>} part alone).
\end{itemize}
\end{example}

This section may be combined with sections 4.1 and/or 4.2.

\subsection*{5. Community Considerations}

\subsubsection*{5.1. Process of identifier assignment}

This section should detail the mechanisms and/or authorities for
assigning URNs to resources.  It should make clear whether
assignment is completely open, or if limited, how to become an
assigner of identifiers, and/or get one assigned by existing
assignment authorities.

Answers could include, but are not limited to:
\begin{itemize}
    \item assignment is completely open, following a particular algorithm
    \item assignment is delegated to authorities recognised by a particular 
      organisation
    \item identifiers are assigned sequentially by some automate process
    \item assignment is handled on a per-identifier basis by community consensus 
      (e.g. by publication of a community reviewed Grid Forum Document.)
\end{itemize}

\begin{example}
Assignment of Example-URN is open to all interested parties. The requirement 
to become an Assigning Organisation is to be the administrative owner of a fully 
qualified domain name (FQDN). The administrative owner picks a domain or 
(sub)domain thereof (e.g. the administrative owner of \texttt{example.net} 
may choose \texttt{example.net} or a subdomain such as \texttt{sc.example.net}). 
The assigning organisation is free to choose any \texttt{<INDEX>}, as long as 
it is syntactically correct, and the resulting URN has never been assigned 
before.

If third parties should be able to verify the validity of Example-URN 
and retrieve basic attributes, the assigning organisation \MUST{} register the 
combination of the \texttt{<DOMAIN>} and \texttt{<YEAR>} with the Grid Lookup 
Organisation that provides the lookup service at grid.example.net.
\end{example}

\subsubsection*{5.2. Identifier immutability considerations}

A URN scheme \MUST{} be designed such that even in rare events, the identifiers 
never need to be changed. This section should include the considerations why 
this given syntax of the URN is most likely to remain stable.

The desire to change an URN may occur when the URN contains attributes which have 
changed. The easiest way to avoid this is to avoid the inclusion of volatile or 
dynamic attributes in the URN syntax.

\begin{example}
Example-URNs are assigned with the \texttt{<YEAR>} and \texttt{<DOMAIN>} as 
it exist \emph{at the time of assignment}. When the year changes or if the domain 
changes, \emph{the URN \MUSTNOT{} change}.

The \texttt{<INDEX>} has a very limited set of allowed characters (only digits), 
to prevent assigning organisations from encoding (potentially volatile) 
attributes in the URN.
\end{example}

Considerations: 
While the following events rarely occur, their occurrence may still be more 
frequent than the anticipated lifetime of the resource identifier. 
Thus, they should be considered in this section.
\begin{itemize}
    \item The prefix contains the name of assigning organisation, and the 
      assigning organisation merges, ceases to exists or changes name.
    \item The identifier contains certain attributes of the resource, and 
      these attributes change.
    \item The resource changes.
    \item Seemingly stable properties (such as a location) may change name 
      (e.g. countries may be split, merged or change name).
\end{itemize}

% For example, the identifier
% \path{urn:ogf:example:supercomputer:2011:colossus.bletchleypark.org.uk:1} 
% contains the domain “colossus.bletchleypark.org.uk”, as well as the year 2011. 
% The Bletchley Park trust could file for bankrupcy; a marketing decision was 
% made to change the name; the community of Bletchley could be annexed by the 
% city of London, etcetera. 

The section may contain a realistic estimate of the maximum lifetime of the resource%
\footnote{Among URN experts, it is uncommon to make estimates of the 
lifetime of \emph{persistent} identifiers. Theoretically, they should last 
indefinitely. It is the believe of the authors that a good estimate of the 
maximum lifetime of a resource identifier (e.g. 50 years, 2000 years, ...) 
helps reviewers evaluate the SNID registration.}%
, a maximum lifetime of the resource \emph{identifier}%
\footnote{the \textbf{maximum} lifetime of a resource \textbf{identifier} is 
usually significantly longer than the \textbf{average} lifetime of a resource 
identifier, and also significantly longer than the maximum lifetime of a 
resource itself.}%
, and thus the minimum time that the resource identifier \MUSTNOT{} change.
This section \SHOULD{} explain the organisational requirements how
this minimum time is ensured.

\subsection*{6. Examples}

This section may list a few example identifiers.

\begin{example}
The following four URNs are syntactically valid Example-URNs:
These examples are informative only. They may not actually exist:

  1- \texttt{urn:ogf:example:cluster:2009:example.net:42} \\
  2- \texttt{urn:ogf:example:cluster:2010:example.net:00784505476220484608} \\
  3- \texttt{urn:ogf:example:supercomputer:2011:colossus.bletchleypark.org.uk:1} \\
  4- \texttt{urn:ogf:example:supercomputer:2011:xn-{}-tcklt4eq6h7d1cbe.example.org:1}

\ifxetex
(Punycode \texttt{tcklt4eq6h7d1cbe} translates to スーパーコンピュータ, Japanese for \emph{supercomputer}.)
\else
(Punycode \texttt{tcklt4eq6h7d1cbe} translates to \begin{CJK}{UTF8}{min}スーパーコンピュータ\end{CJK}, Japanese for \emph{supercomputer}.)
\fi

The following four URNs are syntactically \textbf{invalid} Example-URNs:

  5- \texttt{\st{urn:ogf:example:supercomputer:2011:colossus.bletchleypark.org.uk:mark1}} \\
  6- \texttt{\st{urn:ogf:example:supercomputer:2011:colossus.bletchleypark.org.uk}} \\
  7- \texttt{\st{urn:ogf:example:supercomputer:1949:colossus.bletchleypark.org.uk:1}} \\
  8- \texttt{\st{urn:ogf:example:supercomputer:2000:\%e3\%82\%b9\%e3\%83\%bc\%e3\%83\%91\%e3\%83\%bc} \\
\hspace*{1cm}\st{\%e3\%82\%b3\%e3\%83\%b3\%e3\%83\%94\%e3\%83\%a5\%e3\%83\%bc\%e3\%82\%bf.example.org:1}}

URN 5 has invalid characters in the \texttt{<INDEX>}; URN 6 has no \texttt{<INDEX>}; URN 7 has an invalid \texttt{<YEAR>} (the domain name certainly did not exist in 1949); URN 8 has a \texttt{<LDH-LABEL>} in percentage encoded UTF-8, instead of the required Punycode.
\end{example}

\subsection*{7. Relevant Ancillary Documentation}

This section should list any GFDs, RFCs, standards, or other published
documentation that defines or explains all or part of the
namespace structure.

This section is optional and may be replaced with in-line bibliography references.

\subsection*{8. Changes from Previous Versions / Prior Usage}

This section \SHOULD{} include changes to previous versions of the document, 
or list prior (undocumented) use of the namespace.

This section is optional.

\hrulefill\hspace{1ex} \textsf{End of Template} \hspace{1ex}\hrulefill

\section{Security Considerations}

There are no additional security considerations other than those
normally associated with the use and resolution of URNs in general.

Implementors are recommended to check the OGF registry and documentation 
at \url{http://www.ogf.org/urn/} before assuming that a given identifier 
is valid or has a certain meaning.

\section{Glossary}

\begin{description}

  \item[Assigning Organisation] A namespace organisation that assigns unique 
    identifiers to resources.
  
  \item[Cool URI] A URL (!) that is persistent and is used as both a locator 
    and an identifier.

  \item[Fragment Identifier] The part of a URN that follows a pound sign 
    (\qq{}\texttt{\#}\qq{}).

  \item[Globally Unique Identifier] An identifier which identifies at most one 
    resource worldwide. A global identifier typically consists of a prefix and 
    a local identifier.

  \item[Identifier] A label (string or byte sequence) that identifies a resource.

  \item[Locally Unique Identifier] An identifier assigned by an assigning organisation, 
    that identifies at most than one resource.

  \item[Namespace] A set of all possible identifiers with the same namespace identifier.

  \item[Namespace Identifier] A unique identifier of a namespace organisation, 
    often used as a prefix.

  \item[Namespace Organisation] An organisation maintaining a namespace. 
    A namespace organisation may delegate part of their namespace to another 
    namespace organisation.

  \item[Prefix] A namespace identifier, which is prepended to a local identifier 
    to form a globally unique identifier.

  \item[Question Identifier] The part of a URN that follows a question mark 
    (\qq{}\texttt{?}\qq{}), but excluding the Fragment Identifier.

  \item[Subnamespace Identifier] A word that identifies a part of a a larger 
    namespace. The string \qq{}urn:ogf:\qq{} followed by a subnamespace identifier 
    forms the prefix of a namespace that is subject of the registration procedure 
    as described in this document.
  
  \item[Unique Identifier] An identifier that identifies at most one resource, 
    and is never re-assigned.

  \item[URI] A string that adheres to the URI syntax~\cite{rfc3986}, such as a URL or URN.

  \item[URL] The location where a resource can be found.

  \item[URN] A globally unique identifier that adheres to the URN syntax described 
    in \cite{rfc2141}.
\end{description}

\section{Contributors}

\textbf{Freek Dijkstra (Editor)} \\
SARA \\
Science Park 121 \\
1098 XG\ \  Amsterdam \\
The Netherlands \\
Email: Freek.Dijkstra@sara.nl

\textbf{Richard Hughes-Jones} \\
DANTE \\
City House \\
126-130 Hills Road \\
Cambridge\ \ CB2 1PQ \\
United Kingdom \\
Email: Richard.Hughes-Jones@dante.net

\textbf{Gregory B.\ Newby} \\
Arctic Region Supercomputing Center \\
P.O.\ Box 756020 \\
Fairbanks, Alaska\ \ 99775 \\
USA \\
Email: gbnewby@alaska.edu

\textbf{Joel Replogle (Corresponding Author)} \\
Open Grid Forum Office \\
P.O.\ Box 1738 \\
Muncie, Indiana\ \ 47308 \\
USA \\
Email: replogle@ogf.org

\section{Acknowledgments}

This document is based on the best practice of delegation in the URN hierarchy, 
as described in RFC 3406.

\phantomsection\addcontentsline{toc}{section}{Intellectual Property Statement}
\section*{Intellectual Property Statement}

The OGF takes no position regarding the validity or scope of any intellectual
property or other rights that might be claimed to pertain to the
implementation or use of the technology described in this document or the
extent to which any license under such rights might or might not be
available; neither does it represent that it has made any effort to identify
any such rights. Copies of claims of rights made available for publication
and any assurances of licenses to be made available, or the result of an
attempt made to obtain a general license or permission for the use of such
proprietary rights by implementers or users of this specification can be
obtained from the OGF Secretariat.

The OGF invites any interested party to bring to its attention any
copyrights, patents or patent applications, or other proprietary rights which
may cover technology that may be required to practice this recommendation.
Please address the information to the OGF Executive Director.

\phantomsection\addcontentsline{toc}{section}{Disclaimer}
\section*{Disclaimer}

This document and the information contained herein is provided on an ``As
Is'' basis and the OGF disclaims all warranties, express or implied,
including but not limited to any warranty that the use of the information
herein will not infringe any rights or any implied warranties of
merchantability or fitness for a particular purpose.

\phantomsection\addcontentsline{toc}{section}{Full Copyright Notice}
\section*{Full Copyright Notice}

Copyright \copyright \ Open Grid Forum (\copyrightyears). Some Rights Reserved.

This document and translations of it may be copied and furnished to others,
and derivative works that comment on or otherwise explain it or assist in its
implementation may be prepared, copied, published and distributed, in whole
or in part, without restriction of any kind, provided that the above
copyright notice and this paragraph are included on all such copies and
derivative works. However, this document itself may not be modified in any
way, such as by removing the copyright notice or references to the OGF or
other organizations, except as needed for the purpose of developing Grid
Recommendations in which case the procedures for copyrights defined in the
OGF Document process must be followed, or as required to translate it into
languages other than English.

The limited permissions granted above are perpetual and will not be revoked 
by the OGF or its successors or assignees.


\phantomsection\addcontentsline{toc}{section}{References}
\section*{References}
% 
% Define heading of bibliography to be empty, since we already have a heading above the text.
\renewcommand{\refname}{}

\phantomsection\addcontentsline{toc}{subsection}{Normative References}
\subsection*{Normative References}
\begin{thebibliography}{10}
\vspace*{-3em}

\bibitem[GFD.152]{gfd152}
Charlie Catlett, Cees de Laat, David Martin, Gregory B.\ Newby and Dane Skow.
\newblock {Open Grid Forum Document Process and Requirements}.
\newblock GFD.152 (Community Practice), 24 June 2009.
\newblock URL \url{http://www.ogf.org/documents/GFD.152.pdf}.

\bibitem[RFC 2119]{rfc2119}
Scott Bradner.
\newblock {Key words for use in RFCs to Indicate Requirement Levels}.
\newblock RFC 2119 (Best Current Practice), March 1997.
\newblock URL \url{http://tools.ietf.org/html/rfc2119}.

\bibitem[RFC 2141]{rfc2141}
Ryan Moats.
\newblock {URN Syntax}.
\newblock RFC 2141 (Standards Track), May 1997.
\newblock URL \url{http://tools.ietf.org/html/rfc214}.

\bibitem[RFC 3406]{rfc3406}
Leslie L.\ Daigle, Dirk-Willem van Gulik, Renato Iannella and Patrik Fältström.
\newblock {Uniform Resource Names (URN) Namespace Definition Mechanisms}.
\newblock RFC 3406 (Best Current Practice), October 2002.
\newblock URL \url{http://tools.ietf.org/html/rfc3406}.

\bibitem[RFC 3986]{rfc3986}
Tim Berners-Lee, Roy T. Fielding, and Larry Masinter.
\newblock {Uniform Resource Identifier (URI): Generic Syntax}
\newblock RFC 3986 (Standards Track), January 2005.
\newblock URL \url{http://tools.ietf.org/html/rfc3986}.

\bibitem[RFC 5234]{rfc5234}
Dave Crocker and Paul Overell.
\newblock {Augmented BNF for Syntax Specifications: ABNF}.
\newblock RFC 5234 (Standards Track), January 2008.
\newblock URL \url{http://tools.ietf.org/html/rfc5234}.

\bibitem[RFC 6453]{rfc6453}
Freek Dijkstra and Richard Hughes-Jones.
\newblock {A URN Namespace for the Open Grid Forum (OGF)}.
\newblock RFC 6453 (Informational), December 2011.
\newblock URL \url{http://tools.ietf.org/html/rfc6453}.

\end{thebibliography}

\phantomsection\addcontentsline{toc}{subsection}{Informative References}
\subsection*{Informative References}

\begin{thebibliography}{10}
\vspace*{-3em}

\bibitem[GFD.084]{gfd84}
Michel Drescher and Ali Anjomshoaa.
\newblock {Standardised Namespaces for XML infosets in OGF}.
\newblock GFD.084 (Community Practice), 31 October 2006.
\newblock URL \url{http://www.ogf.org/documents/GFD.84.pdf}.

\bibitem[LOCODE]{unlocode}
United Nations Economic Commission for Europe (UNECE).
\newblock {United Nations Code for Trade and Transport Locations}.
\newblock UN/LOCODE 2010-2, December 2010.
\newblock URL \url{http://www.unece.org/cefact/locode/}.

\bibitem[RFC2141bis]{rfc2141bis}
Alfred Hoenes (Editor).
\newblock {Uniform Resource Name (URN) Syntax}.
\newblock draft-ietf-urnbis-rfc2141bis-urn (Standards Track, if approved), October 2011.
\newblock URL \url{http://tools.ietf.org/html/draft-ietf-urnbis-rfc2141bis-urn}.

% \bibitem[RFC 3305]{rfc3305}
% Michael Mealling and Ray Denenberg.
% \newblock {Report from the Joint W3C/IETF URI Planning Interest Group: Uniform Resource Identifiers (URIs), URLs, and Uniform Resource Names (URNs): Clarifications and Recommendations}.
% \newblock RFC 3305 (Informational), November 2003.
% \newblock URL \url{http://tools.ietf.org/html/rfc3650}.

\bibitem[RFC 3402]{rfc3402}
Michael Mealling.
\newblock {Dynamic Delegation Discovery System (DDDS) - Part Two: The Algorithm}.
\newblock RFC 3402 (Standards Track), October 2002.
\newblock URL \url{http://tools.ietf.org/html/rfc3402}.

\bibitem[RFC 3650]{rfc3650}
Sam Sun, Larry Lannom and Brian Boesch.
\newblock {Handle System Overview}.
\newblock RFC 3650 (Informational), August 2002.
\newblock URL \url{http://tools.ietf.org/html/rfc3305}.

\bibitem[RFC 5890]{rfc5890}
John C. Klensin.
\newblock {Internationalized Domain Names for Applications (IDNA): Definitions and Document Framework}.
\newblock RFC 5890 (Standards Track), August 2010.
\newblock URL \url{http://tools.ietf.org/html/rfc5890}.

\bibitem[RFC 5892]{rfc5892}
Patrik Fältström (editor).
\newblock {The Unicode Code Points and Internationalized Domain Names for Applications (IDNA)}.
\newblock RFC 5892 (Standards Track), August 2010.
\newblock URL \url{http://tools.ietf.org/html/rfc5892}.

\bibitem[UAX\#15]{uax15}
Mark Davis and Ken Whistler.
\newblock {Unicode Normalization Forms}.
\newblock Unicode 6.0, Unicode Standard Annex \#15, September 2010.
\newblock URL \url{http://www.unicode.org/reports/tr15/}.

\bibitem[X.667]{x.667}
Information technology - Open Systems Interconnection.
\newblock {Generation and registration of Universally Unique Identifiers (UUIDs) 
and their use as ASN.1 object identifier components}.
\newblock ITU-T Recommendation X.667 / ISO/IEC 9834-8, August 2008.
\newblock URL \url{http://www.itu.int/rec/T-REC-X.667}.

\end{thebibliography}


\end{document}
