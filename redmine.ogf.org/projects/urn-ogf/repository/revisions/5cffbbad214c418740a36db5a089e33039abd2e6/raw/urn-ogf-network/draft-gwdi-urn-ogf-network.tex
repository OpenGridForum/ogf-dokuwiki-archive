%!TEX TS-program = pdflatex
\documentclass[12pt]{article}  % larger font to compensate for long lines with fullpage
\usepackage{ifpdf,ifxetex}
\ifxetex
% \usepackage{fontspec}
\else % inputenc is incompatible with xetex (which always takes UTF-8 as input)
\usepackage[utf8]{inputenc}
\fi
\usepackage{ifthen}
\usepackage{graphicx}
\usepackage[pdfborder={0 0 0}]{hyperref} % use hyperref without borders
\usepackage{url}
\usepackage{authblk}
\usepackage{framed}  % frames sections with page breaks
\usepackage{textcomp}
% Define some basics for your document:

\title{A URN Namespace for Network Resources}
\author{Freek Dijkstra \and Jeroen van der Ham}
\newcommand{\shortdoctitle}{URN:OGF:network specification}  % Title used in page header
% \date{} and \author{} are currently ignored
\newcommand{\authorsshort}{Freek Dijkstra, SARA \\ Jeroen van der Ham, University of Amsterdam}
\newcommand{\publicationdate}{September 2012}  % Date of first publication of the document
% \newcommand{\revisiondate}{December 2010}  % Optional: date of last revision of the document
\newcommand{\copyrightyears}{2011-2012}  % Years used in copyright notice
\newcommand{\docseries}{GWD-I}  % GWD-R, GWD-I or GWD-C (for working drafts), GFD-I, GFD-R, or GFD-C
\newcommand{\groupname}{NML-WG}  % Optional: name of the authoring working or research group
\newcommand{\groupurl}{\href{mailto:nml-wg@ogf.org}{nml-wg@ogf.org}}  % Optional: URL or email address of the authoring working or research group
%\newcommand{\documenturl}{}  % Optional: URL of this document


% Read pictures from img/ and current directory
\graphicspath{{img/}{./}}

%%% GWD/GFD header follows %%%
% Feel free to make changes, as long as your document follows the guidelines of GFD.152

\usepackage[numbers]{natbib} % Use [1] for references, 
\bibliographystyle{plainnat} % References show full author name(s) and document URL

\usepackage[sf,compact]{titlesec} % Use sans-serif for section headers

\usepackage[titles]{tocloft} % Format table of contents
% (tocloft is used, since titletoc is incompatible with xetex.)
\renewcommand{\cftsecfont}{\sffamily}
\renewcommand{\cftsubsecfont}{\sffamily}
\renewcommand{\cftsubsubsecfont}{\sffamily}
\renewcommand{\cftsecpagefont}{\sffamily}
\renewcommand{\cftsubsecpagefont}{\sffamily}
\renewcommand{\cftsubsubsecpagefont}{\sffamily}
\renewcommand{\cftsecleader}{\cftdotfill{\cftsubsecdotsep}} % dots for sections the same as for subsections
\setlength{\cftbeforesecskip}{0.5ex}


\usepackage{parskip} % Blank lines between paragraphs, no indentation.

% font style for headers and footers
\newcommand{\headerstyle}{\sffamily} % sans-serif

% Set page margins
\usepackage{fancyhdr}
\addtolength{\headheight}{15pt}
\renewcommand{\headrulewidth}{0pt}
% \setlength{\headrulewidth}{0pt}
\setlength{\headsep}{20pt}
\usepackage[headings]{fullpage}  % small margins

% Macro to check if (optional) values above are defined or not.
\newcommand{\ifnonempty}[2]{\ifthenelse{\isundefined{#1}}{}{\ifthenelse{\equal{#1}{}}{}{#2}}}

% Define page header and footers
\pagestyle{fancyplain}
\fancyhf{}
\lhead{\fancyplain{}{\headerstyle\docseries}}
% use \revisiondate if defined, otherwise \publicationdate for right header:
\rhead{\fancyplain{}{\headerstyle\ifthenelse{\isundefined{\revisiondate }}{\publicationdate}{\ifthenelse{\equal{\revisiondate}{}}{\publicationdate}{\revisiondate}}}}
\lfoot{\headerstyle\ifnonempty{\groupurl}{\groupurl}}
\rfoot{\headerstyle\thepage}
\thispagestyle{plain}

\ifpdf
\hypersetup{
    pdftitle = {A URN Namespace for Network Resources},
    pdfauthor = {Freek Dijkstra},
    pdfsubject = {urn:ogf:network specification},
    pdfkeywords = {URN network NML}
}

\fi

\begin{document}

% Title page header
{\noindent
\begin{minipage}[t]{1.5in}
\headerstyle
\docseries \\
\ifnonempty{\groupname}{\groupname \\}
\ifnonempty{\groupurl}{\groupurl \\}
\ifnonempty{\documenturl}{\documenturl \\}
\end{minipage}
\hfill
\raggedleft
\begin{minipage}[t]{4.5in}
\raggedleft
\headerstyle
\authorsshort \\
\vspace{1em}
\publicationdate \\
\ifnonempty{\revisiondate}{Revised \revisiondate \\}
\end{minipage}
}

\vspace{1em}
\begin{center}
\makeatletter
\Large\bf\textsf \@title
\makeatother
\end{center}

%%% End of header, insert content below this line %%%

\subsection*{Status of This Document}

Group working Draft (GWD), Informational.
% TODO: change to:
%Grid Final Draft (GFD), Informational.

% \section{Document Change History}
% 
% This template has been updated to confirm to GFD-C.152~\cite{gfd152}.

\subsection*{Copyright Notice}

Copyright \copyright \ Open Grid Forum (\copyrightyears).  Some Rights Reserved.  
Distribution is unlimited.

\phantomsection\addcontentsline{toc}{section}{Abstract}
\section*{Abstract}

This document specifies the procedure to create Uniform Resource Names (URNs) in the \texttt{urn:ogf:network} namespace. URNs in this namespace can be used to define logical network resources, such a devices, (logical) ports, (logical) links, and topologies.

\phantomsection\addcontentsline{toc}{section}{Contents}
\tableofcontents

\newpage

\section{introduction}%
\label{sec:introduction}

Uniform Resource Names (URNs) are persistent, globally unique identifiers~\cite{rfc2141}.

Topology exchange between network operators requires globally unique identifiers for network resources.  The \texttt{urn:ogf:network} namespace provides globally unique identifiers for naming network resources without central registration.

This document defines and registers the \texttt{urn:ogf:network} namespace in accordance with \cite{urn-ogf-procedure}.  It defines a procedure how organisation can create a globally unique organisation identifier, which can be used to prefix locally unique resource identifiers to form globally unique resource identifiers.

By re-using the domain name system with a date, no additional central registry or procedural overhead is required to create a globally unique organisation identifier.


\subsection*{Notational Conventions}
\label{sec:rfc2119}

\newcommand{\MUST}{\textsc{must}}
\newcommand{\MUSTNOT}{\textsc{must not}}
\newcommand{\REQUIRED}{\textsc{required}}
\newcommand{\SHALL}{\textsc{shall}}
\newcommand{\SHALLNOT}{\textsc{shall not}}
\newcommand{\SHOULD}{\textsc{should}}
\newcommand{\SHOULDNOT}{\textsc{should not}}
\newcommand{\RECOMMENDED}{\textsc{recommended}}
\newcommand{\NOTRECOMMENDED}{\textsc{not recommended}}
\newcommand{\MAY}{\textsc{may}}
\newcommand{\OPTIONAL}{\textsc{optional}}

The keywords “\MUST{}” “\MUSTNOT{}”, “\REQUIRED{}”, “\SHALL{}”, “\SHALLNOT{}”, “\SHOULD{}”, “\SHOULDNOT{}”, “\RECOMMENDED{}”, “\MAY{}”,  and “\OPTIONAL{}” are to be interpreted as described in~\cite{rfc2119}.
% except that the words do not appear in uppercase. 

\newcommand{\qq}{\symbol{34}} % 34 is the decimal LaTeX code for "
% \newcommand{\q}{\symbol{39}} % 39 is the decimal LaTeX code for '  % gives ’ instead of '. GRRR, LaTeX
\newcommand{\q}{\textquotesingle} % 39 is the decimal LaTeX code for '
\newcommand{\underscore}{\symbol{95}} % 95 is the decimal LaTeX code for _


\section{Registration}

\subsection{Namespace Identifier}

“\texttt{urn:ogf:network:}”

\subsection{Document Version}

Registration version number: 1 \\
Registration date: 2012-09-01  %% TODO: replace with date of publication of this GFD

\section{Syntax}

\subsection{Syntactic Structure}

A network resource URN (NURN) is defined by the following rules. These rules follow Augmented BFR~\cite{rfc5234} format.
\begin{quote}
  \texttt{NURN = \qq{}urn:ogf:network:\qq{} ORGID \qq{}:\qq{} OPAQUE-PART *1QUERY *1FRAGMENT} \\
  \texttt{ORGID = FQDN \qq{}:\qq{} DATE } ; ID of assigning organisation  \\
  \texttt{FQDN = 1*(ALPHA / DIGIT / \qq{}-\qq{} / \qq{}.\qq{})} ; Domain name \\
  \texttt{DATE = YEAR *1(MONTH *1DAY)} ; Date of creation of ORGID \\
  \texttt{YEAR = 4DIGIT} \\
  \texttt{MONTH = 2DIGIT} \\
  \texttt{DAY = 2DIGIT} \\
  \texttt{OPAQUE-PART = *(ALPHA / DIGIT / OTHER)} \\
  \texttt{OTHER = ALLOWED / EXTENSION} \\
  \texttt{ALLOWED = \qq{}+\qq{} / \qq{},\qq{} / \qq{}-\qq{} / \qq{}.\qq{} / \qq{}:\qq{} / \qq{};\qq{} / \qq{}=\qq{} / \qq{}\underscore\qq{}} \\
  \texttt{EXTENSION = \qq{}!\qq{} / \qq{}\$\qq{} / \qq{}(\qq{} / \qq{})\qq{} / \qq{}*\qq{} / \qq{}@\qq{} / \qq{}\textasciitilde\qq{} / \qq{}\&\qq{}} \\
  \texttt{QUERY = \qq{}?\qq{} *QFCHAR} \\
  \texttt{FRAGMENT = \qq{}\#\qq{} *QFCHAR} \\
  \texttt{QFCHAR = ALPHA / DIGIT / OTHER}
\end{quote}

\texttt{ALPHA}, \texttt{DIGIT} and \texttt{HEXDIG} are defined by \cite{rfc5234}, \texttt{OTHER} is almost equal to \texttt{<other>} as defined by~\cite{rfc2141}, but lacks the single quote (\texttt{\q{}}) character and includes the ampersand (\texttt{\&}) and tilde characters (\texttt{\textasciitilde}). \texttt{QFCHAR} is a subset of \texttt{pchar} as defined by~\cite{rfc3986}, only lacking the single quote character (\texttt{\q{}}) and percentage encoding (\texttt{\qq\%\qq{} HEXDIG HEXDIG}).

\texttt{ALLOWED} characters \MAY{} be used for the assignment of network resource URNs. 
\texttt{EXTENSION} characters \SHOULDNOT{} be used to assign network resource URNs. 
To allow for future extensions, parsers \SHOULD{} accept network resource URNs with \texttt{EXTENSION} characters.

The \texttt{QUERY} and \texttt{FRAGMENT} parts \MUSTNOT{} be present in any \textbf{assigned} URN.  This specification reserves their use for future standardization.

A network resource URN \MUSTNOT{} contain percentage-encoded characters (\texttt{\qq\%\qq{} HEXDIG HEXDIG}). It should also be noted that the following characters (which are either allowed by the URI or URN specification) \MUSTNOT{} be used in the \texttt{OPAQUE-PART} of a network resource URN: \texttt{\qq\%\qq}, \texttt{\qq/\qq}, \texttt{\qq{}?\qq}, \texttt{\qq\#\qq}, and \texttt{\qq{}\q{}\qq{}}. 

%\texttt{DP} is a prefix (\texttt{\qq{}domain=\qq{}}) of the \texttt{DOMAIN} 
%for backward compatibility with a previous syntax (see section~\ref{sec:perfsonar}). 
%It is deprecated and \SHOULDNOT{} be used for the assignment of new network resource URNs.

\texttt{DOMAIN} is a fully qualified domain name (FQDN) of the URN assigning organisation 
in LDR format~\cite{rfc5890}. 
Valid examples are \texttt{example.net} and \texttt{example.xn--jxalpdlp}. 
%\texttt{\underscore{}http.\underscore{}tcp.example.net} is an invalid example.
\texttt{DATE} is a date (either year, year+month or year+month+day).
The combination of \texttt{DOMAIN} and \texttt{DATE} forms the \emph{organisation identifier}, \texttt{ORGID}.

The full length of a \texttt{NURN} \MUSTNOT{} exceed 255 characters.

\texttt{OPAQUE-PART} is opaque, and \MUSTNOT{} be parsed or interpreted by any 
organisation except for the organisation that assigned the URN.

\subsection{Encoding}

A network resource URI uses a subset of 7-bit ASCII characters. 
No percentage-encoded characters are allowed.

\subsection{Rules for Lexical Equivalence}

Network resource URNs are lexical equivalent if and only if they are byte-equivalent after case normalisation.

Consider the following URNs:

  1- \texttt{urn:ogf:network:example.net:2012:path-glif-0418} \\
  2- \texttt{UrN:oGf:NeTwOrK:eXaMpLe.NeT:2012:pAtH-gLiF-0418} \\
  3- \texttt{URN:OGF:NETWORK:EXAMPLE.NET:2012:PATH-GLIF-0418} \\

URNs 1, 2, and 3 are lexically equivalent to each other. 

\subsection{Assignment}

The ORGID part must belong to the assignment organisation, as described in 
section~\ref{sec:orgid-assignment}.

Assigned network resource URNs \MUSTNOT{} contain a fragment or query part.

The characters defined in \texttt{EXTENSION} \SHOULDNOT{} be used in assignment 
of network resources URNs, and are reserved for future use. 
Only characters in \texttt{ALPHA / DIGIT / ALLOWED} \SHOULD{} be used in the \texttt{OPAQUE-PART}.

The length of the \texttt{URN} \MUSTNOT{} exceed 255 bytes.

\subsection{Validation}

A network resource URN that does not follow the specified syntax \SHOULD{} be rejected.

No specific validation service or resolution service is defined in this document.

A recipient should either use the credibility of the sender or some other mechanism to 
judge the correctness of a given URN.

\subsection{URN Rewriting}

A recipient \MUSTNOT{} rewrite the URN if the rewriting results in a URN which is not 
lexically equivalent to the received URN.
In particular, percentage-decoding of the URN as described in section 6.2.2.2. 
of~\cite{rfc3986} \MUSTNOT{} take place.

If two lexical equivalent URNs with different capitalisation 
have been received, the recipient \MAY{} pick one of the two 
capitalisations, and use that in all communications, effectively rewriting 
the URNs.

With the above exception, URNs \SHOULD{} retain the same capitalisation in a message exchange.

\section{Namespace Considerations}

\subsection{Scope}

The \texttt{urn:ogf:network} namespace is created to allow network 
operators to uniquely define resources in their network and facilitate 
unambiguous exchange of topology data with other network operators.

The only requirement for naming network resources is administrative 
ownership of a domain name during at most one year (see section~\ref{sec:persistence}). 
No other central registration is required.

The intended use of the \texttt{urn:ogf:network} namespace is to describe 
logical network resources roughly on OSI layers 1 and OSI layer 2. 
“Logical network resources” intends to mean elements in a functional topology 
description, rather than physical resources. It is expected that a peering 
network is only interested in the functional description of the network, 
not of its (physical) implementation. Nevertheless, this document does 
not forbid the description of other resources, such a physical network 
resources for inventory management.

\subsection{Resource Type Described}

The exact type of resource described by a URN can not and \MUSTNOT{} be 
determined from the syntax of the URN. This information \MUST{} be provided 
by the context or through other means by the data exchange protocol.

Network resources URNs \SHOULD{} identify \emph{manifestations} of a 
network resources --- they should refer to a functional component in a 
network that remains in place for a prolonged period of time. 
New version of the resource \SHOULDNOT{} receive a new identifier.
The change of attributes over time \SHOULD{} be dealt with by a protocol, 
not by a change of the URN.

\subsection{Identifier uniqueness considerations}

URN identifiers \MUST{} be assigned uniquely -- they are assigned to 
at most one resource, and \MUSTNOT{} be re-assigned.

URN assigning organisations \MUST{} follow these requirements before 
assigning URNs to network resources.

A single network resource \MAY{} be identified by multiple URNs. 

\section{Community Considerations}

\subsection{Process of Organisation Identifier Assignment}
\label{sec:orgid-assignment}

An organisation that wishes to become an assigning organisation, must pick a 
globally unique organisation identifier.

An organisation identifier consists of two components, a fully qualified domain 
name and a date, which must both be chosen by the assigning organisation.

The assigning organisation \MUST{} be the administrative contact of the chosen 
domain~cite{rfc5890} for at least the duration of the date.

It is \RECOMMENDED{} that the date is a year. Organisation that expect their 
DNS registration to be more volatile \SHOULD{} pick a more fine grained date 
specification (year+month or year+month+day).

There is no need for the assigning organisation to register themselves at the 
Open Grid Forum (the Namespace Organisation for the \texttt{urn:ogf:network} namespace).

An organisation \MAY{} use multiple organisation identifiers. 
For example, an organisation may pick a new organisation identifier in order 
to create a new syntax for their \texttt{OPAQUE-PART} syntax.

\subsection{Process of Network Resource Identifier Assignment}

An assigning organisation assigns \texttt{OPAQUE-PART}s to 
it network resources. The following requirements apply to the 
\texttt{OPAQUE-PART}:
\begin{itemize}
    \item The \texttt{OPAQUE-PART} \MUST{} uniquely define at most one network resource;
    \item The \texttt{OPAQUE-PART} \MUSTNOT{} be re-assigned;
    \item The \texttt{OPAQUE-PART} \SHOULDNOT{} specify any properties of the network resource;
    \item The \texttt{OPAQUE-PART} \MAY{} contain some structure according to some policy internal to the assigning organisation.
    \item The \texttt{OPAQUE-PART} \MUST{} have a valid syntax (use only allowed characters, does not exceed maximum length).
\end{itemize}

The reason that the \texttt{OPAQUE-PART} \SHOULDNOT{} contain any properties 
is because a URN \MUST{} be persistent: it \MUSTNOT{} change, even after the 
properties of the described resource change.
Naming these properties in the URN gives a false sense of meaning to the URN. 
Peer may indadvertedly assume the identifier describes certain 
properties, and act upon that, even if the properties have long changed.

Good examples of URNs:
\begin{quote}
\texttt{urn:ogf:network:example.net:2012:9ad7ef-mcasip-139284} \\
\texttt{urn:ogf:network:example.net:2012:link:2013:175} \\
\texttt{urn:ogf:network:example.net:2012:port:eth:98a-3470}
\end{quote}

Not so good examples of URNs:
\begin{quote}
\texttt{urn:ogf:network:example.net:2012:link:24x7-protected:925-175} \\
\texttt{urn:ogf:network:example.net:2012:sw3.rtr.example.net:port3-1:vlan118} \\
\texttt{link:eth:US\underscore{}CHI-NL\underscore{}AMS-3937}
\end{quote}

A useful syntax for \texttt{OPAQUE-PART} is 
\texttt{<type>:<year of creation>:<sequence number>}, 
e.g. \texttt{port:2013:129}.

While \texttt{port:2013:129} contains attributes (type and year of 
creation), these may be acceptable as they will never change. 
\texttt{link:24x7-protected:925-175} contains attributes about the type of link, 
which may change in the future, and is therefore not a good URN. 
\texttt{link:eth:US\underscore{}CHI-NL\underscore{}AMS-3937} is also not a good URN. 
It contains the end points of a path, which are unlikely to change. 
However, if the path is actually an Ethernet LAN, it is possible to add 
another end-point, changing these properties. The network domains along 
the path may use this identifier for monitoring and do not accept a 
change in identifier. For that reason, it is best never to add attributes 
to a URN identifier.

\subsection{Identifier persistence considerations}
\label{sec:persistence}

\cite{rfc3406} requires that URNs \MUSTNOT{} be re-assigned. Ever.
In practice, it is impossible to control what identifiers will be assigned 
in a few decades from now.

The requirement of the date in the organisation is sufficient guarantee.
If a domain name is transferred, or an organisation decides to start over 
with the assignment of local identifiers, it is easy enough to create a 
new organisational identifier.

Any organisation that wishes to assign names in the \texttt{urn:ogf:network} 
namespace must to so after due diligence, and make sure that no 
re-assignment occurs within the namespace(s) of the organisation and 
that the assigned name does not contain attributes which can change 
during the lifetime of the resource.

\section{Examples}

Syntactically valid network resource URNs which \MAY{} be assigned include:
\begin{quote}
\texttt{urn:ogf:network:example.net:2012:9ad7ef-mcasip-139284} \\
\texttt{urn:ogf:network:example.net:2012:link:925-175} \\
\texttt{urn:ogf:network:example.net:2012:port:2011-07-129} \\
\texttt{urn:ogf:network:example.net:20120916:4A6173706572} \\
\texttt{urn:ogf:network:example.net:2012:} \\
\texttt{urn:ogf:network:example.net:2012:l=214.56:x=a5y}
\end{quote}

The following URNs contain characters in the extension range. While they \SHOULDNOT{} be assigned to network resources, a recipient \SHOULD{} accept these examples:
\begin{quote}
\texttt{urn:ogf:network:example.net:20120916:4A6173706572(AMS-GEN)} \\
\texttt{urn:ogf:network:example.net:20120916:l=*:x=a5y} \\
\end{quote}
 
The following example is a syntactically valid URN, which contains a query 
part and hence \MUSTNOT{} be assigned to a network resource, but \MAY{} 
be used to query for a network resource, provided that subsequent standards 
define the syntax of the query part:
\begin{quote}
\texttt{urn:ogf:network:example.net:2012:portgroup:ams3?vlan=18}
\end{quote}

The following URNs is \emph{invalid}, and \SHOULD{} be rejected by a recipient, 
because a slash is not allowed in a URN (this is a limitation of all URNs, 
not just this specification):
\begin{quote}
\texttt{urn:ogf:network:example.net:2012:port:sw3:eth4/1:vlan18}
\end{quote}





\section{Security Considerations}

While this specification goes to great length to avoid accidental naming collisions,
malicious software can easily craft a NURN to collide with an existing NURN. 
Recipients of a NURN \MUST{} take such risks in consideration.

Recipients of a NURN \MUSTNOT{} assume that a NURN was crafted by the domain
specified in the \texttt{DOMAIN} part of the NURN, without a proper validation 
check.

The allowed syntax is so limited that it is not expected that similar-looking 
malicious NURNs will be an issue. Users and applications should be able to detect the 
differences between \texttt{urn:ogf:network:example.c0m:4638I27} 
and \texttt{urn:ogf:network:example.com:4638127}.

Software that takes input from a user \MUST{} ensure that the NURN is 
syntactically correct before transmitting it. For example, it \SHOULD{} remove any 
trailing spaces from the user input.

Information in the \texttt{OPAQUE-PART} \MUSTNOT{} be interpreted to have 
any meaning whatsoever. While the originating domain may have included 
meaningful attributes in the NURN, these attributes may be out-of-date.

\section{Prior Usage}

URN identifiers in the \texttt{urn:ogf:network} namespace have been in use in 
three communities, GLIF, PerfSONAR, and AutoGOLE, with mutually conflicting syntaxes.

\subsection{GLIF Community} %
\label{sec:glif}

The Global Lambda Integrated Facility (GLIF) is a community of research and 
education networks. Operators in this community agreed to use unique identifier 
for \emph{lightpaths}, dedicated inter-domain circuits for researchers.

These identifiers take the form:
\begin{quote}
  \texttt{GLOBAL-ID = \qq{}urn:ogf:network:\qq{} DOMAIN \qq{}:\qq{} LOCAL-PART } \\
  \texttt{DOMAIN = 1*(ALPHA / DIGIT / \qq{}-\qq{} / \qq{}.\qq{})} ; Domain name \\
  \texttt{LOCAL-PART = 1*(ALPHA / DIGIT / \qq{}-\qq{} / \qq{}.\qq{})}
\end{quote}

For example:
\begin{quote}
\texttt{urn:ogf:network:canarie.ca:kisti-uninett-glif-001} \\
\texttt{urn:ogf:network:es.net:4005} \\
\texttt{urn:ogf:network:dcn.internet2.edu:6811}
\end{quote}

The syntax is described in \cite{glif-identifiers}.

Identifiers described by the GLIF community generally do not contain a date, although it is possible to construct a URN which is both a valid \texttt{NURN} and \texttt{GLOBAL-ID}.

\subsection{PerfSONAR Community} %
\label{sec:perfsonar}

PerfSONAR is a distributed system for network performance monitoring on 
paths crossing several networks. Much of the perfSONAR protocols are 
standardised by the OGF in the Network Measurement (NM) and Network 
Measurement and Control (NMC) working groups. 
URNs in the \texttt{urn:ogf:network} namespace are used for topology description.

These identifiers take the form:
\begin{quote}
  \texttt{PS-URN = \qq{}urn:ogf:network\qq{} 1DOMAIN-PART *1NODE-PART *1PORT-PART } \\
  \hspace*{2cm}\texttt{*1LINK-PART *1PATH-PART *1SERVICE-PART *1WILDCARD } \\
  \texttt{DOMAIN-PART  = \qq{}:domain=\qq{} 1*DOMAIN} \\
  \texttt{NODE-PART    = \qq{}:node=\qq{} 1*PART-CHAR} \\
  \texttt{PORT-PART    = \qq{}:port=\qq{} 1*PART-CHAR} \\
  \texttt{LINK-PART    = \qq{}:link=\qq{} 1*PART-CHAR} \\
  \texttt{PATH-PART    = \qq{}:path=\qq{} 1*PART-CHAR} \\
  \texttt{SERVICE-PART = \qq{}:service=\qq{} 1*PART-CHAR} \\
  \texttt{DOMAIN = 1*(ALPHA / DIGIT / \qq{}-\qq{} / \qq{}.\qq{})} ; Domain name \\
  \texttt{PART-CHAR = (ALPHA / DIGIT / \qq{}-\qq{} / \qq{}.\qq{} / \qq{}/\qq{} / \qq{}\underscore\qq{})} \\
  \texttt{WILDCARD = \qq{}:*\qq{}} ; Used for queries.
\end{quote}

For example:

\begin{quote}
\texttt{urn:ogf:network:domain=example.net} \\
\texttt{urn:ogf:network:domain=example.net:node=packrat} \\
\texttt{urn:ogf:network:domain=example.net:link=WASH\underscore{}to\underscore{}ATLA} \\
\texttt{urn:ogf:network:domain=example.net:node=packrat:port=eth0} \\
\texttt{urn:ogf:network:domain=example.net:port=Interface\underscore{}To\underscore{}Geant} \\
\texttt{urn:ogf:network:domain=example.net:node=packrat:service=Optical\underscore{}Converter} \\
\texttt{urn:ogf:network:domain=example.net:node=packrat:port=eth0:link=WASH\underscore{}to\underscore{}ATLA} \\
\texttt{urn:ogf:network:domain=example.net:node=AMS:port=3/1:link=AMS-GEN} \\
\texttt{urn:ogf:network:domain=example.net:path=IN2P3\underscore{}Circuit} \\
\texttt{urn:ogf:network:domain=example.net:node=packrat:*}
\end{quote}

The syntax is described in \cite{ps-urn}.

Identifiers described by the perfSONAR topology service are \textbf{not} valid network resource URNs.
Note that the PS-URN syntax allows a slash in a URN, even though this is not allowed by \cite{rfc2141}.

The meaning of the perfSONAR URNs is fundamentally different from network resource URNs:
whereas perfSONAR URNs should specifically be parsed to find properties of the resource, 
this is not allowed for network resource URNs.

This document does not define a specific migration strategy for perfSONAR URNs.

\subsection{AutoGOLE Community}

AutoGOLE is a proof-of-concept architecture where over ten organisations provide a persistent testbed to show their ability to perform automatic network provisioning across network domains. The resources used in that demo follow the following syntax.

\begin{quote}
  \texttt{AUTOGOLE-URN = \qq{}urn:ogf:network:\qq{} TYPE \qq{}:\qq{} NETWORK *1LOCAL-PART} \\
  \texttt{TYPE = (\qq{}stp\qq{} / \qq{}nsa\qq{} / \qq{}nsnetwork\qq{})} \\
  \texttt{NETWORK = 1*(ALPHA / DIGIT / \qq{}-\qq{} / \qq{}.\qq{})} ; Human readable string \\
  \texttt{LOCAL-PART = \qq{}:\qq{} 1*(ALPHA / DIGIT / \qq{}-\qq{} / \qq{}.\qq{})}
\end{quote}

No formal publication has been made to describe this syntax.

AutoGOLE Identifiers are \textbf{not} valid network resource URNs. A drawback 
of the AutoGOLE identifiers is that it uses a custom name to identify networks, 
and subsequently the organisational identifier of the assigning organisations. 
Deploying this syntax on a large scale would require the set up of a namespace registry.

\subsection{Backwards Compatibility}

Applications that wish to be backward compatible with the GLIF-based, PerfSONAR-based and AutoGOLE-based URNS, are recommended to accept:

\begin{quote}
  \texttt{BC-NURN = \qq{}urn:ogf:network:\qq{} 1*(ALPHA / DIGIT / OTHER / \qq{}/\qq{})}
\end{quote}

Applications that decide to be liberal in the URN that they accept \MUST{} anticipate that other clients may do a more thorough syntax check and reject these URNs. In particular, the slash is formally not allowed in URNs.

Applications that merely accept URNs according to the \texttt{BC-NURN} syntax can still be compatible with this specification. 
However, as soon as a possibility exists that the application sends out URNs that do not comply to \texttt{NURN} syntax, 
then the application is no longer compatible with the specification described in this document.

\section{Contributors}

\textbf{Freek Dijkstra} \\
SARA \\
Science Park 121 \\
1098 XG\ \  Amsterdam \\
The Netherlands \\
Email: Freek.Dijkstra@sara.nl

\textbf{Jeroen van der Ham} \\
Faculty of Science, Informatics Institute, University of Amsterdam \\
Science Park 904, 1098 XH  Amsterdam  \\
The Netherlands \\
Email: vdham@uva.nl \\
\\

\subsection*{Acknowledgments}

The authors would like to thank the following people (in arbitrary order).

Jason Zurawski for also initiating this work in the network markup language (NML) working group.

Jens Jensen, Richard Hughes-Jones, Greg Newby, Joel Replogle, and Alan Sill for their help in establishing the \texttt{urn:ogf} namespace.

Aaron Brown and others at Internet2 for defining the \texttt{urn:ogf:network} namespace specification in the perfSONAR community and the network measurement (NM) working group.

Lars Fischer, Tom Lehman, Ronald van der Pol, and Thomas Tam for defining the \texttt{urn:ogf:network} namespace specification in the GLIF community.

Björn Höhrmann and Kadir Karaca Koçer of the urn and urn-nid mailing lists at the IETF for useful advice on long term requirements for URNs.

\phantomsection\addcontentsline{toc}{section}{Intellectual Property Statement}
\section*{Intellectual Property Statement}

The OGF takes no position regarding the validity or scope of any intellectual
property or other rights that might be claimed to pertain to the
implementation or use of the technology described in this document or the
extent to which any license under such rights might or might not be
available; neither does it represent that it has made any effort to identify
any such rights. Copies of claims of rights made available for publication
and any assurances of licenses to be made available, or the result of an
attempt made to obtain a general license or permission for the use of such
proprietary rights by implementers or users of this specification can be
obtained from the OGF Secretariat.

The OGF invites any interested party to bring to its attention any
copyrights, patents or patent applications, or other proprietary rights which
may cover technology that may be required to practice this recommendation.
Please address the information to the OGF Executive Director.

\phantomsection\addcontentsline{toc}{section}{Disclaimer}
\section*{Disclaimer}

This document and the information contained herein is provided on an ``As
Is'' basis and the OGF disclaims all warranties, express or implied,
including but not limited to any warranty that the use of the information
herein will not infringe any rights or any implied warranties of
merchantability or fitness for a particular purpose.

\phantomsection\addcontentsline{toc}{section}{Full Copyright Notice}
\section*{Full Copyright Notice}

Copyright \copyright \ Open Grid Forum (\copyrightyears). Some Rights Reserved.

This document and translations of it may be copied and furnished to others,
and derivative works that comment on or otherwise explain it or assist in its
implementation may be prepared, copied, published and distributed, in whole
or in part, without restriction of any kind, provided that the above
copyright notice and this paragraph are included on all such copies and
derivative works. However, this document itself may not be modified in any
way, such as by removing the copyright notice or references to the OGF or
other organizations, except as needed for the purpose of developing Grid
Recommendations in which case the procedures for copyrights defined in the
OGF Document process must be followed, or as required to translate it into
languages other than English.

The limited permissions granted above are perpetual and will not be revoked by the OGF or its successors or assignees.


\phantomsection\addcontentsline{toc}{section}{References}
% \section{References}
% 
% % Define heading of bibliography to be empty, since we already have a heading above the text.
% \renewcommand{\refname}{}
% \vspace*{-3em}
\begin{thebibliography}{2}
% \bibitem[draft-dijkstra-urn-ogf]{urn-ogf-delegation}
% Freek Dijkstra and Richard Highes-Jones.
% \newblock {A URN Namespace for the Open Grid Forum (OGF)}.
% \newblock draft-dijkstra-urn-ogf (Informational), April 2011.
% \newblock URL \url{http://tools.ietf.org/html/draft-dijkstra-urn-ogf}.

% \bibitem[GFD-C.152]{gfd152}
% Charlie Catlett, Cees de Laat, David Martin, Gregory B.\ Newby and Dane Skow.
% \newblock {Open Grid Forum Document Process and Requirements}.
% \newblock GFD-C.152 (Community Practice), 24 June 2009.
% \newblock URL \url{http://www.ogf.org/documents/GFD.152.pdf}.

\bibitem[GFD.191]{urn-ogf-procedure}
Freek Dijkstra, Richard Hughes-Jones, Gregory B.\ Newby and Joel Replogle.
\newblock {Procedure for Registration of Subnamespace Identifiers in the URN:OGF Hierarchy}.
\newblock December 2011.
\newblock URL \url{http://www.ogf.org/documents/GFD.191.pdf}.

\bibitem[GLIF-ID]{glif-identifiers}
Lars Fischer, Tom Lehman, Ronald van der Pol and Thomas Tam.
\newblock {Global Lightpath Identifiers Naming Scheme}.
\newblock Version 2.2. April 2009.
\newblock URL \url{http://www.glif.is/working-groups/tech/global-identifiers-v2.2.pdf}.

% \bibitem[NFKD]{nfkd}
% Mark Davis and Ken Whistler.
% \newblock {Unicode Standard Annex \#15; Unicode Normalization Forms}.
% \newblock Version 6.0.0. September 2010.
% \newblock URL \url{http://unicode.org/reports/tr15/}.

\bibitem[perfSONAR-URN]{ps-urn}
Aaron Brown (editor).
\newblock {A short description of the URN scheme}.
\newblock Retrieved February 2010.
\newblock URL \url{http://code.google.com/p/perfsonar-ps/wiki/URNs}.

\bibitem[RFC 2119]{rfc2119}
Scott Bradner.
\newblock {Key words for use in RFCs to Indicate Requirement Levels}.
\newblock RFC 2119 (Best Current Practice), March 1997.
\newblock URL \url{http://tools.ietf.org/html/rfc2119}.

\bibitem[RFC 2141]{rfc2141}
Ryan Moats.
\newblock {URN Syntax}.
\newblock RFC 2141 (Standards Track), May 1997.
\newblock URL \url{http://tools.ietf.org/html/rfc2141}.

% \bibitem[RFC2141bis]{rfc2141bis}
% Alfred Hoenes (Editor).
% \newblock {Uniform Resource Name (URN) Syntax}.
% \newblock draft-ah-rfc2141bis-urn (Standards Track, if aproved), May 2010.
% \newblock URL \url{http://tools.ietf.org/html/draft-ah-rfc2141bis-urn}.

% \bibitem[RFC 3305]{rfc3305}
% Michael Mealling and Ray Denenberg.
% \newblock {Report from the Joint W3C/IETF URI Planning Interest Group: Uniform Resource Identifiers (URIs), URLs, and Uniform Resource Names (URNs): Clarifications and Recommendations}
% \newblock RFC 3305 (Informational), November 2003.
% \newblock URL \url{http://tools.ietf.org/html/rfc3650}.

% \bibitem[RFC 3650]{rfc3650}
% Sam Sun, Larry Lannom and Brian Boesch.
% \newblock {Handle System Overview}
% \newblock RFC 3650 (Informational), August 2002.
% \newblock URL \url{http://tools.ietf.org/html/rfc3305}.

\bibitem[RFC 3406]{rfc3406}
Leslie L.\ Daigle, Dirk-Willem van Gulik, Renato Iannella and Patrik Fältström.
\newblock {Uniform Resource Names (URN) Namespace Definition Mechanisms}.
\newblock RFC 3406 (Best Current Practice), October 2002.
\newblock URL \url{http://tools.ietf.org/html/rfc3406}.

\bibitem[RFC 3986]{rfc3986}
Tim Berners-Lee, Roy T.\ Fielding, and Larry Masinter.
\newblock {Uniform Resource Identifier (URI): Generic Syntax}.
\newblock RFC 3986 (Standards Track), January 2005.
\newblock URL \url{http://tools.ietf.org/html/rfc3986}.

% \bibitem[RFC 3912]{rfc3912}
% Leslie Daigle.
% \newblock {WHOIS Protocol Specification}.
% \newblock RFC 3912 (Standards Track), September 2004.
% \newblock URL \url{http://tools.ietf.org/html/rfc3912}.

% \bibitem[RFC 3987]{rfc3987}
% Martin Dürst and Michel Suignard.
% \newblock {Internationalized Resource Identifiers (IRIs)}.
% \newblock RFC 3987 (Standards Track), January 2005.
% \newblock URL \url{http://tools.ietf.org/html/rfc3987}.

\bibitem[RFC 5234]{rfc5234}
Dave Crocker and Paul Overell.
\newblock {Augmented BNF for Syntax Specifications: ABNF}.
\newblock RFC 5234 (Standards Track), January 2008.
\newblock URL \url{http://tools.ietf.org/html/rfc5234}.

\bibitem[RFC 5890]{rfc5890}
John C.\ Klensin.
\newblock {Internationalized Domain Names in Applications (IDNA): Definitions and Document Framework}.
\newblock RFC 5890 (Standards Track), August 2010.
\newblock URL \url{http://tools.ietf.org/html/rfc5890}.

% \bibitem[RFC 5891]{rfc5891}
% John C.\ Klensin.
% \newblock {Internationalized Domain Names in Applications (IDNA): Protocol}.
% \newblock RFC 5891 (Standards Track), August 2010.
% \newblock URL \url{http://tools.ietf.org/html/rfc5891}.

% \bibitem[RFC 5892]{rfc5892}
% Patrik Fältström (editor).
% \newblock {The Unicode Code Points and Internationalized Domain Names for Applications (IDNA)}.
% \newblock RFC 5892 (Standards Track), August 2010.
% \newblock URL \url{http://tools.ietf.org/html/rfc5892}.

\end{thebibliography}


\end{document}
