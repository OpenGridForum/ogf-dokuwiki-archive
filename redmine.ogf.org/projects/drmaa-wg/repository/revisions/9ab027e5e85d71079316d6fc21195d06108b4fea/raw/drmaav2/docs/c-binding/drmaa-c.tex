% save trees by replacing OGF-mandated 12pt with 10pt
\documentclass{article} 

\sloppy

% OGF-defined document template stuff 
\usepackage{ifpdf}
\usepackage[utf8]{inputenc}
\usepackage{ifthen}
\usepackage{graphicx}
\usepackage[pdfborder={0 0 0}]{hyperref}
\usepackage{url}
\usepackage{authblk}
\usepackage[numbers]{natbib} 
\bibliographystyle{plainnat} 
\usepackage[sf,compact]{titlesec} 
\usepackage[titles]{tocloft}
\usepackage{parskip} 
\newcommand{\headerstyle}{\sffamily}
\usepackage{fancyhdr}
\addtolength{\headheight}{15pt}
\renewcommand{\headrulewidth}{0pt}
\setlength{\headsep}{20pt}
\usepackage[headings]{fullpage}  
\graphicspath{{img/}{./}}
\renewcommand{\cftsecfont}{\sffamily}
\renewcommand{\cftsubsecfont}{\sffamily}
\renewcommand{\cftsubsubsecfont}{\sffamily}
\renewcommand{\cftsecpagefont}{\sffamily}
\renewcommand{\cftsubsecpagefont}{\sffamily}
\renewcommand{\cftsubsubsecpagefont}{\sffamily}
\renewcommand{\cftsecleader}{\cftdotfill{\cftsubsecdotsep}} 
\setlength{\cftbeforesecskip}{0.5ex}
\newcommand{\ifnonempty}[2]{\ifthenelse{\isundefined{#1}}{}{\ifthenelse{\equal{#1}{}}{}{#2}}}
\pagestyle{fancyplain}
\fancyhf{}
\lhead{\fancyplain{}{\headerstyle\docseries}}
\rhead{\fancyplain{}{\headerstyle\ifthenelse{\isundefined{\revisiondate}}{\publicationdate}{\ifthenelse{\equal{\revisiondate}{}}{\publicationdate}{\revisiondate}}}}
\lfoot{\headerstyle\ifnonempty{\groupurl}{\groupurl}}
\rfoot{\headerstyle\thepage}
\thispagestyle{plain}

% OGF-defined meta data 
\title{Distributed Resource Management Application API Version 2 (DRMAA) -\\C Language Binding}  
\newcommand{\shortdoctitle}{DRMAA}  
\newcommand{\authorsshort}{Peter Tröger, Hasso Plattner Institute\footnotemark[1]\\Roger Brobst, Cadence Design Systems\\Daniel Gruber, Univa\\Mariusz Mamoński, PSNC\\Andre Merzky, LSU}  
\newcommand{\publicationdate}{April 2012}  
%\newcommand{\revisiondate}{December 2010}  % Optional: date of last revision of the document
\newcommand{\copyrightyears}{2012-2012}  
\newcommand{\docseries}{GWD-R}  % GWD-R, GWD-I or GWD-C (for working drafts), GFD-I, GFD-R, or GFD-C
\newcommand{\groupname}{DRMAA-WG} 
\newcommand{\groupurl}{\href{mailto:drmaa-wg@ogf.org}{drmaa-wg@ogf.org}}  
\newcommand{\documenturl}{\href{http://www.drmaa.org/}}

% Our additional Latex stuff
\usepackage{todonotes}
\usepackage{listings}
\usepackage{tabularx}
\lstset{language=C, tabsize=2, keywordstyle=\bfseries, basicstyle=\ttfamily\scriptsize,  rangeprefix=////,rangesuffix=////,includerangemarker=false}
\newcommand{\h}[1]{\texttt{#1}}
\setcounter{tocdepth}{2}
\usepackage{csquotes}

%%%%%%%% Enable this block for the annotaed version
%\newcommand{\rat}[1]{ {\tiny(See footnote)}\footnote{#1} }
%\interfootnotelinepenalty=10000
%\newcommand{\tbd}[2][] {\todo[caption={#2}, size=\small, #1]{\renewcommand{\baselinestretch}{0.5}\selectfont#2\par}}
%\usepackage{lineno}
%\linenumbers
%%%%%%% Enable this block for the official version
\newcommand{\rat}[1]{}
\newcommand{\tbd}[2][] {}

\begin{document}

% OGF-defined title page rendering 
{\noindent
\begin{minipage}[t]{3.0in}
\headerstyle
\docseries \\
\ifnonempty{\groupname}{\groupname \\}
\ifnonempty{\groupurl}{\groupurl \\}
\ifnonempty{\documenturl}{\documenturl \\}
\end{minipage}
\hfill
\raggedleft
\begin{minipage}[t]{3.0in}
\raggedleft
\headerstyle
\authorsshort \\
\publicationdate \\
\ifnonempty{\revisiondate}{Revised \revisiondate \\}
\end{minipage}
}

\begin{center}
\makeatletter
\Large\bf\textsf \@title
\makeatother
\end{center}

\subsection*{Status of This Document}

Group Working Draft - Proposed Recommendation (GWD-R)

\rat{
This is the non-normative annotated version of the specification with line numbers. It includes historical information concerning the content and why features were included or discarded by the working group. It also emphasizes the consequences of some aspects that may not be immediately apparent. This document in only intended for internal working group discussions.
}	


%%%%%%%%%%%%%%%%%%%%%%%%%%%%%%%%%%%%%%%%%%%%%%%%%%%%%%%%%%%%%%%%%%%%%%%%%%%%%%%%%%%%
%%% End of header, insert content below this line
%%%%%%%%%%%%%%%%%%%%%%%%%%%%%%%%%%%%%%%%%%%%%%%%%%%%%%%%%%%%%%%%%%%%%%%%%%%%%%%%%%%%

\subsection*{Document Change History}
\begin{table}[ht]
\centering
\begin{tabularx}{\textwidth}{|X|X|}
\hline
\emph{Date} & \emph{Notes} \\
\hline
April 26th, 2012 & Initial submission for public comment period \\
\hline
\end{tabularx}
\end{table}

\subsection*{Copyright Notice}

Copyright \copyright \ Open Grid Forum (\copyrightyears).  Some Rights Reserved.  
Distribution is unlimited.

\subsection*{Trademark}

All company, product or service names referenced in this document are used for identification purposes only and may be trademarks of their respective owners. 

\section*{Abstract}

This document describes the C language binding for the \emph{Distributed Resource Management Application API Version 2 (DRMAA)}. The intended audience for this specification are DRMAA Version 2 interface implementors. 

\footnotetext[1]{Corresponding author}
\newpage

\subsection*{Notational Conventions}
\label{sec:rfc2119}

In this document, C language elements and definitions are represented in a \h{fixed-width} font. 

The key words \enquote{MUST} \enquote{MUST NOT}, \enquote{REQUIRED}, \enquote{SHALL}, \enquote{SHALL NOT}, \enquote{SHOULD}, \enquote{SHOULD NOT}, \enquote{RECOMMENDED}, \enquote{MAY},  and \enquote{OPTIONAL} are to be interpreted as described in RFC 2119~\cite{rfc2119}. 

\newpage
\tableofcontents
\newpage

\section{Introduction}
\label{sec:introduction}

 The \emph{Distributed Resource Management Application API Version 2 (DRMAA)} specification defines an interface for tightly coupled, but still portable access to the majority of DRM systems. The scope is limited to job submission, job control, reservation management, and retrieval of job and machine monitoring information. 

The \emph{DRMAA root specification} \cite{gfd194} describes the abstract API concepts and the behavioral rules of a compliant implementation, while this document standardizes the representation of API concepts in the C programming language.

\section{General Design}
\label{sec:concepts}

The mapping of DRMAA IDL constructs to C follows a set of design principles. Implementation-specific extensions of the DRMAA C API described here SHOULD follow these conventions for their own naming and method signatures:

\begin{itemize}
\item Namespacing of the DRMAA API, as demanded by by the root specification, is realized with the \h{drmaa2\_} prefix for lower- and upper-case identifiers.
\item In identifier naming, "job" is shortened as "j" and "reservation" is shortened as "r" for improved readability.
\item The root specification demands a consistent parameter passing strategy for non-scalar values. In DRMAA for C, all such values are passed as call-by-reference parameter.
\item Structs and enums are typedef'ed for better readability.
\item Struct types get a \h{\_s} suffix on their name. Structures with a non-standardized layout are defined as forward references for the DRMAA implementation. \rat{This avoids the usage of void* pointers, f.e. with dictionaries and lists.} 
\item Functions with IDL return type \h{void} have \h{drmaa2\_error} as return type.
\item The IDL \h{boolean} type maps to the \h{drmaa2\_boolean} type.
\item The IDL  \h{long} type maps to \h{long long} in C. One exception is the \h{exitStatus} variable, which is defined as \h{int} in order to provide a more natural mapping to operating system interfaces.
\item The IDL \h{string} type maps to \h{char*} pointer. The memory for strings that are part of function results SHALL be allocated by the implementation itself. The application frees such memory regions by calling the newly introduced function \h{drmaa2\_string\_free}. Implementations MUST accept calls to \h{drmaa2\_string\_free} for all \h{char *} pointers. \rat{This means that even if the implementation returns string literal pointers at some occations, \h{drmaa2\_string\_free} SHALL not fail for this. This may be realized by avoiding string literal pointers at all, or by maintaining a list of malloc'ed pointers.}  
\item The language binding defines one UNSET macro per utilized C data type (\h{DRMAA2\_UNSET\_*}). \rat{For UNSET values, the language binding adheres mainly to typical language conventions and not to GLUE as mandated in the root spec.}  
\item  All numerical types are signed, in order to support \h{-1} as numerical UNSET value.
\item  Application-created structs should be allocated by the additional support methods (such as \h{drmaa2\_jinfo\_create}) to realize the neccessary initialization to UNSET. 
\item All structures have a specific support function for freeing them (\h{drmaa2\_*\_free}). \rat{The deallocation functions are needed to make sure that the allocating entity (the library) also performs the freeing operation. This is needed for cases where the DRMAA library is compiled with a different heap allocator than the DRMAA-based application. It is mainly a problem with Windows-based implementations.}
\item Both \h{AbsoluteTime} and \h{TimeAmount} map directly to \h{time\_t}. RFC 822 support as mandated by the root specification is given by the \h{\%z} formatter for \h{sprintf}.
\item Multiple output parameters are realized by declaring all but one of them as pointer variable. For this reason, the \h{substate} parameter in \h{drmaa2\_j\_get\_state} SHALL be interpreted as pointer to a character pointer variable. The DRMAA library creates the buffer and stores the pointer to it as variable value.
\item The \h{const} declarator is used to mark parameters declared as \h{readonly} in the root specification.
\item The two string list types in DRMAA, ordered and unordered, are mapped to one ordered list with the \h{DRMAA2\_STRING\_LIST} type.
\item The largest possible value for \h{end\_index} in \h{drmaa2\_js\_run\_bulk\_jobs} SHOULD be \h{ULONG\_MAX}.
\item The \h{any} member for job sub-state information is defined as \h{char*}, in order to achieve application portability. 
\end{itemize}

The following structures are only used in result values. For this reason, the according allocation functions are not part of the API:

\begin{itemize}
\item \h{drmaa2\_slotinfo}
\item \h{drmaa2\_rinfo}
\item \h{drmaa2\_notification}
\item \h{drmaa2\_queueinfo}
\item \h{drmaa2\_version}
\item \h{drmaa2\_machineinfo}
\end{itemize}

The interface membership of a function is sometimes expressed by an additional prefix, as shown in Table \ref{tab:naming}.

\begin{table}[ht]
\centering
\begin{tabularx}{\textwidth}{|X|X|l|X|}
\hline
DRMAA interface & C binding prefix \\
\hline
\h{DrmaaReflective} & \h{drmaa2\_} \\
\h{SessionManager} & \h{drmaa2\_} \\
\h{JobSession} & \h{drmaa2\_jsession\_} \\
\h{ReservationSession} & \h{drmaa2\_rsession\_} \\
\h{MonitoringSession}  & \h{drmaa2\_msession\_} \\
\h{Reservation} & \h{drmaa2\_r\_} \\
\h{Job} & \h{drmaa2\_j\_} \\
\h{JobArray} & \h{drmaa2\_jarray\_} \\
\h{JobTemplate} & \h{drmaa2\_jtemplate\_} \\
\h{ReservationTemplate} & \h{drmaa2\_rtemplate\_} \\
\hline
\end{tabularx}
\caption{Mapping of DRMAA interface name to C method prefix}
\label{tab:naming}
\end{table}

The C binding specifies the function pointer type \h{drmaa2\_callback} for a notification callback function. This represents the \h{DrmaaCallback} interface from the root specification. The new constant value \h{DRMAA2\_UNSET\_CALLBACK} can be used by the application for the de-registration of callback functions.

\subsection{Error Handling}

The list of exceptions in the DRMAA root specification is mapped to the new enumeration \h{drmaa2\_error}. The enumeration member \h{DRMAA2\_LASTERROR} is intended to ensure application portability while allowing additional implementation-specific error codes. It MUST always be the enumeration member with the highest value.

The language binding adds two new functions for fetching error number and error message of the last error that occurred: \h{drmaa2\_lasterror} and \h{drmaa2\_lasterror\_text}. These functions MUST operate in a thread-safe manner, meaning that both error informations are managed per application thread by the DRMAA implementation.  

\subsection{Lists and Dictionaries}

The C language binding adds generic support functions for the collection data types used by the root specification. The newly defined \h{drmaa2\_lasterror} and \h{drmaa2\_lasterror\_text} functions MUST return according error information for these operations.

\rat{The definition of list operations in the language binding keeps the application code portable. The original DRMAA error codes are good enough to support them, there is no need for additional ones. DRMAA dictionaries are only used for strings, so we make the dictionary interface less general.}  

Both \h{drmaa2\_list\_create} and \h{drmaa2\_dict\_create} have an optional parameter \h{callback}. It allows the application to provide a callback pointer to a collection element cleanup function. It MUST be allowed for the application to provide \h{DRMAA2\_UNSET\_CALLBACK} instead of a valid callback pointer.

\rat{This is again for the heap allocator problem. The entity allocating some memory should also free it.}

The following list operations are defined:

\begin{description}
\item[\h{drmaa2\_list\_create}:] Creates a new list instance for the specified type of items. Returns a pointer to the list or NULL on error.
\item[\h{drmaa2\_list\_free}:] Frees the list and the contained members and returns a success indication. If a callback function was provided on list creation, it SHALL be called once per list item.
\item[\h{drmaa2\_list\_get}:] Gets the list element at the indicated position. The element index starts at zero. If the index is invalid, the function returns NULL. 
\item[\h{drmaa2\_list\_add}:] Adds a new item at the end of the list and returns a success indication. The list MUST contain only the provided pointer, not a deep copy of the provided data structure. 
\item[\h{drmaa2\_list\_remove}:] Removes the list element at the indicated position and returns a success indication. If a callback function was provided on list creation, it SHALL be called before this function returns.  
\item[\h{drmaa2\_list\_size}:] Gets the number of elements in the list. If the list is empty, then the function returns 0, which SHALL NOT be treated as an error case.
\end{description}

Similarly, a set of new functions for dictionary handling is introduced:

\begin{description}
\item[\h{drmaa2\_dict\_create}:] Creates a new dictionary instance. Returns a pointer to the dictionary or NULL on error.
\item[\h{drmaa2\_dict\_free}:] Frees the dictionary and the contained members and returns a success indication. If a callback function was provided on dictionary creation, it SHALL be called once per dictionary entry.
\item[\h{drmaa2\_dict\_list}:] Gets all dictionary keys as DRMAA \h{drmaa2\_string\_list}. If the dictionary is empty, a valid string list with zero elements SHALL be returned. The application is expected to use \h{drmaa2\_list\_free} for freeing the returned data structure.  
\item[\h{drmaa2\_dict\_has}:] Returns a boolean indication if the given key exists in the dictionary. On error, the function SHALL return FALSE. 
\item[\h{drmaa2\_dict\_get}:] Gets the dictionary value for the specified key. If the key is invalid, the function returns NULL. 
\item[\h{drmaa2\_dict\_del}:] Removes the dictionary entry with the given key and returns a success indication. If a callback function was provided on dictionary creation, it SHALL be called before this function returns. 
\item[\h{drmaa2\_dict\_set}:] Sets the specified dictionary key to the specified value. Key and value strings MUST be stored as the provided character pointers. If the dictionary already has an entry for this name, the value is replaced and the old value is removed. If a callback was provided on dictionary creation, it SHALL be called with a NULL pointer for the key and the pointer of the previous value. 
\end{description}

\section{Implementation-specific Extensions}

The DRMAA root specification allows the product-specific extension of the DRMAA API in a standardized way. 

New methods added to a DRMAA implementation SHOULD follow the conventions from Section \ref{sec:concepts}. Extended \h{struct} definitions SHOULD use a product-specific prefix for a clear separation of non-portable and portable parts of the API. The extension MUST support the casting of product-specific \h{struct} pointers to their standard-compliant counterparts (see Listing \ref{lst:extension}). Any compiler or linking options necessary for this feature MUST be documented accordingly by the DRMAA implementation.

\begin{lstlisting}[caption=Code example for implementation-specific extension,label=lst:extension]
typedef struct  
{
     [attributes from drmaa2_jtemplate_s] ...
     int gridengine_specific_attr;
} gridengine_jtemplate_s;
typedef gridengine_jtemplate_s * gridengine_jtemplate; 
\end{lstlisting}

\newpage

\section{Complete Header File}
\label{sec:idl}

The following text shows the complete C header file for the DRMAAv2 application programming interface. DRMAA-compliant C libraries MUST declare all functions and data structures described here. Implementations MAY add custom parts in adherence to the extensibility principles of this specification and the root specification.

The source file is also available at \url{http://www.drmaa.org}.

\lstinputlisting{drmaa2.h}

\section{Security Considerations}
\label{sec:security}

The DRMAA root specification \cite{gfd194}  describes the behavioral aspects of a standard-compliant implementation. This includes also security aspects.

Software written in C language has well-known security attack vectors, especially with memory handling. Implementors MUST clarify in their documentation which kind of memory management is expected by the application. Implementations MUST also consider the possibility for multi-threaded applications performing re-entrant calls to the library. The root specification clarifies some of these scenarios.

\section{Contributors}

\textbf{Roger Brobst}\\
Cadence Design Systems, Inc.\\
555 River Oaks Parkway \\
San Jose, CA 95134, United States\\
Email: rbrobst@cadence.com\\  

\textbf{Daniel Gruber}\\
Univa GmbH\\
c/o Rüter und Partner\\
Prielmayerstr. 3\\
80335 München, Germany\\
Email: dgruber@univa.com\\

\textbf{Mariusz Mamoński}\\
Poznań Supercomputing and Networking Center\\
ul. Noskowskiego 10\\
61-704 Poznań, Poland\\
Email: mamonski@man.poznan.pl\\ 

\textbf{Andre Merzky}\\
Center for Computation and Technology\\
Louisiana State University\\
216 Johnston Hall\\
70803 Baton Rouge,  Louisiana, USA\\
Email:  andre@merzky.net\\ 

\textbf{Peter Tröger (Corresponding Author)} \\
Hasso Plattner Institute at University of Potsdam \\
Prof.-Dr.-Helmert-Str. 2-3 \\
14482 Potsdam, Germany \\
Email: peter@troeger.eu \\

Special thanks go to \emph{Stefan Klauck (Hasso Plattner Institute)} for the DRMAA C binding reference implementation and the  debugging of the implementation-related language binding issues. 

%%%%%%%%%%%%%%%%%%%%%%%%%%%%%%%%%%%%%%%%%%%%%%%%%%%%%%%%%%%%%%%%%%%%%%%%%%%%%%%%%%%%
%%% Insert content above this line
%%%%%%%%%%%%%%%%%%%%%%%%%%%%%%%%%%%%%%%%%%%%%%%%%%%%%%%%%%%%%%%%%%%%%%%%%%%%%%%%%%%%

\section{Intellectual Property Statement}

The OGF takes no position regarding the validity or scope of any intellectual property or other rights that might be claimed to pertain to the implementation or use of the technology described in this document or the extent to which any license under such rights might or might not be available; neither does it represent that it has made any effort to identify any such rights.  Copies of claims of rights made available for publication and any assurances of licenses to be made available, or the result of an attempt made to obtain a general license or permission for the use of such proprietary rights by implementers or users of this specification can be obtained from the OGF Secretariat.

The OGF invites any interested party to bring to its attention any copyrights, patents or patent applications, or other proprietary rights which may cover technology that may be required to practice this recommendation.  Please address the information to the OGF Executive Director.

\section{Disclaimer}

This document and the information contained herein is provided on an \enquote{as-is} basis and the OGF disclaims all warranties, express or implied, including but not limited to any warranty that the use of the information herein will not infringe any rights or any implied warranties of merchantability or fitness for a particular purpose.

\section{Full Copyright Notice}

Copyright \copyright \ Open Grid Forum (\copyrightyears). Some Rights Reserved.

This document and translations of it may be copied and furnished to others, and derivative works that comment on or otherwise explain it or assist in its implementation may be prepared, copied, published and distributed, in whole or in part, without restriction of any kind, provided that the above copyright notice and this paragraph are included on all such copies and derivative works. However, this document itself may not be modified in any way, such as by removing the copyright notice or references to the OGF or other organizations, except as needed for the purpose of developing Grid Recommendations in which case the procedures for copyrights defined in the OGF Document process must be followed, or as required to translate it into languages other than English.

The limited permissions granted above are perpetual and will not be revoked by the OGF or its successors or assignees.

% \phantomsection\addcontentsline{toc}{section}{References}
\section{References}
\renewcommand{\refname}{}
\vspace*{-3em}
\bibliography{bibliography}
\end{document}

